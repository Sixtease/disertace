\noindent Název práce: Mluven\'{y} korpus Karla Makon\v{e}\\
Autor: Mgr. Jan Oldřich Krůza\\
Katedra (ústav): Ústav formální a aplikované lingvistiky\\
Vedoucí diplomové práce: Doc. Vladislav Kuboň\\
E-mail vedoucího: Vladislav.Kubon@mff.cuni.cz\\
Konzultant: Mgr. Nino Peterek, Ph.D.\\
E-mail konzultanta: Nino.Peterek@mff.cuni.cz\\

\noindent Abstrakt: Tato disertační práce se zabývá zpřístupněním zvukových
záznamů jednoho mluvčího úzké i široké veřejnosti.

Motivací práce byla existence chátrajících nahrávek hovorů českého mystika
Karla Makoně na kazetách a kotoučích. Cílem je zachování materiálu pro
budoucí generace a zpřístupnění nahrávek pomocí digitálních technologií.
Především přístpnost nahrávek na internetu a možnost vyhledávání v nich.

V~práci se popisuje samotný mluvený korpus. Představí se autor a
jeho dílo,
témata v~nahrávkách, nahrávání samotné a jejich digitalizace. Dále se rozvede tvorba systému pro automatický
přepis korpusu od akustického modelování, přes jazykové modelování, různé
provedené experimenty až k~vyhodnocení úspěšnosti. V~neposlední řadě se popíše
webová aplikace, která jednak umožňuje návštěvníkům
poslech nahrávek a jednak slouží pro sběr korekcí k~automatickým přepisům od
laických uživatelů. Tyto korekce se využívají jako trénovací
data pro akustický a jazykový model.

\vspace{10mm}

\noindent
Title: Spoken Corpus of Karel Makoň\\
Author: Mgr. Jan Oldřich Krůza\\
Department: Institute of Formal and Applied Linguistics\\
Supervisor: Doc. Vladislav Kuboň\\
Supervisor's e-mail address: Vladislav.Kubon@mff.cuni.cz\\
Consultant: Mgr. Nino Peterek, Ph.D.\\
Consultant's e-mail address: Nino.Peterek@mff.cuni.cz\\

\noindent Abstract:  This Ph.D. thesis deals with making a corpus of audio
recordings of a single speaker accessible to wide public and interested community.

The work has been motivated by the existence of a set of perishing recordings
of the Czech mystic Karel Makoň on magnetophone tapes. The aim is to conserve
the material for future generations and making them accessible using
digital technologies. In particular publishing the recordings online
and enabling the users to search through them.

The thesis elaborates on the spoken corpus itself. The
author and his works, topics covered in the talks, the process of recording
and digitization are introduced. Next, the development of a system for automated transcription of
the corpus, from acoustic modeling, to language modeling, various experiments
undertaken and evaluation is presented. Then, a web application is described,
that on one hand enables visitors to listen to the recordings and on the other
hand serves for gathering corrections to the automatically acquired transcripts from lay users.
These corrections are then used as training data for the acoustic and language
model.
