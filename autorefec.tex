%%% Hlavní soubor. Zde se definují základní parametry a odkazuje se na ostatní části. %%%

%% Verze pro jednostranný tisk:
% Okraje: levý 40mm, pravý 25mm, horní a dolní 25mm
% (ale pozor, LaTeX si sám přidává 1in)
\documentclass[12pt,a4paper]{report}
\setlength\textwidth{145mm}
\setlength\textheight{247mm}
\setlength\oddsidemargin{15mm}
\setlength\evensidemargin{15mm}
\setlength\topmargin{0mm}
\setlength\headsep{0mm}
\setlength\headheight{0mm}
% \openright zařídí, aby následující text začínal na pravé straně knihy
\let\openright=\clearpage

%% Pokud tiskneme oboustranně:
% \documentclass[12pt,a4paper,twoside,openright]{report}
% \setlength\textwidth{145mm}
% \setlength\textheight{247mm}
% \setlength\oddsidemargin{14.2mm}
% \setlength\evensidemargin{0mm}
% \setlength\topmargin{0mm}
% \setlength\headsep{0mm}
% \setlength\headheight{0mm}
% \let\openright=\cleardoublepage

%% Vytváříme PDF/A-2u
\usepackage[a-2u]{pdfx}

%% Přepneme na českou sazbu a fonty Latin Modern
\usepackage[czech]{babel}
\usepackage{lmodern}
\usepackage[T1]{fontenc}
\usepackage{textcomp}

%% Použité kódování znaků: obvykle latin2, cp1250 nebo utf8:
\usepackage[utf8]{inputenc}

%%% Další užitečné balíčky (jsou součástí běžných distribucí LaTeXu)
\usepackage{amsmath}        % rozšíření pro sazbu matematiky
\usepackage{amsfonts}       % matematické fonty
\usepackage{amsthm}         % sazba vět, definic apod.
\usepackage{bbding}         % balíček s nejrůznějšími symboly
			    % (čtverečky, hvězdičky, tužtičky, nůžtičky, ...)
\usepackage{bm}             % tučné symboly (příkaz \bm)
\usepackage{graphicx}       % vkládání obrázků
\usepackage{fancyvrb}       % vylepšené prostředí pro strojové písmo
\usepackage{indentfirst}    % zavede odsazení 1. odstavce kapitoly
\usepackage{natbib}         % zajištuje možnost odkazovat na literaturu
			    % stylem AUTOR (ROK), resp. AUTOR [ČÍSLO]
\usepackage[nottoc]{tocbibind} % zajistí přidání seznamu literatury,
                            % obrázků a tabulek do obsahu
\usepackage{icomma}         % inteligetní čárka v matematickém módu
\usepackage{dcolumn}        % lepší zarovnání sloupců v tabulkách
\usepackage{booktabs}       % lepší vodorovné linky v tabulkách
\usepackage{paralist}       % lepší enumerate a itemize
\usepackage{xcolor}         % barevná sazba

\usepackage{rotating}
%\usepackage{fontspec}
\usepackage{tipa}
\usepackage{enumitem}
\usepackage{makecell}
\usepackage{alltt}
\usepackage{refcount}
\usepackage{tabularx}
\usepackage{caption}


%%% Údaje o práci

% Název práce v jazyce práce (přesně podle zadání)
\def\NazevPrace{Iterativní zdokonalování přepisu zvukových nahrávek s~využitím zpětné vazby posluchačů}

% Název práce v angličtině
\def\NazevPraceEN{Iterative Improving of Transcribed Speech Recordings Exploiting Listener's Feedback}

% Jméno autora
\def\AutorPrace{Mgr. Jan Oldřich Krůza}

% Rok odevzdání
\def\RokOdevzdani{2020}

% Název katedry nebo ústavu, kde byla práce oficiálně zadána
% (dle Organizační struktury MFF UK, případně plný název pracoviště mimo MFF)
\def\Katedra{Ústav formální a aplikované lingvistiky}
\def\KatedraEN{Institute of Formal and Applied Linguistics}

% Jedná se o katedru (department) nebo o ústav (institute)?
\def\TypPracoviste{Ústav}
\def\TypPracovisteEN{Institute}

% Vedoucí práce: Jméno a příjmení s~tituly
\def\Vedouci{Doc. RNDr. Vladislav Kuboň, Ph.D.}

% Pracoviště vedoucího (opět dle Organizační struktury MFF)
\def\KatedraVedouciho{
  Ústav formální a aplikované lingvistiky\\
  Matematicko-fyzikální fakulta\\
  Univerzita Karlova\\
  Malostranské náměstí 25\\
  11800 Praha 1
}
\def\KatedraVedoucihoEN{Institute of Formal and Applied Linguistics}

% Studijní program a obor
\def\StudijniProgram{informatika}
\def\StudijniObor{matematická lingvistika}

% Nepovinné poděkování (vedoucímu práce, konzultantovi, tomu, kdo
% zapůjčil software, literaturu apod.)

% Abstrakt (doporučený rozsah cca 80-200 slov; nejedná se o zadání práce)
\def\Abstrakt{%
Tato disertační práce se zabývá zpřístupněním zvukových
záznamů jednoho mluvčího úzké i široké veřejnosti.

Motivací práce byla existence chátrajících nahrávek hovorů českého filozofa
ing. Karla Makoně na kazetách a kotoučích. Cílem je zachování materiálu pro
budoucí generace a zpřístupnění nahrávek pomocí digitálních technologií,
především přístupnosti nahrávek na internetu a možnosti vyhledávání v~nich.

Práce představuje tvorbu systému pro přepis velké sady zvukových záznamů
se zapojením laické komunity. Navržené řešení spočívá ve vytvoření základního
přepisu nízké kvality pomocí automatického rozpoznávání řeči a vyvinutí
aplikace, která umožní od členů komunity i nahodilých zájemců získávat opravy
automatického přepisu, použitelné jako trénovací data pro další zlepšování.

Popíše se samotný mluvený korpus. Představí se autor a
jeho dílo,
témata v~nahrávkách, nahrávání samotné, digitalizace a získané přepisy.
Dále se rozvede tvorba systému pro automatický
přepis korpusu od sběru dat, přes akustické a jazykové modelování, různé
provedené experimenty až k~vyhodnocení úspěšnosti. V~neposlední řadě se popíše
webová aplikace pro sběr manuálních přepisů. Zmíní se odlišnosti od ostatních
systémů, detaily návrhu a řešení, mechanismus pro kompenzaci vysokých nároků na kvalitu
přepisu a nízkých nároků na odbornost přispěvatelů a vyhodnocení funkčnosti
po osmi letech provozu.
}
\def\AbstraktEN{%
This Ph.D. thesis deals with making a corpus of audio
recordings of a single speaker accessible to wide public and interested community.

The work has been motivated by the existence of a set of perishing recordings
of the Czech philosopher Karel Makoň on magnetophone tapes. The aim is to conserve
the material for future generations and making it accessible using
digital technologies, in particular publishing the recordings online
and enabling the users to search through them.

The thesis introduces the creation of a system for transcribing a large set of
speech recordings employing a lay community. The solution designed is based on
obtaining a baseline low-quality transcription by means of automated speech
recognition and developing an application that allows for collecting corrections
of the automatic transcription in a fashion that makes it usable as training
data for further improvement of said transcription.

The spoken corpus itself is described. The
author and his works, topics covered in the talks, the process of recording
and digitization as well as the gained transcription are introduced.
Next, the development of a system for automated transcription of
the corpus, from collecting data, to acoustic and language modeling, various experiments
undertaken and evaluation are presented. Then, the web application for gathering
manual transcript corrections is described.
Differences to other settings, design and implementation details, a way to
compensate high demand for transcription quality and low demand for worker
expertise, as well as an evaluation of the system's performance after eight
years of operation are covered.
}

% 3 až 5 klíčových slov (doporučeno), každé uzavřeno ve složených závorkách
\def\KlicovaSlova{%
{přepis zvukových nahrávek} {uživatelská interakce} {komunitní spolupráce}
}
\def\KlicovaSlovaEN{%
{speech transcription} {user inetraction} {community cooperation}
}

%% Balíček hyperref, kterým jdou vyrábět klikací odkazy v PDF,
%% ale hlavně ho používáme k uložení metadat do PDF (včetně obsahu).
%% Většinu nastavítek přednastaví balíček pdfx.
\hypersetup{unicode}
\hypersetup{breaklinks=true}

%% Definice různých užitečných maker (viz popis uvnitř souboru)
\include{makra}

%% Titulní strana a různé povinné informační strany
\begin{document}
%%% Titulní strana práce a další povinné informační strany

%%% Titulní strana práce

\pagestyle{empty}
\hypersetup{pageanchor=false}

\begin{center}

\centerline{\mbox{\includegraphics[width=166mm]{../img/logo-cs.pdf}}}

\vspace{-8mm}
\vfill

{\bf\Large ABSTRACT OF DOCTORAL THESIS}

\vfill

{\LARGE\AutorPrace}

\vspace{15mm}

{\LARGE\bfseries\NazevPraceEN}

\vfill

\KatedraEN

\vfill

{
\centerline{\vbox{\halign{\hbox to 0.45\hsize{\hfil #}&\hskip 0.5em\parbox[t]{0.45\hsize}{\raggedright #}\cr
Supervisor:&\Vedouci \cr
\noalign{\vspace{2mm}}
Study programme:&Computer science \cr
\noalign{\vspace{2mm}}
Branch:& Computational linguistics\cr
}}}}

\vfill

% Zde doplňte rok
Praha \RokOdevzdani

\end{center}

\newpage

%%% Povinná informační strana disertační práce

\openright

\vbox to 0.5\vsize{
\setlength\parindent{0mm}
\setlength\parskip{5mm}

The results of this thesis were achieved in the pediod of a doctoral study at
the Faculty of Mathematics and Physics, Charles University in years
2011 -- 2020.

\vspace{2cm}

{
\centerline{\vbox{\halign{\hbox to 0.45\hsize{\hfil #}&\hskip 0.5em\parbox[t]{0.45\hsize}{\raggedright #}\cr
Student:&
\AutorPrace
\cr \noalign{\vspace{7mm}}
Supervisor:&
\Vedouci, \KatedraVedoucihoEN
\cr \noalign{\vspace{7mm}}
Department:&
\KatedraVedoucihoEN
\cr \noalign{\vspace{7mm}}
Opponents:&
Prof. Ing. Luděk Müller, Ph.D.\\
Fakulta aplikovaných věd ZČU, Katedra kybernetiky\\
Technická 8, 30614 Plzeň
\cr \noalign{\vspace{7mm}}
&Doc. Ing. Petr Pollák, CSc.\\
ČVUT FEL K13131,\\
Technická 2, 16627 Praha 6
\cr
}}}}

The thesis defense will take place on September 30th 2020 at 11:00 a.m. in front of a
committee for thesis defenses in the branch P4I3 Computational linguistics
at the Faculty of Mathematics and Physics, Charles University in
Prague, Malostranské náměstí 25, in the room S510.

The thesis can be viewed at the Study Department of Doctoral Studies of the Faculty of
Mathematics and Physics, Charles University in Prague, Ke Karlovu 3, Prague 2.

This abstract has been distributed on September 23rd 2020.

\vss}\nobreak\vbox to 0.49\vsize{
\setlength\parindent{0mm}
\setlength\parskip{5mm}

\vss}

%%% Titulní strana práce

\pagestyle{empty}
\hypersetup{pageanchor=false}

\begin{center}

\centerline{\mbox{\includegraphics[width=166mm]{../img/logo-cs.pdf}}}

\vspace{-8mm}
\vfill

{\bf\Large AUTOREFERÁT DISERTAČNÍ PRÁCE}

\vfill

{\LARGE\AutorPrace}

\vspace{15mm}

{\LARGE\bfseries\NazevPrace}

\vfill

\Katedra

\vfill

{
\centerline{\vbox{\halign{\hbox to 0.45\hsize{\hfil #}&\hskip 0.5em\parbox[t]{0.45\hsize}{\raggedright #}\cr
Vedoucí disertační práce:&\Vedouci \cr
\noalign{\vspace{2mm}}
Studijní program:&\StudijniProgram \cr
\noalign{\vspace{2mm}}
Studijní obor:&\StudijniObor \cr
}}}}

\vfill

% Zde doplňte rok
Praha \RokOdevzdani

\end{center}

\newpage

%%% Povinná informační strana disertační práce

\openright

\vbox to 0.5\vsize{
\setlength\parindent{0mm}
\setlength\parskip{5mm}

Disertační práce byla vypracována na základě výsledků získaných během
doktorského studia na Matematicko-fyzikální fakultě Univerzity Karlovy v~letech
2011 -- 2020.


{
\centerline{\vbox{\halign{\hbox to 0.45\hsize{\hfil #}&\hskip 0.5em\parbox[t]{0.45\hsize}{\raggedright #}\cr
Doktorand:&
\AutorPrace
\cr \noalign{\vspace{7mm}}
Školitel:&
\Vedouci, \KatedraVedouciho
\cr \noalign{\vspace{7mm}}
Školicí pracoviště:&
\KatedraVedouciho
\cr \noalign{\vspace{7mm}}
Oponenti:&
Prof. Ing. Luděk Müller, Ph.D.\\
Fakulta aplikovaných věd ZČU, Katedra kybernetiky\\
Technická 8, 30614 Plzeň
\cr \noalign{\vspace{7mm}}
&Doc. Ing. Petr Pollák, CSc.\\
ČVUT FEL K13131,\\
Technická 2, 16627 Praha 6
\cr
}}}}




Obhajoba disertační práce se koná dne 30. září 2020 v 11:00 před komisí pro
obhajoby disertačních prací v oboru P4I3 - Matematická lingvistika na
Matematicko-fyzikální fakultě UK, Malostranské náměstí 25, v~místnosti S510.

S~disertační prací je možno se seznámit na studijním oddělení
Matematicko-fyzikální fakulty UK, Ke Karlovu 3, Praha 2.

Autoreferát byl rozeslán dne 23. září 2020.

\vss}\nobreak\vbox to 0.49\vsize{
\setlength\parindent{0mm}
\setlength\parskip{5mm}

\vss}

\newpage

%%% Povinná informační strana disertační práce



\openright
\pagestyle{plain}
\pagenumbering{arabic}
\setcounter{page}{1}

%%% Strana s automaticky generovaným obsahem disertační práce

\tableofcontents

%%% Jednotlivé kapitoly práce jsou pro přehlednost uloženy v samostatných souborech

\chapter{Úvod}

V~situacích, kde jsou k~dispozici záznamy mluveného slova, je často žádoucí
jejich přepis, protože zpracování digitálního textu je mnohem snazší než
zpracování digitálního audia. Manuální přepisy jsou nákladné a žádný systém
automatického přepisu není dokonalý, takže manuální revize výstupu rozpoznávání
řeči jsou běžným postupem.

V~mém případě, kde je k~dispozici sbírka nahrávek bez katalogizace a komunita
laických dobrovolníků, vyvstává potřeba systému, který by využil moderní
technologie rozpoznávání řeči a který by pečlivě zapojil cennou mozkovou
kapacitu zúčastněných osob.

Na začátku jsem měl k~dispozici analogové audio, ne digitální: Archiv nahrávek českého filozofa
Karla Makoně na kazetách a kotoučích. Mým cílem, který tuto práci motivoval, je
tento archiv zdigitalizovat, zpřístupnit a umožnit jeho další zpracování.


%\chapter{Úvod}
\label{kap:uvod}

TODO

\chapter{Data}
\label{kap:data}

Zvukový odkaz Karla Makoně je původní a ústřední motivací pro tuto práci. Osobně
považuji Makoňovo dílo za jedno z~nejzásadnějších vůbec v~oblasti duchovního
průkopnictví, a to jeho systematičností, obsáhlostí, návodností, novátorstvím a
především hloubkou. Jeho nauka se od moderních duchovních směrů odlišuje kladným
postojem k~civilizačnímim trendům, nikoliv jejich zavrhováním, dále
konzistentním souladem s~rozumovým poznáním a pevnými základy v~náboženských
tradicích. Od vědeckého bádání se odlišuje zejména tím, že rozum a hmotu
považuje za odrazové můstky k~hlubšímu poznání, nikoliv za vrchol a jedinou
platformu lidského poznání. Trvá však na tom, že duchovní zákonitosti jsou
stejně tak pevně dané, univerzální a ověřitelné (ovšem pouze osobní, subjektivní
zkušeností), jako zákony přírodní, popsané vědecky. Do třetice od klasické
křesťanské literatury se liší obzvláště tím, že Ježíšovu nauku považuje za návod
prvotřídní kvality pro vědomý vstup do věčného života zde na zemi a v~těle,
nikoliv po smrti. Věčným životem se míní stav, kdy člověk je vědomě věčnou
bytostí nezávislou na pomíjejícím těle. Tvrdě kritizuje překonaný a naivní
výklad, podle nějž se ctnostným životem dá dojít po smrti do nebe, dále dojít
spásy pouhou proklamací o~víře v~Krista a dodržováním přikázání a náboženských
obřadů.

Odkrývá smysl života a návod na jeho uskutečnění, který nestojí na slepé víře
ani na vlasní omezené lidské invenci.

\section{Karel Makoň}

\subsection{Duchovní vzestup Karla Makoně}

Ing. Karel Makoň se narodil 12. prosince 1912. Ve věku dvou let ho postihl zánět
levého ramene. Lékaři doporučovali amputaci ruky, k~čemuž jeho matka nedala
souhlas a na vlastní zodpovědnost nechala dítě operovat. Vzhledem k~tomu, že
ještě nebyly objeveny krevní skupiny, nebyla možná transfúze a proto musela být
operace prováděna opakovaně, aby dítě nevykrvácelo. Tehdejší anestetika nebylo
možné tak často podat tak mladému organizmu, proto byly operace prováděny při
vědomí. Malý Karel Makoň se zažívaje nesnesitelnou bolest naučil v~raném věku
opouštět svoje tělo. Tato opakovaná zkušenost měla u něho následek, že po
určitou dobu nepoznával svoji matku, zato začal spontánně rozpoznávat správné od
nesprávného a důsledně činit, co poznával jako správné.

Pro omezení rizika komplikací s~operovanou rukou bylo Karlovi zakázáno hrát si
s~dětmi. Svůj předškolní čas proto trávil sám na venkově jen se zvířaty. Díky
extrakorporálním zkušenostem se naučil rozumět řeči zvířat, obzvláště hus, které
ho vzaly za svého a s~nimiž pronikal do stavu zvířecího ráje.

Období ,,činění správného'', kdy si kupříkladu zapověděl kouření, alkohol i
veškerý pohlavní život, vyvrcholilo v~Makoňových sedmnácti letech, kdy si
přečetl myšlenku, že ,,tento život je mostem do věčnosti''. Tím započalo období
extází a vědomí, že je nesmrtelnou bytostí a smyslem jeho života je spojení
s~Bohem. Svojí matkou a prarodiči byl sice veden ke tradiční katolické víře, ale
nikdy na ni nepřistoupil, protože ,,v~nebi, kde by se jen díval na Boží tvář by
byla strašná nuda''. Nikdy tedy nevěřil, v~sedmnácti letech poznal.

V~tomto období se ustavičně modlil za to, aby dokázal Boha více milovat. Tato
modlitba trvala devět let a jejím vyvrcholením byla deportace do koncentračního
tábora v~Sachsenhausen v~roce 1939, coby českého vysokoškolského studenta.

V~koncentračním táboře byl Makoň sužován více, než ostatní: měl jakýsi
obzvláštní talent chytat rány a kopance. Prožíval nesmírné zmatení a frustraci
nad tím, že tak dlouho tak věrně sloužil Bohu, a teď se s~ním jedná jako s~kusem
hadru. Po čtyřech dnech utrpení nastal zlomový okamžik. Tamějším vězňům bylo
zakázáno pod trestem smrti přihlížet zabití spoluvězňů příslušníky SS. Makoň si
však nedal pozor a hleděl na právě takovou scénu. Vykonávající Němec si toho
povšiml a vyzval Karla Makoně, ať zůstane stát na místě, že hned, jak dobije
svoji momentální oběť, přijde zabít i jeho. Karel Makoň v~tu chvíli raději
odevzdal svůj život Bohu, a to bez přemýšlení a bezpodmínečně, se silou, nabytou
onou devítiletou modlitbou. Překvapivým výsledkem toho bylo, že SS-Mann tváří
v~tvář Makoňovi zbledl a v~hrůze se obrátil na útěk.

Makoň tehdy obdržel všeobjímající poznání smyslu života, jak svého, tak obecně,
a absolutní svobodu. Pohyboval se volně od baráku k~baráku, nezažíval hlad ani
jiný nedostatek, esesáci ho neviděli. Trávil svoje dny v~koncentračním táboře
burcováním ostatních k~probuzení k~pravdě, kterou sám zažíval.

Zanedlouho byl z~koncentračního tábora propuštěn a celý zbytek svého života
věnoval předávání svojí zkušenosti a hlavně návodu, jak k~ní přijít bez nutnosti
tak výrazných krizí, jakými sám v~životě procházel, a které považuje za
nepříkladné.

Zemřel v~roce 1993.

\subsection{Spisovatelská a přednášková činnost}

% získávání přepisů
%  iniciální přepis v Praatu nebo to byl Transcriber?
%  graf přibývání podle přispěvatelů, času, nahrávek atp.
% nahrávky samotné (z předchozích článků)
% grafické indexy

\chapter{Akustické vlastnosti Makoňova korpusu}
\label{kap:akustika}

Mluvený korpus Karla Makoně vyniká vzhledem ke své velikosti konzistencí téměř
výlučně jediného mluvčího a velmi úzkou tematickou doménou. Jistou protiváhu
této konsistentnosti představují jeho akustické vlastnosti.

\section{Výchozí akustická kvalita}

Akustická kvalita nahrávek je největší slabinou korpusu. Kvalita není
konzistentně špatná, je velmi kolísavá. Na kvalitu záznamu má vliv jeho stáří,
použité médium, rychlost záznamu, způsob skladování, použitý
magnetofon, mikrofon, pozice mikrofonu, akustické vlastnosti prostředí jako
ozvěna, hluk na pozadí, momentální dispozice mluvčího a také to, zda se jedná o
původní nahrávku nebo její kopii\footnote{Kopírování magnetofonových pásek je ztrátový proces.}.

Obrázky~\ref{fig:spectr-ok} až~\ref{fig:spectr-fasttalk} ukazují spektrogramy
nahrávek různých kvalit. Vždy se jedná přibližně o třísekundový úsek a součástí
popisku je odkaz pro přehrání odpovídajícího zvuku.

\begin{figure}[htpb]
\includegraphics[scale=0.89]{rc/spectrum-dobry-90-02A.png}
\caption{
    Kvalitní záznam bez zjevných defektů.\\
    \texttt{http://radio.makon.cz/zaznam/90-02A\#ts=673.14}
}
\label{fig:spectr-ok}
\end{figure}

\begin{figure}[htpb]
\includegraphics[scale=0.89]{rc/spectrum-echo-90-24A.png}
\caption{
    Výrazné echo.\\
    \texttt{http://radio.makon.cz/zaznam/90-24A-24.4.90\#ts=664.33}
}
\label{fig:spectr-echo}
\end{figure}

\begin{figure}[htpb]
\includegraphics[scale=0.89]{rc/spectrum-noise-92-04A.png}
\caption{
    Širokopásmový šum.\\
    \texttt{http://radio.makon.cz/zaznam/92-04A\#ts=691.37}
}
\label{fig:spectr-noise}
\end{figure}

\begin{figure}[htpb]
\includegraphics[scale=0.89]{rc/spectrum-narrow-92-03B.png}
\caption{
    Úzkopásmový šum.\\
    \texttt{http://radio.makon.cz/zaznam/92-03B\#ts=664.43}
}
\label{fig:spectr-narrow}
\end{figure}

\begin{figure}[htpb]
\includegraphics[scale=0.89]{rc/spectrum-nohighs-88-04A.png}
\caption{
    Absence vysokých frekvencí.\\
    \texttt{http://radio.makon.cz/zaznam/88-04A\#ts=678.94}
}
\label{fig:spectr-nohi}
\end{figure}

%\begin{figure}[htpb]
%\includegraphics[scale=0.89]{rc/spectrum-overdrive-91-20A.png}
%\caption{
%    Přebuzení.
%    \texttt{http://radio.makon.cz/zaznam/91-20A\#ts=600.00}
%}
%\label{fig:spectr-overdrive}
%\end{figure}

\begin{figure}[htpb]
\includegraphics[scale=0.89]{rc/spectrum-accel-90-18A.png}
\caption{
    Zrychlený záznam způsobený zpomalením převíjení pásky při nahrávání.\\
    \texttt{http://radio.makon.cz/zaznam/90-18A-XX-zrychlene\#ts=2473.56}
}
\label{fig:spectr-accel}
\end{figure}

\begin{figure}[htpb]
\includegraphics[scale=0.89]{rc/spectrum-2cms-ktplzneid01a.png}
\caption{
    Silně degradovaná nahrávka pořízená rychlostí 2,38 cm/s.\\
    \texttt{http://radio.makon.cz/zaznam/kotouc-plzen-neident01-a\#ts=660}
}
\label{fig:spectr-2cms}
\end{figure}

\begin{figure}[htpb]
\includegraphics[scale=0.89]{rc/spectrum-pomala-mluva-76-04A.png}
\caption{
    Pomalá mluva.\\
    \texttt{http://radio.makon.cz/zaznam/76-04A-Kaly-7-IEOUA\#ts=13.79}
}
\label{fig:spectr-slowtalk}
\end{figure}

\begin{figure}[htpb]
\includegraphics[scale=0.89]{rc/spectrum-rychla-mluva-89-11B.png}
\caption{
    Rychlá mluva.\\
    \texttt{http://radio.makon.cz/zaznam/89-11B\#ts=203.17}
}
\label{fig:spectr-fasttalk}
\end{figure}

Systematicky můžeme újmu na kvalitě rozdělit takto:
\begin{enumerate}
\item{
    aditivní šum, jevící se jako hučení až syčení, obzvlášť patrný v~tichých
    pasážích,
}
\item{
    stacionární nebo téměř stacionární rušení, například monotónní pískání na
    několika frekvencích, které produkují nekvalitní obvody magnetofonu
    nebo mazací tón,
}
\item{
    nestacionární rušení, např. řeč na pozadí, bouchnutí dveří apod.,
}
\item{
    velká ozvěna místnosti nebo špatně ekvalizovaný mikrofon zesilující
    některé frekvence na úkor jiných,
}
\item{
    nelineární zkreslení magnetofonu,
}
\item{
    kolísání rychlosti.
}
\end{enumerate}
K~těmto újmám na kvalitě dochází při záznamu na magnetofonový pásek. Při
přehrávání nastávají další.

\section{Metrika}
\label{sec:akustika:metrika}

Předpokladem pro práci s~akustickými vlastnostmi dat je porovnávání na jejich
základě. Je nutné mít metriku, která by odrážela akustickou podobnost
jednotlivých souborů. K~tomu využívám algoritmu, který navrhují Mandel a Ellis\cite{mandel2005song} v~implementaci
programu Musly od Dominka Schnitzera\cite{schnitzer2011using}. Pro tuto metriku
používám v~textu pojmu {\em akustická vzdálenost}.

Metrika se počítá následovně: Z~každého porovnávaného zvukového záznamu se pořídí
dvacetidimenzionální melfrekvenční kepstrální koeficienty. K~nim se najde pomocí
E-M algoritmu normální distribuce, která je generuje. Na těchto distribucích se
pak spočte symetrizovaná Kullback-Leiblerova divergence, tedy
$KL(a|b) + KL(b|a)$.

V~rámci nahrávek dochází často ke změnám akustických vlastností. To je dáno
hlavně tím, že se nahrávání uprostřed pásky přerušilo a obnovilo se v~jiných
podmínkách. Není proto žádoucí porovnávat celé nahrávky. Ideální by bylo
detekovat akustické zlomy v~nahrávkách a korpus přerozdělit ne podle hranic
nahrávek, ale podle těchto zlomů.

Pro jednoduchost jsem nahrávky rozdělil do menších úseků a matici
akustické vzdálenosti jsem udělal na nich. Pokusil jsem se využít hotových úseků rozdělených
v~bodech ticha, viz sekci~\ref{sec:segmenty}. Výsledná matice o rozměrech
80~000 × 80~000 však trpěla defekty při čtení a zápisu, proto jsem velikost
úseků pro tento účel zvětšil na 10 minut a tím dosáhl počtu 8146 úseků.

Pro orientaci uvedu některé údaje z~matice akustické vzdálenosti. 
Medián vzdálenosti je 55,8.
Maximální vzdálenost je $3,40\cdot{}10^{38}$, ovšem ta nastává v~okrajových
případech, bez zjevného důvodu patrného lidskému uchu. Bez této astronomické
maximální vzdálenosti dosahují vzdálenosti hodnot do 27~814.
Počet nulových vzdáleností je 275 z~celkových 33~174~585 a skutečně odpovídají
duplicitním úsekům.

Abychom tato čísla mohli interpretovat, porovnejme je se vzdálenostmi jiných
zvuků, které si snad čtenář dokáže představit.
Provedl jsem pro zajímavost porovnání 1) řeči tří mluvčích ze záznamu jednání
Poslanecké sněmovny, 2) řeči a ruchu na pozadí z~filmu a 3) řeči a populární hudby.

Tři různí mluvčí jednoho záznamu jednání Poslanecké sněmovny
Parlamentu České republiky mají vzdálenost v rozmezí 6,5 až 9,5. Dva různé úseky
téhož mluvčího mají vzdálenost 1,4. Tyto záznamy z PSP ČR jsou i pro lidské ucho
velmi podobné a na rozdílu ve vzdálenosti mezi různými mluvčími oproti
vzdálenosti v~rámci jednoho mluvčího je vidět, že algoritmus funguje dobře.

Jako druhý příklad vezměme typickou, úsměvně známou ukázku mluveného slova
s~hlukem na pozadí, a sice úryvek ,,to je dost, žes nás taky jednou vyvez, žes
udělal něco pro rodinu`` z~filmu Slavnosti sněženek. Jednotliví mluvčí (Blažena
Holišová a Rudolf Hrušínský) mají vzájemnou vzdálenost 13,9. Holišová od
samotného malotraktoru 48,2 a Hrušínský od téhož 23,7.

Srovnejme ještě mluvenou řeč s~populární hudbou. Uvedu dva extrémní příklady, na
které jsem narazil:
Skladba Billie Jean od Michalea Jacksona má od řeči poslance Sklenáka vzdálenost
23,1, zatímco refrén skladby Shadow Sun od metalové skupiny Moonspell od řeči
poslance Okamury 519.

Je tedy patrno, že akustická variabilita korpusu Karla Makoně je i podle tohoto
měřítka obrovská.

\section{Shlukování}

Kompenzovat akustické nedostatky znamená měnit akustické kvality dat tak, aby
lépe odpovídaly nějakému kritériu. To může být subjektivní: aby se zvuk určitému
posluchači lépe poslouchal. Může být také objektivní, strojově vyhodnotitelné,
což pak umožní nasazení strojových metod. Jako vhodné objektivní kritérium lze
zvolit akustickou vzdálenost k~záznamům kýžených kvalit.

V~korpusu Karla Makoně se nevyskytuje asi žádná nahrávka ve vysoké kvalitě
srovnatelné s~materiálem pořízeným ve studiových podmínkách. Velká část
nahrávek je však veskrze srozumitelná a bez zásadních defektů. Z~celého korpusu jsem
vybral množinu 431 souboru v~uspokojivé kvalitě.

Tato referenční množina je mnohem konzistentnější co do akustické metriky než
celý korpus, ale v~porovnání s~ilustračními příklady mimo korpus stále řádově
divergentnější. Vzdálenosti se pohybují od 1,6 do 14~653 s~průměrem  54,7 a
mediánem 9,40.

Na množině úseků, které jsem porovnával akustickou metrikou, jsem provedl
hierarchické shlukování\cite{johnson1967hierarchical}. Vzdálenost clusteru jsem nastavil jako maximální
vzdálenost mezi dvěma prvky, aby jednotlivé clustery byly co nejkompaktnější.

\section{Kompenzace}
\label{sec:akustika:kompenzace}

Akustické nedostatky ztěžují rozpoznávání řeči a jsou tak už dlouho předmětem
bádání. Ku příkladu Gillespie a Atlas (2002)\cite{gillespie2002diversity} ukazují
zdrcující vliv dozvuku (angl. {\em reverberation}) na rozpoznávání řeči pomocí
markovovských modelů a zkoumají možnosti kompenzace. Jošioka et
al. (2012)\cite{reverbmagazine} předkládají souhrn technik pro kompenzaci dozvuku, Ko
et al. (2017)\cite{reverbaugment} digitálně simulují dozvuk v~trénovacích datech.
Seltzer et al. (2013)\cite{dnnnoiserobust} se věnují tematice šumu v~rozpoznávání řeči
založeném na hlubokých neuronových sítích.

\subsection{Spektrální odečet šumu}

Jako baseline svého druhu jsem se pokusil automatizovaně aplikovat zavedenou
metodu zvanou redukce šumu, \textit{noise reduction}. Pro jednoduchost zde
předpokládám, že každá nahrávka trpí konstantním stacionárním šumem. To pro
všechny nahrávky neplatí, ale pro ověření způsobilosti metody to není podstatné.
Pro redukci šumu jsem použil program \texttt{sox}. Metoda spočívá pro každou
nahrávku v~těchto krocích:
\begin{enumerate}
\item{Identifikovat a izolovat vzorek čistého stacionárního šumu,}
\item{extrahovat profil šumu a}
\item{aplikovat redukci šumu na základě získaného profilu.}
\end{enumerate}

O body 2 a 3 se postará sox. Co se týče získání vhodného vzorku šumu, použil jsem
následující metodu:
\begin{enumerate}
\item{
    Určit a extrahovat všechna predikovaná ticha za použití zarovnaného
    automatického přepisu.
}
\item{
    Seřadit ticha sestupně podle délky a vybrat jich 100 kolem 25. percentilu.
    Tak se zajistí, že se nepoužijí ani příliš krátká ani příliš dlouhá ticha.
    V~dlouhých bývají nestacionární ruchy, proto se jim vyhýbám.
}
\item{
    Pomocí programu musly vygenerovat matici vzdáleností na tiších.
}
\item{
    Vybrat deset tich s~nejmenším mediánem na vzdálenostní matici. Tato ticha
    jsou nejvíce podobna ostatním, a tím pádem s~nejmenší pravděpodobností
    obsahují nestacionární události.
}
\item{
    Konkatenovat vybraná ticha.
}
\end{enumerate}

Subjektivní vyhodnocení potvrzuje očekávaný výsledek, že relativně kvalitní
záznamy, které trpí pouze trochou aditivního šumu, se po redukci lépe
poslouchají. Záznamům trpícím jinými defekty a celkově nižší kvalitou je někdy po
operaci hůře rozumět.

Pro kvantitativní vyhodnocení jsem natrénoval systém rozpoznávání řeči, který
popisuji v~kapitole~\ref{kap:asr}, na datech po odstranění šumu. Chybovost na
slovech vzrostla skoro na sto procent, což znamená, že systém zcela přestal
fungovat. Přesnou příčinu neznám, ovšem vzhledem k~nízkým očekáváním, které jsem
od metody měl, nepovažuji za účelné se po ní pídit.
V~tabulce~\ref{tab:results-denoise} uvádím chybovost modelu natrénovaného na
původních datech a datech po redukci, testovaného opět na původních datech a
datech po redukci šumu.

\begin{table}[htpb]
\begin{center}
\begin{tabular}{|l||r|r|}
\hline
WER    & původní model & model po redukci \\
\hline
původní testovací sada & 19,2\% & 94,0\% \\
sada po redukci šumu   & 69,8\% & 94,1\% \\
\hline
\end{tabular}
\caption{Word error rate při trénovacích i testovacích datech před redukcí šumu
a po ní.}\label{tab:results-denoise}
\end{center}
\end{table}

\subsection{Neurální doménový transfer}

Revoluční článek\cite{cyclegan}, v~němž Žú et al. (2017) představují CycleGAN, čili
cyklicky konzistentní generativní oponentní síť, dal lidstvu do rukou mocný
nástroj a zábavnou hračku, která našla využití pro odstraňování mlhy
z~fotografií\cite{Engin_2018_CVPR_Workshops}, udělování cizích grimas
tvářím\cite{jin2017faceoff}, v~biomedicíně\cite{yang2018biogan} a také pro
zpracování mluvené řeči: Kaneko a Temeoka (2017)\cite{kaneko2017parallel} předkládají
doménový transfer hlasu a Hoseini-Asl et al. (2018)\cite{hosseini2018malevoicegan} činí
totéž pro účely rozpoznávání řeči. Nejblíže mému problému je Pascual et al. (2017)
s~projektem SEGAN\cite{pascual2017segan}, kde se GAN používá pro odstranění
šumu.

CycleGAN je vytvořena pro použití na dvou sadách dat, z~nichž každá reprezentuje
určitou doménu, mezi nimiž lze najít mapovací funkci. V~případě korpusu Karla
Makoně je jedna jasná doména akusticky relativně dobrých nahrávek a potom celý zbytek,
který ovšem konzistentní doménu netvoří. Poškozené nahrávky trpí různými
kombinacemi neduhů v~různé míře. Nelze proto CycleGAN přímo aplikovat na
,,zdravé`` a ,,poškozené`` nahrávky. Jak by taky bylo lze nalézt funkci mapující
nahrávku bez neduhů na poškozenou, když není jasné, jaké poškození by měla
vykazovat? Tento směr sice není kýžený, ale pro natrénování dané architektury
nezbytný.

Adaptace GAN tak, aby si s~nekonzistentními daty na jedné straně převodu
poradila, by jistě byla zajímavým a přínosným počinem, nicméně začít se dá
využitím výše zmíněného shlukování, které poskytuje potřebné konzistentní
domény poškozených nahrávek.

Experiment jsem provedl na dvou shlucích: 1) na přebuzených nahrávkách a 2) na
nahrávkách pořízených nízkou rychlostí 2,38 cm/s. Oba shluky jsem vybral tak, aby
měly maximální interní vzdálenost 25. Pro trénování jsem použil metodu
navrhovanou dvojicí Kaneko a Tameoka (2017)\cite{kaneko2017parallel}, jak ji
implementoval Lei Mao\footnote{github.com/leimao/Voice\_Converter\_CycleGAN}.
Trénink běžel po 200 epoch.

\subsection{Vyhodnocení}
\label{sec:akustika:vyhodnoceni}

Tabulka~\ref{tab:ganeval} uvádí WER při automatickém přepisu na tom kterém
shluku původně a po transferu. Toto porovnání bylo provedeno pomocí modelu
natrénovaného pouze na promluvách Karla Makoně. Robustnější model popsaný
v~sekci~\ref{sec:deepspeech} má chybovost na nízkorychlostních nahrávkách
42,1\% %0.421338    , bez číslic 43,4\%
a na nahrávkách přebuzených 34,8\%. %0.348183, bez číslic 36,6\%

\begin{table}[htpb]
\begin{center}
\begin{tabular}{|l||r|r|}
\hline
                 & původní & po transferu \\
\hline
přebuzené        & 45,0\%  & 44,1\% \\
nízkorychlostní  & 68,5\%  & 93,9\% \\
\hline
\end{tabular}
\caption{Word error rate u dvou skupin poškozených nahrávek před doménovým
transferem a po něm.}\label{tab:ganeval}
\end{center}
\end{table}

Mírné umenšení chybovosti u přebuzených nahrávek je statisticky nevýznamné a
nepřináší kýžené řešení problému s~neuspokojivými výsledky rozpoznávání
poškozených nahrávek. Snad důležitější je však zvýšený komfort při
poslechu. Bohužel pro tento přínos zatím nemám kvantitativní vyhodnocení, ale
subjektivně ho potvrdit mohu.

U katastrofálního zvýšení chybovosti po transferu velmi poškozených nahrávek
pořizovaných nízkou rychlostí je nutno se pozastavit.
Obrázek~\ref{fig:gan:plzen} ukazuje, jak se u těchto nahrávek doménový transfer
odrazil v~průběhu signálu a ve spektru. Obrázek~\ref{fig:gan:overdrive} ukazuje
pro porovnání totéž u přebuzených nahrávek. Povšimněme si, jak u nahrávek
pořízených nízkou rychlostí po přesunu zcela zmizela některá slova. Je-li signál
příliš těžko odlišitelný od ruchů, transfer patrně raději změní takový úsek
v~ticho.

\begin{figure}[htpb]
\includegraphics[scale=0.4]{rc/gan-plzen.eps}
\caption{Průběh signálu (nahoře) a spektrogram (dole) nahrávky pořízené
rychlostí 2,38 cm/s před doménovým transferem (vlevo) a po něm (vpravo).}
\label{fig:gan:plzen}
\end{figure}

\begin{figure}[htpb]
\includegraphics[scale=0.4]{rc/gan-overdrive.eps}
\caption{Průběh signálu (nahoře) a spektrogram (dole) přebuzené nahrávky
před doménovým transferem (vlevo) a po něm (vpravo).}
\label{fig:gan:overdrive}
\end{figure}

\chapter{Ostatní zdroje trénovacích dat}
\label{kap:svolocz}

Trénovacích dat pro rozpoznávání řeči není nikdy dost. V~době psaní tohoto textu
je manuálně přepsaných asi 100 hodin z~Mluveného korpusu Karla Makoně. To je pro
natrénování modelu pro jednoho mluvčího použitelné množství a takto dosažená WER
je 20\%. Zda by více trénovacích dat od jiných mluvčích mohlo pomoci, je však
otevřená otázka.

Veřejně k~dispozici je několik zdrojů trénovacích dat pro účely trénování
rozpoznávače češtiny, jichž jsem si vědom:
\begin{itemize}
\item{
    Vystadial\cite{vystadialarticle} se 77 hodinami záznamů internetových
    rozhovorů\cite{vystadialdata},
}
\item{
    The Prague Database of Spoken Czech\cite{pdtscarticle} se 122 hodinami
    spontánních dialogů anotovaných na několika úrovních\cite{pdtscdata},
}
\item{
    Korpus expresivní mluvy COMPANION s~5 hodinami namluvenými jednou
    profesionální mluvčí\cite{companiondata},
}
\item{
    Otázky Václava Moravce: 35 hodin přepsaných záznamů české talk
    show\cite{ovmdata},
}
\item{
    STAZKA: 35 hodin záznamů z~vozidel na silnicích obsahujících anotované
    promluvy\cite{stazkadata},
}
\item{
    88 hodin automaticky přepsaných záznamů z~jednání poslanecké
    sněmovny\cite{pspdata}.
}
\end{itemize}
Celkem se tak dostaneme přibližně na 350 hodin dalších trénovacích dat.

Ze záznamů jednání poslanecké sněmovny však existují také ruční stenografické
přepisy. Velká část jednání a jejich přepisů je veřejně ke stažení na webových
stránkách poslanecké sněmovny. Pokusil jsem se proto z~těchto dat připravit
korpus pro trénování rozpoznávání řeči.

\section{Příprava dat}

Ruční přepisy jednání poslanecké sněmovny jsou, pokud je mi známo, k~dispozici
pouze ve formátu čitelném pro člověka. Přepisy nejsou ve zdrojovém kódu webové
stránky nijak oddělené a jsou promíchané s~metainformacemi. Je tedy nutné
extrakci pojmout jako aproximační úkol. Používám velice jednoduchý algoritmus,
který má své nedostatky, ale pokrývá drtivou většinu záznamů. Extrahuji podstrom
všech elementů s~hodnotou atributu \texttt{[align=justify]} vyjma elementů
\texttt{<b>}, neboť ty obsahují jména mluvčích.

Známé nedostatky spočívají jednak v~tom, že se jména mluvčích sice správně z~přepisu
oddělují, ale navzdory jejich hodnotě coby metainformace zahazují, a jednak
v~tom, že se z~přepisu vynechávají odkazy na jiné schůze. Ty jsou totiž
formátované jinak než ostatní části, jak je vidět např. v~přepisu schůze z 12.
února 2020 od
10:10\footnote{https://www.psp.cz/eknih/2017ps/stenprot/040schuz/s040372.htm}.
Opravení obého je otázkou napsání chytřejšího scraperu a pro účely vybudování
korpusu pro tréning rozpoznávání řeči je obé bezvýznamné: Označení mluvčích
z~principu, vynechání odkazů pro jejich řídkost.

\subsection{Zarovnávání}
\label{subsec:svolocz:zarovnavani}

Jedna z~překážek použití stenografických přepisu pro trénování rozpoznávačů řeči
je velmi volné párování neboli zarovnání přepisů ke zvuku. Každý zvukový záznam
má 14 minut a překrývá se se sousedními čtyři minuty na každé straně.
Přepisy jsou rozdělené na úseky odpovídající těmto záznamům. Zarovnání je tedy
do desetiminutových bloků s~dvouminutovým přesahem na každé straně. Na
obrázku~\ref{fig:svolocz:overlap} je vyobrazeno schéma tohoto apriorního
zarovnání.

\begin{figure}[htpb]
\includegraphics[scale=0.25]{rc/svolocz-overlap.eps}
\caption{Apriorní zarovnání a překruv zvukových záznamů k~přepisům. Vyobrazen je
záznam z 12. února 2020 kolem 10. hodiny. Přepis záznamu vlevo nahoře pokrývá
pozice od 01:34 do 11:24. Vpravo dole pak od 01:24 do 12:00.}
\label{fig:svolocz:overlap}
\end{figure}

Systémů pro zarovnávání dlouhých zvukových záznamů existuje několik, publikovali
je např. Moreno et al\cite{moreno1998recursive} nebo
Hazen\cite{hazen2006automatic}. Oba jsou založeny na využití předem získaného a
zarovnaného automatického přepisu. Taktéž tento přistup využívám, ale
zjednodušený a přizpůsobený úloze.

Za použití výše zmíněného datasetu\cite{pspdata} jsem natrénoval markovovský
akustický model obdobný tomu, jenž je popsán v~kapitole~\ref{kap:asr}. Jazykový
model jsem natrénoval ze stažených stenografických přepisů. Těch je podstatně
víc než nahrávek, protože z~mně neznámého důvodu je valná část odkazů na zvukové
záznamy nefunkčních, končíc chybovým kódem 404 nebo v~menším počtu případů 403.

Pro celý korpus jsem pomocí programu julius vygeneroval automatický přepis se
zarovnáním. Automaticky vygenerovaný zarovnaný přepis každého záznamu jsem pak
porovnal s~odovídajícím manuálním přepisem pomocí Levenshteinovy metody počítání
editačních operací. Zjistil jsem samotné editační operace pro přechod
z~automatického přepisu k~manuálnímu a pro každé slovo v~automatickém přepisu
spočetl, kolik úprav naň připadá. Na základě toho definuji pro každé automaticky
vygenerované slovo spolehlivost párování se slovem manuálně zapsaným jako
\begin{equation}1 - \frac{počet editačních operací}{počet písmen}\end{equation}

Obrázek~\ref{fig:svolocz:align} proces zarovnání manuálních přepisů se zvukovým
záznamem zachycuje.

\begin{figure}[htpb]
\includegraphics[scale=0.4]{rc/svolocz-align.eps}
\caption{schéma zarovnání zvukových záznamů ke stenografickým přepisům na úrovni
slov}
\label{fig:align}
\end{figure}

\subsection{Tvorba potenciálních trénovacích vzorků}

Kvalita trénovacích dat pro rozpoznávání řeči závisí i na tom, jak jsou
rozdělena. Za prvé je žádoucí, aby byly trénovací vzorky podobně dlouhé
jako testovací vzorky\cite{nagorski2003search}.
V~případě automatického přepisu korpusu Karla Makoně lze
nastavit délku vstupních úseků libovolně. V~obecném případě nelze předvídat.
Za druhé je pro tréning výhodné, aby jednotlivé vzorky měly podobnou délku kvůli
efektivnímu využití operační paměti grafické procesní jednotky. Stačí jeden
dlouhý vzorek a dávka {\em (batch)} způsobí vyčerpání paměti. Naopak mnoho
kraťoučkých vzorků způsobí, že se paměť při dávce zaplní jen málo a plýtvá se
časem. Za třetí, chci-li, aby byla trénovací sada dala použitelná i pro ostatní,
je dobré, aby se délkou vzorků příliš neodlišovala od ostatních datových sad.

Nad to je ale důležité, aby přepis dokonale odpovídal obsahu. A protože
vyřezáváme trénovací vzorky z~delších souborů, při čemž může dojít
k~nepřesnostem, je záhodno volit místa řezu tak, aby padla pokud možno do
delších pauz v~řeči. Je to týž problém jako v~sekci~\ref{sec:segmenty}. V~tomto
případě jsem dospěl k~hranicím 12 - 30 sekund.

\subsection{Výběr trénovacích vzorků}

Po rozdělení desetiminutových nahrávek na úseky vhodné délkou pro trénink je
nutné spárovat tyto úseky s~odpovídajícími úseky manuálních přepisů, a ty, které
jsou spárovány spolehlivě, zařadit do samotné trénovací množiny.
V~podsekci~\ref{subsec:svolocz:zarovnavani} jsem popsal, že máme párování
jednotlivých slov v~automatickém a manuálním přepisu s~určitou mírou
spolehlivosti a že automatický přepis je spárován se zvukovým záznamem. Zbývá
tedy vybrat úseky, které můžeme považovat za spolehlivě přepsané, a zbytek
zahodit.

Nahrávky se překrývají tak, že z~každého čtrnáctiminutového souboru je jen deset
minut pokryto odpovídajícím přepisem, takže zahodit musíme minimálně 40\% úseků.
Dospěl jsem k~následujícím kritériím:

\begin{enumerate}
\item{Spolehlivost prvního a posledního slova v~úseku alespoň 70\%.}
\item{Průměrná spolehlivost všech slov v~úseku alespoň 70\%.}
\item{Alespoň 5 slov v~úseku.}
\end{enumerate}



%           cstest      selftest
% ovm       0.728954    0.216131
% bible     0.947226    0.091958
% cucfgn    0.728446    0.316287
% vystadial 0.739838    0.509930
% svolocz   0.397490    0.083978
% makon     0.773216    0.191772
% oral2013  0.607015    0.783525
% all       0.283574    0.283574
%
% cucfgnOF  0.683059    0.141142
% all\o                 0.130555

\chapter{Akustický model}
\label{kap:akusticky-model}

% - množina fonémů
% - vektorový formát
% - HMM X CNN
% - HTK X Kaldi X sphinx X Bourlard X TensorFlow
% - Monofonémy X trifonémy
% - mixtury
%   - individuální
%   - globální
%   - na monofonémech vs. trifonémech

Koncept celého projektu se zakládá na~přítomnosti automatického přepisu a jeho
následném zdokonalování. Jelikož nebyl k~dispozici uspokojivý hotový nástroj pro
získání automatického přepisu, nezbylo než jej vytvořit.

Zjednodušený řetězec vedoucí od~zvukových dat k~jejich přepisu v~našem případě
vypadá takto:\begin{enumerate}
\item{sběr trénovacích dat,}
\item{stavba akustického modelu,}
\item{stavba jazykového modelu,}
\item{automatické rozpoznávání.}
\end{enumerate}

V tomto řetězci je stavba akustického modelu patrně nevyznamnějším článkem a
sama tvoří řetěz o~mnoha článcích. V~této kapitole pojednáme o~krocích
podniknutých k~jeho sestavení, experimentech a různých volbách.

Do~užšího výběru potenciálních platforem tvorby akustického modelu jsem zařadil
starší systém \textit{HTK}\footnote{hmm toolkit; Hidden Markov Model Toolkit} a
modernější \textit{Kaldi}. HTK pomyslný konkurz nakonec vyhrál díky zkušenostem
Mgr. Nina Peterka, Ph.D. s~tímto systémem, z~níž jsem mohl čerpat.

Akustický model jsem trénoval výhradně z~vlastníh dat. Obětoval jsem tedy
potenciální přínos většího množství trénovacích dat a upřednostnil trénování
přímo na~konkrétního mluvčího.
%TODO: od kdy se vyplatí dělat model na konkrétního mluvčího?

Tvorba akustického modelu probíhala podle návodu v~manuálu k~HTK, \textit{HTK
Book}. V~hrubých rysech probíhá takto:

Blíže popíšu jen rozhodnutí, která nejsou zřejmá.

\section{Segmentace}

Zpravidla jeden zvukový soubor odpovídá jednomu přetočení magnetofonové pásky,
obvyklá délka je tedy 45 až 120 minut. Takto dlouhé úseky nelze použít ani jako
trénovací příklady ani jako cíl automatického rozpoznávání.

Celá aplikace je pojata jako nástroj pro~zdokonalování automatického přepisu,
takže vychází z~předpokladu, že nějaký přepis již existuje. Vycházeje z~téhož
předpokladu při~segmentaci, realizoval jsem ji tak, že zvukový soubor se rozdělí
na~úseky odpovídající jednotlivým větám v~přepisu, ne však delší než patnáct
sekund. Pokud by věta byla delší, rozdělí se u nejbližšího slova
před~patnáctisekundovou hranicí.

V~případě, že pro daný záznam zatím žádný přepis neexistuje, rozdělí se naivně
na~patnáctisekundové úseky.

Nutno dodat, že rozdělování podle automaticky rozpoznaných hranic vět by šlo
snadno vylepšit. Jde-li o~ručně přepsaná data, jsou hranice vět dobrým vodítkem,
ale u~automaticky přepsaných by bylo lépe rozdělovat podle ticha mezi slovy.

\section{Fonetika}

Množinu fonémů jsem použil od~doc. Pavla Ircinga ze~Západočeské univerzity
z~jeho skriptů z~devadesátých let minulého století. Fonémy mají tzv. pražský a
plzeňský zápis. V~souladu se~zjevným územ jsem použil plzeňský zápis.
V~tabulce~\ref{tab:phones} jsou uvedeny.

\begin{table}[htpb]
\fontspec{DoulosSIL}
\begin{center}
\begin{tabular}{|l|l|l|l||l|l|l|l|}
\hline
IPA & plz. & praž. & grafém & IPA & plz. & praž. & grafém \\
\hline
% TODO: IPA
a  & a   & a   & a      &     ɱ  & mg  & mg  & tra\underline{m}vaj \\
aː & aa  & aa  & á      &     n  & n   & n   & \underline{n}e \\
aʊ̯ & aw  & au  & au     &     ŋ  & ng  & ng  & pa\underline{n}t \\
b  & b   & b   & b      &     ɲ  & nj  & nj  & \v{n} \\
t͡s & c   & c   & c      &     o  & o   & o   & o \\
t͡ʃ & ch  & cz  & č      &     oː & oo  & oo  & ó \\
d  & d   & d   & d      &     oʊ̯ & ow  & ou  & ou \\
ɟ  & dj  & dj  & \v{d}  &     p  & p   & p   & p \\
d͡z & dz  & dz  & dz     &     r  & r   & r   & r \\
d͡ʒ & dzh & dzz & dž     &     r̝̊  & rsh & rsz & t\underline{\v{r}}i \\
ɛ  & e   & e   & e      &     r̝  & rzh & rzz & \underline{\v{r}}íz \\
ɛː & ee  & ee  & é      &     s  & s   & s   & s \\
eʊ̯ & ew  & eu  & eu     &     ʃ  & sh  & sz  & š \\
f  & f   & f   & f      &     t  & t   & t   & t \\
g  & g   & g   & g      &     c  & tj  & tj  & \v{t} \\
ɦ  & h   & h   & h      &     ʊ  & u   & u   & u \\
i  & i   & i   & i      &     uː & uu  & uu  & ú, \r{u} \\
iː & ii  & ii  & í      &     v  & v   & v   & v \\
j  & j   & j   & j      &     x  & x   & ch  & ch \\
k  & k   & k   & k      &     z  & z   & z   & z \\
l  & l   & l   & l      &     ʒ  & zh  & zz  & ž \\
m  & m   & m   & \underline{m}ák
                        &        &     &     & \\
\hline
\end{tabular}
\caption{použité fonémy: IPA, plzeňský zápis, pražský zápis a nejčastější
odpovídající grafém}\label{tab:phones}
\end{center}
\end{table}
\normalfont

V~závislosti na~množství trénovacích dat bylo vhodné nahradit některé fonémy
častějšími podobnými. V~tabulce~\ref{tab:phonesed} jsou záměny vyčísleny.
\begin{table}[htpb]
\fontspec{DoulosSIL}
\begin{center}
\begin{tabular}{|r|l|l||l|l|}
\hline
&
\multicolumn{2}{|c||}{před záměnou} &
\multicolumn{2}{|c|}{po záměně} \\
\hline
& IPA & plz. & IPA & plz. \\
\hline
    & ɱ  & mg & m & m \\
    & aʊ̯ & aw & a ʊ & a u \\
    & oː & oo & o & o \\
\** & d͡z & dz & t͡s & c \\
    & d͡ʒ & dzh & t͡ʃ & ch \\
\** & eʊ̯ & ew & ɛ ʊ & e u \\
\hline
\end{tabular}
\caption{použité záměny fonémů; hvězdičkou jsou vyznačeny záměny použité ještě
v~době psaní textu}\label{tab:phonesed}
\end{center}
\end{table}
\normalfont

\section{Rozdělení dat}

Pro natrénování modelu strojovým učením je potřeba trénovacích dat a pro
vyhodnocení jeho úspěšnosti dat testovacích, která ve~fázi trénování nesmí být
algoritmem spatřena. Při trénování samotném se pak mnohdy používá vyhrazených,
tzv.~\textit{heldout} dat pro průběžné měření úspěšnosti. V~případě trénování
akustického modelu s~použitím HTK je tomu nejinak. Heldout data jsou používána
pro zjištění optimálního počtu mixtur modelů jednotlivých fonémů, a testovací
pro závěrečné vyhodnocení.

Anotovaná data mi přibývala velice pozvolna a začínal jsem s~několika minutami,
ovšem přírůstky byly časté. Nemohl jsem si tedy dovolit udělat od~začátku pevnou
testovací sadu, kterou bych používal po~celou dobu provádění experimentů. Místo
toho jsem s~každou novou dávkou anotovaných dat celou sadu rozdělil podle vět
v~poměru 18:1:1 do trénovací, heldout a testovací sady. Tak jsem měl neustále
vyvážený poměr jednotlivých datových sad. Zřejmou velkou nevýhodou bylo, že
nešlo spolehlivě porovnávat výsledky jednotlivých experimentů vzhledem
k~variabilní testovací sadě.

Až když jsem měl několik desítek hodin anotovaných dat, vyhradil jsem si fixní
testovací sadu. Běžně se testovací sada vybere jako náhodná podmnožina vzorků
z~trénovací sady tak, aby měla kýženou velikost. V~mém případě vzorků zvíci
hodinových nahrávek jsem sadu určil manuálně jako úsek druhé až jedenácté minuty
(tedy deset minut minutu po~začátku) v~pěti nahrávkách,
\begin{enumerate}
\item{jedné kazety z~roku 1976,}
\item{jedné z~roku 1982,}
\item{jedné z~roku 1986,}
\item{jedné z~roku 1990 a}
\item{jednoho nedatovaného kotouče.}
\end{enumerate}

Sadu heldout nyní vybírám jako každou čtyřicátou větu. Z~každé dvacáté jsem
snížil na~polovic nejen abych neplýtval trénovacími daty, nýbrž také protože
vyhodnocování mixtur zabírá při trénování zdaleka nejvíce času, a ten je přímo
úměrný velikosti sady heldout.

\section{Spojování trifonémů}

% triphone-tree
% neznámé trifonémy

\chapter{Webové rozhraní}
\label{kap:webove-rozhrani}

\section{Prototyp}

Webové rozhraní, umožňující přístup zájemcům k~nahrávkám a jejich přepisu, byl
od začátku plánovanou součástí projektu. Rozhraní bylo navrženo s~těmito
požadovanými vlastnosmi:

\begin{itemize}
\item{výběr nahrávky ze~seznamu}
\item{poslech nahrávky s~obvyklými ovládacími prvky přehrávače}
\item{zobrazení přepisu nahrávky}
\item{vyznačení právě přehrávaného slova}
\item{možnost provést změnu v~přepisu}
\item{automatické zarovnání přepisu se~zvukem}
\item{případné odmítnutí přepisu, pokud zarovnání selže (přepis neodpovídá
vyřčeným slovům)}
\end{itemize}

První implementace byla založena na přehrávači \textit{jPlayer}, modulu pro
knihovnu \textit{jQuery}, který využívá standard HTML5 s~jeho elementem
\texttt{<audio>} a technologii \textit{Adobe Flash}. Pro~dynamickou odezvu
zobrazených prvků na~změny v~datovém modelu jsem použil knihovnu
\textit{knockout}.

Aplikace měla formu jediné stránky s~rozbalovatelným výběrem nahrávky,
ovládacími prvky přehrávače a třemi řádky přepisu. Při označení části
zobrazeného přepisu se stránka překryla rozhraním pro opravu přepisu, jež zvu
\textit{editačním okénkem}. V~editačním okénku se zobrazilo vstupní pole
(\texttt{<textarea>}) s předvyplněným současným přepisem, ovládací prvky pro
přehrátí odpovídající pasáže, odeslání opraveného přepisu a opuštění editačního
okénka.

Šlo o~prostou statickou HTML stránku s~JavaScriptem. K~audiu se přistupovalo
pomocí externí CDN, zatímco přepisy a API pro~zarovnávání oprav byly
na~zvláštním serveru.

Přepis byl uložen a přenášen ve~formátu \textit{JSONp}\footnote{JSON =
JavaScript Object Notation, JSONp = JSON with Padding}, čili jako \textit{JSON}
obalený v~javascriptové funkci kvůli zamezení problémů s~přístupem napříč
doménami.  Každé slovo s~sebou neslo informaci o~svojí pozici v~nahrávce
s~přesností na~setiny sekundy, výslovnost, zápis, slovníkovou formu, délku ticha
za~slověm, informaci o~tom, zda bylo manuálně přepsáno nebo automaticky
rozpoznáno, a v~případě automaticky rozpoznaných slov \textit{confidence
measure} čili míru jistoty rozpoznání.

Převod ze~zápisu slova do~jeho fonetické podoby se děje na~základě pravidlového
algoritmu z~dílny Doc. Pavla Ircinga po~úpravě od~Mgr.~Nina Peterka, Ph.D. Tento
algoritmus zahrnuje časté výjimky z~českých výslovnostních pravidel, ale
neobsahuje rozsáhlý výslovnostní slovník cizích slov. Karel Makoň navíc nezřídka
hovoří o~osobách, jejichž jména se v~mnoha korpusech neobjeví vůbec.

Nad rámec výše popsaných funkcionalit přibyly další na~základě přání uživatelů a
autorovy potřeby:

\begin{itemize}
\item{indikace, do~jaké míry je která nahrávka přepsána,}
\item{manuální posouvání hranic přepisovaného zvukového úseku,}
\item{úprava zápisu slova s~ponecháním výslovnosti,}
\item{identifikace uživatelů včetně sezení, prohlížeče atp.,}
\item{vyhledávání v~přepisech.}
\end{itemize}

Výslovnost je interně zaznamenána pomocí mezerou oddělených fonémů, z~nichž
každý má reprezentaci z~malých písmen sady ASCII. Taková reprezentace není pro
laické uživatele praktická, proto je součástí aplikace převod do~českého
fonetického zápisu a zpět.

Uživatelé jsou proto instruováni, aby slova s~nestandardní výslovností, na~která
narazí poprvé, přepsali foneticky \textit{(džordž)} a poté, co uspěje
automatické zarovnání, slovu opravil přepis \textit{(George)}.

Tato původní verze posloužila k~přepsání asi 600 tisíc slov a běžela asi 5 let,
než bylo nutné ji nahradit.

\section{Verze 2}

Pro kompletní přepis aplikace se postupně objevilo několik důvodů. Hlavním
z~nich bylo, že původní aplikace mohla jen těžko sloužit pro širokou veřejnost
jako prostředek k~popularizaci nahrávek. Dalším důvodem bylo, že některé kýžené
funkce nebylo možné zprovoznit bez zásadních změn v~provedení. Především šlo
o~ekvalizér, čili frekvenční korekci při poslechu. Akutním důvodem pak byl fakt,
že všechny významné prohlížeče opouštěly podporu Flashe.

Pro novou verzi jsem zvolil technologie React + Redux jako aplikační rámec, Web
Audio API jako platformu pro nakládání se~zvukem a Twitter Bootstrap jako základ
pro vzhled prvků. Zdrojový kód píšu v~ECMAScript~6 a o~kompilaci se stará
webpack.

\subsection{Výběr nahrávky}

První verze obsahovala všudypřítomný rozklikávatelný kategorizovaný seznam
nahrávek. Toto jednoduché řešení mělo jen málo nevýhod. Jedna z~nich byla, že
nešlo použít běžné textové vyhledávání. V~nové verzi je proto použit
dvousloupcový formát, kde vlevo je rozbalovací seznam kategorií a vpravo
lineární seznam nahrávek. Jednotlivé kategorie jsou pak skrolovacími odkazy
do~pravého sloupce a podle stupně skrolování se příslušná kategorie sama
rozbalí (tzv. scrollspy).

Pro lepší přehlednost a v souladu s principem Separation of Concerns je seznam
nahrávek pouze na~úvodní stránce a nikoliv všude.

\subsection{Zobrazení přepisu}

V~první verzi se zobrazovaly vždy tři řádky, kde v~prostředním bylo aktuálně
přehrávané slovo. Výjimkou samozřejmě byly případy, kdy se přehrával začátek
nebo konec nahrávky. Délka řádku odpovídala šířce okna prohlížeče. Takto malý
rozsah zobrazeného textu byl zvolen proto, že aby bylo možné vizuálně odlišit
ručně přepsaná slova od~automaticky rozpoznaných a od~nich ještě slovo aktuálně
přehrávané, muselo být každé slovo obaleno ve~vlastním HTML elementu. Při~větším
množství slov pak bylo rozhraní velice náročné na~výpočetní výkon a znatelně
pomalé, což při synchronním zobrazování přepisu s~přehráváním zvuku není
přijatelné.

Zobrazení jen tří řádků mělo pochopitelně velké nevýhody. Především to, že se
člověk nemohl zorientovat v~širším rámci nahrávky (jejíž průměrná délka je kolem
hodiny) a že opět nebylo možné vyhledávat v~jejím rámci. Výhodou naopak bylo, že
aktuálně přehrávané slovo bylo vždy snadné najít. Nemožnost označit a tedy ani
přepsat příliš dlouhý úsek bylo pro uživatele možná někdy nepříjemné, ale
redukovalo chyby jak v~přepisu, tak v~automatickém zarovnávání.

Zobrazení celého přepisu při zachování plynulosti a grafickém odlišení
automaticky a manuálně přepsaných slov, slova aktuálně přehrávaného a navíc
slova vybraného kliknutím bylo první velkou výzvou pro návrh stránky přehrávání.
První, nezaslouženou pomocí k~tomu byl vývoj výkonu počítačů od~vzniku první
verze, jakož i optimalizace prohlížečů na~rychlost. Množství elementů, které lze
nyní realisticky zobrazit, se znatelně zvýšilo, ačkoliv naivní řešení zabalení
každého slova stále není praktické. Druhým pomocným faktorem je, že manuálně a
automaticky přepsaná slova se většinou vyskytují ve~větších shlucích. Jen
v~málokterých souborech se často střídají manuálně a automaticky přepsané úseky.
To vypovídá o~nevoli uživatelů k~jinému než kompletnímu, lineárnímu přepisu.
% TODO: ref. aktivní učení
Každý shluk manuálně respektive automaticky přepsaných slov stačí tedy zabalit
do~jednoho elementu.

Zvýraznění jednotlivých slov -- aktuálně přehrávaného a vybraného klikem myší --
se realizuje pomocí umístění barevného rámečku pod~zvýrazněné slovo. To lze
provést díky tomu, že prohlížeče nyní umožňují zjistit polohu označeného textu a
označení lze provést automaticky.\footnote{Viz \texttt{getClientRects} a
\texttt{Range} ve~webových standardech.}

\subsection{Web Audio API}

Přechod na~tuto technologii umožnil některé pokročilé funkce, avšak za~relativně
vysokou cenu. Web Audio API je standard pro~pokročilé zpracování zvukového
signálu v~prohlížeči. Základním konceptem je graf procesních uzlů, které mají
vstup a výstup a mohou se libovolně propojovat. K~dispozici jsou zdroje zvuku
jako oscilátory nebo přehrávače streamů, souborů (tag \texttt{<audio>}) a dat
v~paměti (\texttt{AudioBuffer}) a efekty jako zesílení, dynamická komprese,
mixování kanálů.

Velká výhoda Web Audio API oproti elementu \texttt{<audio>} je možnost přesného
časování, až na~jednotlivé samply. Přehrávání výseku odpovídajícího označenému
textu se proto nemusí provádět pomocí velice nepřesného časovače
\texttt{setTimeout}.

Bez~Web Audio API by také nebylo možné provádět frekvenční korekci při poslechu,
čili mít tzv. \textit{ekvalizér}. Ten je zapotřebí, protože některé nahrávky
mají v~určitém frekvenčním pásmu silný šum, jehož odstranění je s~ekvalizérem
snadné a komfort poslechu se tak razantně zvýší.

Další funkcí, kterou Web Audio API umožňuje, je stahování úseků. Označením
přepsaného textu se definuje úsek nahrávky a ten je možné uložit bez~dalšího
síťového přenosu. Tato funkce však vyžaduje, aby nahrávka byla dekódovaná
v~paměti. Vzhledem k~tomu, že nahrávky mají běžně i hodinu a půl, trvá její
stažení a dekódování opravdu dlouho a navíc prohlížeč kvůli tomu spotřebuje přes
gigabyte operační paměti.

Jsou plány na~to, aby Web Audio API umožnila dekódovat jen část
nahrávky,\footnote{github.com/WebAudio/web-audio-api/issues/1305} proto tento
neutěšený stav zatím nechávám být. Případným řešením by mohlo být buď
streamování (\texttt{<audio>} jako zdrojový uzel) a stažení pouze při potřebě
ukládání úseku nebo změna uložení nahrávek nikoliv jeden soubor na jednu pásku,
nýbrž např. po~minutových úsecích.

Díky tomu, že Web Audio API umožňuje přehrávání binárních dat z~proměnné
v~paměti, nabízí se dekódovanou nahrávku uložit na~persistentní úložiště
uživatelova počítače a při opětovné návštěvě stránky data místo stahování odsud
nahrát.

Moderní prohlížeče poskytují několik bran k~úložišti na~místním disku.
Nejtradičnějšími jsou bezesporu \textit{cookies}, které jsou však pro ukládání
objemnějších dat zcela nepoužitelné. Velice slibnou se jeví
\textit{localStorage}, umožňující ukládání párů klíč-hodnota. I zde však
narážíme na~příliš omezující kvóty. Kupříkladu Firefox ji má na 10MB, přičemž
potřeba je asi 1GB. Dalším kandidátem je \textit{File System API}. Tento
standard pro~izolovaný souborový systém k~dispozici webové aplikaci je zcela
ideálním řešením -- dá se zde i explicitně požádat o~konkrétní diskovou kvótu a
uživatel tak má volbu bez nutnosti práce programátora webové aplikace. Kamenem
úrazu je zde však podpora, která se momentálně omezuje pouze na Google Chrome.

Naštěstí existuje ještě standard \textit{IndexedDB API}, který má uspokojivou
podporu a uložení gigabytu dat je s~ním možné, byť ne zaručené. S~využitím
abstrahující knihovny \textit{Dexie} je proto skrz tento standard ukládání
implementováno. Pro uživatele, kteří delší dobu pracují na jedné a téže nahrávce
se tím přináší velká úspora času a přenesených dat.

\section{Rozdělení nahrávek na úseky}

Vzhledem k~tomu, že ani v~roce 2019 není kurzorový přístup ke zvukovým datům
skrze Web Audio API v~dohlednu, a k tomu jak odrazující dopad má nutnost
stahovat a dekódovat celou nahrávku aspoň při jejím prvním načtení, nezbylo mi,
než změnit způsob, jakým jsou nahrávky uloženy.

Nahrávky jsou uloženy v~několika instancích pro různé účely:

\begin{enumerate}
\item{na backendovém serveru ve formátu MFCC pro nucené zarovnávání,}
\item{v~repozitáři LINDAT ve formátu FLAC za účelem archivace a bádání,}
\item{na CDN ve formátu mp3 za účelem přímého stažení uživatelem,}
\item{taktéž na CDN ve formátech OGG/Vorbis a mp3 pro webové rozhraní.}
\end{enumerate}

Pouze poslední jmenovanou instanci je žádoucí ukládat tak, aby každý soubor byl
jen tak velký, aby jeho stažení a dekódování trvalo únosně dlouho. V~ostatních
případech je lépe zachovat uložení, kde jedna nahrávka odpovídající většinou
jedné straně kazety či jednomu průchodu pásky z~kotouč na kotouč. Třetí a čtvrtá
instance však navzdory rozdílnému účelu sdílejí tatáž data. Bylo proto nutné je
duplikovat.

\subsection{Délka segmentů}

Délka úseků, na které nahrávky rozděluji, ovlivňuje, jak dlouho se každý segment
bude stahovat a dekódovat. Čas stahování a dekódování segmentu, který obsahuje
slovo, na němž je kurzor při prvním požadavku o~přehrávání, je roven zpoždění od
uživatelské akce k začátku přehrávání. Podle internetového periodika
UXMovement\cite{foursecondrule}, začíná uživatel po čtyřech sekundách čekání
upouštět od předchozího záměru. Podle článku Nielsen Norman
Group\cite{websiteresponsetimes} je hranice únosnosti 10 sekund.

Pokud budou úseky příliš dlouhé, jejich stahování a dekódování zabere příliš
mnoho času. Na druhou stranu s každým předělem vnášíme do přehrávání bod, kde se
úseky nalepují a může tam vyvstat artefakt. Také s~každým segmentem se pojí
extra HTTP request s~nezanedbatelnou režií.

Jako vhodný kompromis se jeví segmenty o délce 30 - 120 sekund. Velikost
dvouminutového segmentu je v~komprimovaném formátu při mono/24KHz kolem 0.6MB a
na Intel Core2 o 2,5GHz se dekóduje asi 1.6 sekundy.

\subsection{Hledání bodů předělu}

Vhodným výběrem bodů předělu můžeme omezit dopad případných artefaktů
způsobených nepřesným navázáním. Ideálním by bylo dělit nahrávky v~momentech
ticha. Ne vždy jsou momenty ticha každé dvě minuty, proto z momentů ticha
ustupme k~požadavku pauzy v~řeči. Hovořit dvě minuty bez nádechu hraničí
s~nemožností. Potýkáme se tedy s~úlohou nalézt pauzy v~řeči. Jednak je třeba
ujasnit, podle jakého klíče budeme pauzy vybírat, a jednak, jak je budeme přesně
hledat.

Hledat pauzy v~řeči lze různými způsoby. Nejspolehlivější a nejnáročnější je
manuální označování pauz. Pokoušel jsem se o~to sám a dosáhl jsem rychlosti
přibližně čtyřnásobku rychlosti přehrávání, tedy jeden zapsaný bod předělu za
třicet sekund. Dalších asi pět dobrovolných anotátorů se o~tento úkol pokusilo a
došla jim trpělivost po nule až deseti minutách označkovaného materiálu.

Další velice spolehlivou metodou je hledání podle predikovaných pseudofonémů
ticha v~zarovnaném přepisu. Tuto metodu jsem mohl namnoze použít, neboť
k~většině nahrávek mám automatický nebo i manuální přepis.

Tam, kde pořízení přepisu nebo jeho automatické zarovnání selhalo, lze použít
detekci ticha prostou akustickou analýzou. Tato metoda je velice náchylná
k~chybám v~případě nahrávek s~malým poměrem signálu k~šumu, kterých se v~korpusu
Karla Makoně vyskytuje neutěšeně mnoho.

\chapter{Vyhledávání}
\label{kap:vyhledavani}

Možnost vyhledávat v~nahrávkách byl pro mne jeden z~hlavních cílů od začátku
projektu. Se~získáním přepisů náhrávek, byť kolísavé kvality, bylo možné
vyhledávání realizovat.
Fulltextové vyhledávání jsem implementoval nástroje Elastic.

Elastic je svobodný vyhledávač napsaný v~Javě, který umožňuje fulltextové
vyhledávání v~dokumentech. Dokumenty se rozumí datové struktury, které se
vyhledávači poskytnou ve formátu JSON. Elastic má mnoho funkcionalit,
z~nichž pro mne je klíčové rozhraní na základě HTTP naplňující konvence REST,
automatický stemming, zvýrazňování nalezených pasáží a možnost vyhledávat
v~libovolných položkách dokumentu.

Aby bylo možné každý nalezený výsledek proměnit v~odkaz na příslušnou pasáž
v~nahrávce, zvolil jsem za jednotlivé dokumenty nikoliv celé nahrávky, nýbrž
věty.

Ke každému dokumentu se ukládá
\begin{itemize}
\item{textová reprezentace,}
\item{posloupnost hlásek,}
\item{stupeň
manuálního přepisu, tedy zda je přepis pořízen zcela automaticky, zcela manuálně
nebo kombinací obého}
\item{a také vektor confidence measure jednotlivých automaticky přepsaných slov.}
\end{itemize}

Pro skloňování je použito pravidlového stemmingu, který je dodáván s~distribucí
Elastic a pro češtinu, obzvlášť tam, kde se vyskytuje mnoho
nestandardních a archaických slov, funguje báječně.

Momentálně je vyhledávač nainstalován na témž serveru jako API a je dostupný
z~webové aplikace. Důležitým bodem budoucí práce je automatizace indexování
manuálních oprav, jak přicházejí. Dále pak zakomponování automatického přepisu
pořízeného bez použití jazykového modelu, jak se diskutuje
v~podsekci~\ref{ssec:data:topicsearch}.

\section{Případová studie}

Vyhledávání v~přepisech mluveného korpusu našlo využití v~kompilaci referátů o
určitých tématech, kterým se Karel Makoň věnuje. V průběhu let 2018 až 2020
vznikly alespoň čtyři takové, a to na témata
\begin{itemize}
\item{karma,}
\item{převtělování,}
\item{Otčenáš,}
\item{relativní dobro a zlo.}
\end{itemize}
Každé téma bylo zpracováno do formy souboru krátkých úseků nahrávek, které se
prezentovaly sekvenčním přehráním s~podporou přepisů jako zrakového vodítka.

Autor referátu o tématu relativního dobra a zla dohledal poznámkový aparát
k~tvorbě a rekonstruoval svůj postup, který zde popíšu jako příklad použití
přepisů, z~něhož lze usoudit na efektivitu práce.

Téma relativního dobra a zla bylo předem zamyšleno a bylo vybráno pro autorův
zájem a nikoliv s~ohledem na to, jak snadné bude pro vyhledání. Dopředu byl dán
časový rámec výsledku na cca. dvě hodiny zvukového záznamu.

Autorova metodika byla následovná: vyhledal frázi ,,relativní dobro a zlo`` a
výsledky procházel ve výchozím relevančním řazení nástroje Elastic. Prošel
prvních sto z~celkových 7379 výsledků. U každého posoudil, zda se jedná o
pasáž, skutečně o tématu pojednávající, nebo jen o letmou zmínku či vůbec
falešný zásah, popřípadě o duplicitní výskyt.

Výběr vzorků probíhal ve dvou průchodech. V~prvním autor vybíral relevantní a
relativně obsáhlejší zásahy, čímž se tvořila užší množina kandidátských úseků.
V~druhém průchodu pak vybíral z~této užší množiny s~ohledem na ročník zdrojové
nahrávky, aby byl v~kompilátu zastoupen průřez vývoje Makoňova myšlení,
výjimečně podle návaznosti či pro závěrečnou část shrnující charakter výpovědi.

Po prvním průchodu se do užšího výběru dostalo 25 nalezených úseků, tedy
čtvrtina procházené množiny, a všechny byly z~prvních 53 zásahů. V~první stovce
výsledků vyhledávání autor identifikoval 16 duplicit.

\chapter{Závěr}
\label{kap:zaver}

TODO


%%% Seznam použité literatury
\include{literatura}

%%% Obrázky v disertační práci
%%% (pokud jich je malé množství, obvykle není třeba seznam uvádět)
\listoffigures

%%% Tabulky v disertační práci (opět nemusí být nutné uvádět)
%%% U matematických prací může být lepší přemístit seznam tabulek na začátek práce.
\listoftables

%%% Použité zkratky v disertační práci (opět nemusí být nutné uvádět)
%%% U matematických prací může být lepší přemístit seznam zkratek na začátek práce.
%\chapwithtoc{Seznam použitých zkratek}

%%% Součástí doktorských prací musí být seznam vlastních publikací
\chapter*{Seznam publikací}
\addcontentsline{toc}{chapter}{Seznam publikací}

\begin{enumerate}
\item{
  Oldřich Krůza and Nino Peterek.
  Making Community and ASR Join Forces in Web Environment.
  In \textit{International Conference on Text, Speech and Dialogue},
  pages 415--421.
  Springer, 2012.
}
\item{
  Oldřich Krůza and Vladislav Kuboň.
  Second-Generation Web Interface to Correcting ASR Output.
  In Kohei Arai, Rahul Bhatia and Supriya Kapoor, editors,
  \textit{Proceedings of the Future Technologies Conference (FTC) 2018},
  number 1, pages 749--762, Cham, Switzerland, 2018.
  Science and Information Organization, Springer-Verlag.
}
\item{
  Oldřich Krůza.
  Phonetic Transcription by Untrained Annotators.
  In Stani\-slav Krajči, editor,
  \textit{Proceedings of the 18th conference ITAT 2018:
  Slovenskočeský NLP workshop (SloNLP 2018)},
  volume 2203 of \textit{CEUR Workshop Proceedings},
  pages 35--40,
  Košice, Slovakia, 2018.
  Šafárik University, Košice,
  CreateSpace Indepedent Publishing Platform.
}
\item{
  Jan Oldřich Krůza.
  Spoken Corpus of Karel Makoň.  
  In \textit{Book of Abstracts XI International Conference on Corpus Linguistics},
  pages 189--190.
  ADEIT - Fundación Universidad-Empresa de la Universitat de València, 2019.\\
  \url{https://adeit-estaticos.econgres.es/19\_CILC/book\_abstracts.pdf}
}
\item{
  Jan Oldřich Krůza.
  Czech parliament meeting recordings as ASR training data.
  In \textit{Proceedings of the 2020 Federated Conference on Computer Science
  and Information Systems (FedCSIS)},
  pages 185--188.
  IEEE.\\
  \url{https://annals-csis.org/Volume\_21/drp/119.html}
}
\item{
  Matyáš Kopp and Vladislav Stankov and Jan Oldřich Krůza and Pavel Straňák and Ondřej Bojar.
  ParCzech 3.0: A Large Czech Speech Corpus with Rich Metadata.
  In \textit{International Conference on Text, Speech and Dialogue},
  Springer, 2021.
}
%\item{
%    Jan Oldřich Krůza.
%    Restructuring Spoken Corpus for Streaming Emulation.
%    In \textit
%}
\end{enumerate}


%%% Přílohy k disertační práci, existují-li. Každá příloha musí být alespoň jednou
%%% odkazována z vlastního textu práce. Přílohy se číslují.
%%%
%%% Do tištěné verze se spíše hodí přílohy, které lze číst a prohlížet (dodatečné
%%% tabulky a grafy, různé textové doplňky, ukázky výstupů z počítačových programů,
%%% apod.). Do elektronické verze se hodí přílohy, které budou spíše používány
%%% v elektronické podobě než čteny (zdrojové kódy programů, datové soubory,
%%% interaktivní grafy apod.). Elektronické přílohy se nahrávají do SISu a lze
%%% je také do práce vložit na CD/DVD. Povolené formáty souborů specifikuje
%%% opatření rektora č. 72/2017.
%\appendix
%\chapter{Přílohy}

%\section{První příloha}

\openright
\end{document}
