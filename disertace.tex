\documentclass[12pt,notitlepage]{report}
%\pagestyle{headings}
\pagestyle{plain}

\frenchspacing % aktivuje použití některých českých typografických pravidel

\usepackage[utf8]{inputenc} % nastavuje použité kódování cp1250
% \usepackage[czech]{babel}
\usepackage[czech]{babel}
\usepackage{a4wide} % nastavuje standardní evropský formát stránek A4
%\usepackage{index} % nutno použít v případě tvorby rejstříku balíčkem makeindex
%\usepackage{fancybox} % umožňuje pokročilé rámečkování :-)
\usepackage{graphicx} % nezbytné pro standardní vkládání obrázků do dokumentu

\usepackage[left=4cm,right=3cm,top=3cm,bottom=2cm]{geometry} % nastavení dané velikosti okrajů

\usepackage{rotating}
\usepackage{fontspec}

%\newindex{default}{idx}{ind}{Rejstřík} % zavádí rejstřík v případě použití balíku index

\title{Zpřístupnění zvukových nahrávek}   % tyto dvě položky jsou zde v podstatě formálně, 
\author{Mgr. Oldřich Krůza} % ve skutečnosti nejsou nikde dále v dokumentu použity

%\date{}

\begin{document}

%\hyphenation{tran-sla-tion}

%\csprimeson % zapne jednoduché psaní českých uvozovek pomocí klasických znaků, ale potom pozor 
            % na originální apostrofy, které budou chybně interpretovány!!!

\begin{titlepage}
\begin{center}

\vspace{15mm}

\large
Univerzita Karlova v~Praze\\
Matematicko-fyzikální fakulta\\

\vspace{5mm}

{\Large\bf DISERTAČNÍ PRÁCE}

\vspace{25mm}

\includegraphics[scale=0.5]{rc/mff-logo.eps} 

\vspace{25mm}
%\vspace{\fill}

%\normalsize

Mgr. Oldřich Krůza\\
\vspace{5mm}
{\Large\bf Zpřístupnění korpusu zvukových záznamů jednoho mluvčího}\\
\vspace{5mm}
Ústav formální a aplikované lingvistiky\\
\end{center}
\vspace{10mm}

\large
\noindent Vedoucí disertační práce práce: Doc. Vladislav Kuboň

\noindent Studijní program: Informatika, počítačová a formální lingvistika, Ph.D.

\vspace{20mm}

\begin{center}
2016
\end{center}

\end{titlepage} % zde končí úvodní strana

\thispagestyle{empty} %nastavuje prazdne zahlavi a zapati
\normalsize % nastavení normální velikosti fontu
\ \vspace{10mm} 

%\noindent Děkuji komu a komu

\vspace{\fill} % nastavuje dynamické umístění následujícího textu do spodní části stránky
\noindent Prohlašuji, že jsem svou disertační práci napsal samostatně a výhradně s~použitím citovaných pramenů. Souhlasím se zapůjčováním práce.

\bigskip
\noindent V~Praze dne \today \hspace{\fill}Mgr. Oldřich Krůza

\setcounter{page}{2} % nastavení číslování stránek
\tableofcontents % vkládá automaticky generovaný obsah dokumentu

\newpage % přechod na novou stránku

%%% Následuje strana s abstrakty.
\noindent Název práce: Iterativní zdokonalování přepisu zvukových nahrávek s využitím zpětné vazby posluchačů\\
Autor: Mgr. Jan Oldřich Krůza\\
Katedra (ústav): Ústav formální a aplikované lingvistiky\\
Vedoucí diplomové práce: Doc. RNDr. Vladislav Kuboň, Ph.D.\\
E-mail vedoucího: Vladislav.Kubon@mff.cuni.cz\\
Konzultant: Mgr. Nino Peterek, Ph.D.\\
E-mail konzultanta: Nino.Peterek@mff.cuni.cz\\

\noindent Abstrakt: Tato disertační práce se zabývá zpřístupněním zvukových
záznamů jednoho mluvčího úzké i široké veřejnosti.

Motivací práce byla existence chátrajících nahrávek hovorů českého filozofa
ing. Karla Makoně na kazetách a kotoučích. Cílem je zachování materiálu pro
budoucí generace a zpřístupnění nahrávek pomocí digitálních technologií,
především přístupnosti nahrávek na internetu a možnosti vyhledávání v~nich.

Práce představuje tvorbu systému pro přepis velké sady zvukových záznamů
se zapojením laické komunity. Navržené řešení spočívá ve vytvoření základního
přepisu nízké kvality pomocí automatického rozpoznávání řeči a vyvinutí
aplikace, která umožní od členů komunity i nahodilých zájemců získávat opravy
automatického přepisu, použitelné jako trénovací data pro další zlepšování.

Popíše se samotný mluvený korpus. Představí se autor a
jeho dílo,
témata v~nahrávkách, nahrávání samotné, digitalizace a získané přepisy.
Dále se rozvede tvorba systému pro automatický
přepis korpusu od akustického modelování, přes jazykové modelování, různé
provedené experimenty až k~vyhodnocení úspěšnosti. V~neposlední řadě se popíše
webová aplikace pro sběr manuálních přepisů. Zmíní se odlišnosti od ostatních mně známých
systémů, detaily návrhu a řešení, mechanismus pro kompenzaci vysokých nároků na kvalitu
přepisu a nízkých nároků na odbornost přispěvatelů a vyhodnocení funkčnosti
po osmi letech provozu.

% Výsledkem práce je především samotný mluvený korpus Karla Makoně,
% implementovaný systém pro komunitní přepis, 98 manuálně
% přepsaných hodin a automatické rozpoznání akusticky obtížného materiálu
% s~word error rate 38,48\%.

\vspace{10mm}

\noindent
Title: Iterative Improving of Transcribed Speech Recordings Exploiting Listener's Feedback\\
Author: Mgr. Jan Oldřich Krůza\\
Department: Institute of Formal and Applied Linguistics\\
Supervisor: Doc. RNDr. Vladislav Kuboň, Ph.D.\\
Supervisor's e-mail address: Vladislav.Kubon@mff.cuni.cz\\
Consultant: Mgr. Nino Peterek, Ph.D.\\
Consultant's e-mail address: Nino.Peterek@mff.cuni.cz\\

\noindent Abstract:  This Ph.D. thesis deals with making a corpus of audio
recordings of a single speaker accessible to wide public and interested community.

The work has been motivated by the existence of a set of perishing recordings
of the Czech philosopher Karel Makoň on magnetophone tapes. The aim is to conserve
the material for future generations and making it accessible using
digital technologies, in particular publishing the recordings online
and enabling the users to search through them.

The thesis introduces the creation of a system for transcribing a large set of
speech recordings employing a lay community. The solution designed is based on
obtaining a baseline low-quality transcription by means of automated speech
recognition and developing an application that allows for collecting corrections
of the automatic transcription in a fashion that makes it usable as training
data for further improvement of said transcription.

The spoken corpus itself is described. The
author and his works, topics covered in the talks, the process of recording
and digitization as well as the gained transcription are introduced.
Next, the development of a system for automated transcription of
the corpus, from acoustic modeling, to language modeling, various experiments
undertaken and evaluation are presented. Then, the web application for gathering
manual transcript corrections is described.
Differences to other settings, design and implementation details, a way to
compensate high demand for transcription quality and low demand for worker
expertise, as well as an evaluation of the system's performance after eight
years of operation are covered.

% The results comprise mainly the spoken corpus of Karel Makoň itself,
% the implemented system for community-driven transcription,
% 98 manually transcribed hours
% and automatically transcribed acoustically difficult material with word error
% rate of 38.48\%.


\newpage

%%% Následuje text bakalářské práce členěný do kapitol, které se číslují, označí názvy a graficky oddělí.
%%% Nedoporučuje se používat víc než dvě úrovně číslování kapitol.

\chapter{Úvod}
\label{kap:uvod}

TODO

\chapter{Předchozí práce}
\label{kap:predchozi}

TODO

\chapter{Data}
\label{kap:data}

Zvukový odkaz Karla Makoně je původní a ústřední motivací pro tuto práci. Osobně
považuji Makoňovo dílo za jedno z~nejzásadnějších vůbec v~oblasti duchovního
průkopnictví, a to jeho systematičností, obsáhlostí, návodností, novátorstvím a
především hloubkou. Jeho nauka se od moderních duchovních směrů odlišuje kladným
postojem k~civilizačnímim trendům, nikoliv jejich zavrhováním, dále
konzistentním souladem s~rozumovým poznáním a pevnými základy v~náboženských
tradicích. Od vědeckého bádání se odlišuje zejména tím, že rozum a hmotu
považuje za odrazové můstky k~hlubšímu poznání, nikoliv za vrchol a jedinou
platformu lidského poznání. Trvá však na tom, že duchovní zákonitosti jsou
stejně tak pevně dané, univerzální a ověřitelné (ovšem pouze osobní, subjektivní
zkušeností), jako zákony přírodní, popsané vědecky. Do třetice od klasické
křesťanské literatury se liší obzvláště tím, že Ježíšovu nauku považuje za návod
prvotřídní kvality pro vědomý vstup do věčného života zde na zemi a v~těle,
nikoliv po smrti. Věčným životem se míní stav, kdy člověk je vědomě věčnou
bytostí nezávislou na pomíjejícím těle. Tvrdě kritizuje překonaný a naivní
výklad, podle nějž se ctnostným životem dá dojít po smrti do nebe, dále dojít
spásy pouhou proklamací o~víře v~Krista a dodržováním přikázání a náboženských
obřadů.

Odkrývá smysl života a návod na jeho uskutečnění, který nestojí na slepé víře
ani na vlasní omezené lidské invenci.

\section{Karel Makoň}

\subsection{Duchovní vzestup Karla Makoně}

Ing. Karel Makoň se narodil 12. prosince 1912. Ve věku dvou let ho postihl zánět
levého ramene. Lékaři doporučovali amputaci ruky, k~čemuž jeho matka nedala
souhlas a na vlastní zodpovědnost nechala dítě operovat. Vzhledem k~tomu, že
ještě nebyly objeveny krevní skupiny, nebyla možná transfúze a proto musela být
operace prováděna opakovaně, aby dítě nevykrvácelo. Tehdejší anestetika nebylo
možné tak často podat tak mladému organizmu, proto byly operace prováděny při
vědomí. Malý Karel Makoň se zažívaje nesnesitelnou bolest naučil v~raném věku
opouštět svoje tělo. Tato opakovaná zkušenost měla u něho následek, že po
určitou dobu nepoznával svoji matku, zato začal spontánně rozpoznávat správné od
nesprávného a důsledně činit, co poznával jako správné.

Pro omezení rizika komplikací s~operovanou rukou bylo Karlovi zakázáno hrát si
s~dětmi. Svůj předškolní čas proto trávil sám na venkově jen se zvířaty. Díky
extrakorporálním zkušenostem se naučil rozumět řeči zvířat, obzvláště hus, které
ho vzaly za svého a s~nimiž pronikal do stavu zvířecího ráje.

Období ,,činění správného'', kdy si kupříkladu zapověděl kouření, alkohol i
veškerý pohlavní život, vyvrcholilo v~Makoňových sedmnácti letech, kdy si
přečetl myšlenku, že ,,tento život je mostem do věčnosti''. Tím započalo období
extází a vědomí, že je nesmrtelnou bytostí a smyslem jeho života je spojení
s~Bohem. Svojí matkou a prarodiči byl sice veden ke tradiční katolické víře, ale
nikdy na ni nepřistoupil, protože ,,v~nebi, kde by se jen díval na Boží tvář by
byla strašná nuda''. Nikdy tedy nevěřil, v~sedmnácti letech poznal.

V~tomto období se ustavičně modlil za to, aby dokázal Boha více milovat. Tato
modlitba trvala devět let a jejím vyvrcholením byla deportace do koncentračního
tábora v~Sachsenhausen v~roce 1939, coby českého vysokoškolského studenta.

V~koncentračním táboře byl Makoň sužován více, než ostatní: měl jakýsi
obzvláštní talent chytat rány a kopance. Prožíval nesmírné zmatení a frustraci
nad tím, že tak dlouho tak věrně sloužil Bohu, a teď se s~ním jedná jako s~kusem
hadru. Po čtyřech dnech utrpení nastal zlomový okamžik. Tamějším vězňům bylo
zakázáno pod trestem smrti přihlížet zabití spoluvězňů příslušníky SS. Makoň si
však nedal pozor a hleděl na právě takovou scénu. Vykonávající Němec si toho
povšiml a vyzval Karla Makoně, ať zůstane stát na místě, že hned, jak dobije
svoji momentální oběť, přijde zabít i jeho. Karel Makoň v~tu chvíli raději
odevzdal svůj život Bohu, a to bez přemýšlení a bezpodmínečně, se silou, nabytou
onou devítiletou modlitbou. Překvapivým výsledkem toho bylo, že SS-Mann tváří
v~tvář Makoňovi zbledl a v~hrůze se obrátil na útěk.

Makoň tehdy obdržel všeobjímající poznání smyslu života, jak svého, tak obecně,
a absolutní svobodu. Pohyboval se volně od baráku k~baráku, nezažíval hlad ani
jiný nedostatek, esesáci ho neviděli. Trávil svoje dny v~koncentračním táboře
burcováním ostatních k~probuzení k~pravdě, kterou sám zažíval.

Zanedlouho byl z~koncentračního tábora propuštěn a celý zbytek svého života
věnoval předávání svojí zkušenosti a hlavně návodu, jak k~ní přijít bez nutnosti
tak výrazných krizí, jakými sám v~životě procházel, a které považuje za
nepříkladné.

Zemřel v~roce 1993.

\subsection{Spisovatelská a přednášková činnost}

% získávání přepisů
%  iniciální přepis v Praatu nebo to byl Transcriber?
%  graf přibývání podle přispěvatelů, času, nahrávek atp.
% nahrávky samotné (z předchozích článků)
% grafické indexy

\chapter{Akustický model}
\label{kap:akusticky-model}

% - množina fonémů
% - vektorový formát
% - HMM X CNN
% - HTK X Kaldi X sphinx X Bourlard X TensorFlow
% - Monofonémy X trifonémy
% - mixtury
%   - individuální
%   - globální
%   - na monofonémech vs. trifonémech

Koncept celého projektu se zakládá na~přítomnosti automatického přepisu a jeho
následném zdokonalování. Jelikož nebyl k~dispozici uspokojivý hotový nástroj pro
získání automatického přepisu, nezbylo než jej vytvořit.

Zjednodušený řetězec vedoucí od~zvukových dat k~jejich přepisu v~našem případě
vypadá takto:\begin{enumerate}
\item{sběr trénovacích dat,}
\item{stavba akustického modelu,}
\item{stavba jazykového modelu,}
\item{automatické rozpoznávání.}
\end{enumerate}

V tomto řetězci je stavba akustického modelu patrně nevyznamnějším článkem a
sama tvoří řetěz o~mnoha článcích. V~této kapitole pojednáme o~krocích
podniknutých k~jeho sestavení, experimentech a různých volbách.

Do~užšího výběru potenciálních platforem tvorby akustického modelu jsem zařadil
starší systém \textit{HTK}\footnote{hmm toolkit; Hidden Markov Model Toolkit} a
modernější \textit{Kaldi}. HTK pomyslný konkurz nakonec vyhrál díky zkušenostem
Mgr. Nina Peterka, Ph.D. s~tímto systémem, z~níž jsem mohl čerpat.

Akustický model jsem trénoval výhradně z~vlastníh dat. Obětoval jsem tedy
potenciální přínos většího množství trénovacích dat a upřednostnil trénování
přímo na~konkrétního mluvčího.
%TODO: od kdy se vyplatí dělat model na konkrétního mluvčího?

Tvorba akustického modelu probíhala podle návodu v~manuálu k~HTK, \textit{HTK
Book}. V~hrubých rysech probíhá takto:

Blíže popíšu jen rozhodnutí, která nejsou zřejmá.

\section{Segmentace}

Zpravidla jeden zvukový soubor odpovídá jednomu přetočení magnetofonové pásky,
obvyklá délka je tedy 45 až 120 minut. Takto dlouhé úseky nelze použít ani jako
trénovací příklady ani jako cíl automatického rozpoznávání.

Celá aplikace je pojata jako nástroj pro~zdokonalování automatického přepisu,
takže vychází z~předpokladu, že nějaký přepis již existuje. Vycházeje z~téhož
předpokladu při~segmentaci, realizoval jsem ji tak, že zvukový soubor se rozdělí
na~úseky odpovídající jednotlivým větám v~přepisu, ne však delší než patnáct
sekund. Pokud by věta byla delší, rozdělí se u nejbližšího slova
před~patnáctisekundovou hranicí.

V~případě, že pro daný záznam zatím žádný přepis neexistuje, rozdělí se naivně
na~patnáctisekundové úseky.

Nutno dodat, že rozdělování podle automaticky rozpoznaných hranic vět by šlo
snadno vylepšit. Jde-li o~ručně přepsaná data, jsou hranice vět dobrým vodítkem,
ale u~automaticky přepsaných by bylo lépe rozdělovat podle ticha mezi slovy.

\section{Fonetika}

Množinu fonémů jsem použil od~doc. Pavla Ircinga ze~Západočeské univerzity
z~jeho skriptů z~devadesátých let minulého století. Fonémy mají tzv. pražský a
plzeňský zápis. V~souladu se~zjevným územ jsem použil plzeňský zápis.
V~tabulce~\ref{tab:phones} jsou uvedeny.

\begin{table}[htpb]
\fontspec{DoulosSIL}
\begin{center}
\begin{tabular}{|l|l|l|l||l|l|l|l|}
\hline
IPA & plz. & praž. & grafém & IPA & plz. & praž. & grafém \\
\hline
% TODO: IPA
a  & a   & a   & a      &     ɱ  & mg  & mg  & tra\underline{m}vaj \\
aː & aa  & aa  & á      &     n  & n   & n   & \underline{n}e \\
aʊ̯ & aw  & au  & au     &     ŋ  & ng  & ng  & pa\underline{n}t \\
b  & b   & b   & b      &     ɲ  & nj  & nj  & \v{n} \\
t͡s & c   & c   & c      &     o  & o   & o   & o \\
t͡ʃ & ch  & cz  & č      &     oː & oo  & oo  & ó \\
d  & d   & d   & d      &     oʊ̯ & ow  & ou  & ou \\
ɟ  & dj  & dj  & \v{d}  &     p  & p   & p   & p \\
d͡z & dz  & dz  & dz     &     r  & r   & r   & r \\
d͡ʒ & dzh & dzz & dž     &     r̝̊  & rsh & rsz & t\underline{\v{r}}i \\
ɛ  & e   & e   & e      &     r̝  & rzh & rzz & \underline{\v{r}}íz \\
ɛː & ee  & ee  & é      &     s  & s   & s   & s \\
eʊ̯ & ew  & eu  & eu     &     ʃ  & sh  & sz  & š \\
f  & f   & f   & f      &     t  & t   & t   & t \\
g  & g   & g   & g      &     c  & tj  & tj  & \v{t} \\
ɦ  & h   & h   & h      &     ʊ  & u   & u   & u \\
i  & i   & i   & i      &     uː & uu  & uu  & ú, \r{u} \\
iː & ii  & ii  & í      &     v  & v   & v   & v \\
j  & j   & j   & j      &     x  & x   & ch  & ch \\
k  & k   & k   & k      &     z  & z   & z   & z \\
l  & l   & l   & l      &     ʒ  & zh  & zz  & ž \\
m  & m   & m   & \underline{m}ák
                        &        &     &     & \\
\hline
\end{tabular}
\caption{použité fonémy: IPA, plzeňský zápis, pražský zápis a nejčastější
odpovídající grafém}\label{tab:phones}
\end{center}
\end{table}
\normalfont

V~závislosti na~množství trénovacích dat bylo vhodné nahradit některé fonémy
častějšími podobnými. V~tabulce~\ref{tab:phonesed} jsou záměny vyčísleny.
\begin{table}[htpb]
\fontspec{DoulosSIL}
\begin{center}
\begin{tabular}{|r|l|l||l|l|}
\hline
&
\multicolumn{2}{|c||}{před záměnou} &
\multicolumn{2}{|c|}{po záměně} \\
\hline
& IPA & plz. & IPA & plz. \\
\hline
    & ɱ  & mg & m & m \\
    & aʊ̯ & aw & a ʊ & a u \\
    & oː & oo & o & o \\
\** & d͡z & dz & t͡s & c \\
    & d͡ʒ & dzh & t͡ʃ & ch \\
\** & eʊ̯ & ew & ɛ ʊ & e u \\
\hline
\end{tabular}
\caption{použité záměny fonémů; hvězdičkou jsou vyznačeny záměny použité ještě
v~době psaní textu}\label{tab:phonesed}
\end{center}
\end{table}
\normalfont

\section{Rozdělení dat}

Pro natrénování modelu strojovým učením je potřeba trénovacích dat a pro
vyhodnocení jeho úspěšnosti dat testovacích, která ve~fázi trénování nesmí být
algoritmem spatřena. Při trénování samotném se pak mnohdy používá vyhrazených,
tzv.~\textit{heldout} dat pro průběžné měření úspěšnosti. V~případě trénování
akustického modelu s~použitím HTK je tomu nejinak. Heldout data jsou používána
pro zjištění optimálního počtu mixtur modelů jednotlivých fonémů, a testovací
pro závěrečné vyhodnocení.

Anotovaná data mi přibývala velice pozvolna a začínal jsem s~několika minutami,
ovšem přírůstky byly časté. Nemohl jsem si tedy dovolit udělat od~začátku pevnou
testovací sadu, kterou bych používal po~celou dobu provádění experimentů. Místo
toho jsem s~každou novou dávkou anotovaných dat celou sadu rozdělil podle vět
v~poměru 18:1:1 do trénovací, heldout a testovací sady. Tak jsem měl neustále
vyvážený poměr jednotlivých datových sad. Zřejmou velkou nevýhodou bylo, že
nešlo spolehlivě porovnávat výsledky jednotlivých experimentů vzhledem
k~variabilní testovací sadě.

Až když jsem měl několik desítek hodin anotovaných dat, vyhradil jsem si fixní
testovací sadu. Běžně se testovací sada vybere jako náhodná podmnožina vzorků
z~trénovací sady tak, aby měla kýženou velikost. V~mém případě vzorků zvíci
hodinových nahrávek jsem sadu určil manuálně jako úsek druhé až jedenácté minuty
(tedy deset minut minutu po~začátku) v~pěti nahrávkách,
\begin{enumerate}
\item{jedné kazety z~roku 1976,}
\item{jedné z~roku 1982,}
\item{jedné z~roku 1986,}
\item{jedné z~roku 1990 a}
\item{jednoho nedatovaného kotouče.}
\end{enumerate}

Sadu heldout nyní vybírám jako každou čtyřicátou větu. Z~každé dvacáté jsem
snížil na~polovic nejen abych neplýtval trénovacími daty, nýbrž také protože
vyhodnocování mixtur zabírá při trénování zdaleka nejvíce času, a ten je přímo
úměrný velikosti sady heldout.

\section{Spojování trifonémů}

% triphone-tree
% neznámé trifonémy

\chapter{Jazykový model}
\label{kap:jazykovy-model}

TODO

\chapter{Webové rozhraní}
\label{kap:webove-rozhrani}

\section{Prototyp}

Webové rozhraní, umožňující přístup zájemcům k~nahrávkám a jejich přepisu, byl
od začátku plánovanou součástí projektu. Rozhraní bylo navrženo s~těmito
požadovanými vlastnosmi:

\begin{itemize}
\item{výběr nahrávky ze~seznamu}
\item{poslech nahrávky s~obvyklými ovládacími prvky přehrávače}
\item{zobrazení přepisu nahrávky}
\item{vyznačení právě přehrávaného slova}
\item{možnost provést změnu v~přepisu}
\item{automatické zarovnání přepisu se~zvukem}
\item{případné odmítnutí přepisu, pokud zarovnání selže (přepis neodpovídá
vyřčeným slovům)}
\end{itemize}

První implementace byla založena na přehrávači \textit{jPlayer}, modulu pro
knihovnu \textit{jQuery}, který využívá standard HTML5 s~jeho elementem
\texttt{<audio>} a technologii \textit{Adobe Flash}. Pro~dynamickou odezvu
zobrazených prvků na~změny v~datovém modelu jsem použil knihovnu
\textit{knockout}.

Aplikace měla formu jediné stránky s~rozbalovatelným výběrem nahrávky,
ovládacími prvky přehrávače a třemi řádky přepisu. Při označení části
zobrazeného přepisu se stránka překryla rozhraním pro opravu přepisu, jež zvu
\textit{editačním okénkem}. V~editačním okénku se zobrazilo vstupní pole
(\texttt{<textarea>}) s předvyplněným současným přepisem, ovládací prvky pro
přehrátí odpovídající pasáže, odeslání opraveného přepisu a opuštění editačního
okénka.

Šlo o~prostou statickou HTML stránku s~JavaScriptem. K~audiu se přistupovalo
pomocí externí CDN, zatímco přepisy a API pro~zarovnávání oprav byly
na~zvláštním serveru.

Přepis byl uložen a přenášen ve~formátu \textit{JSONp}\footnote{JSON =
JavaScript Object Notation, JSONp = JSON with Padding}, čili jako \textit{JSON}
obalený v~javascriptové funkci kvůli zamezení problémů s~přístupem napříč
doménami.  Každé slovo s~sebou neslo informaci o~svojí pozici v~nahrávce
s~přesností na~setiny sekundy, výslovnost, zápis, slovníkovou formu, délku ticha
za~slověm, informaci o~tom, zda bylo manuálně přepsáno nebo automaticky
rozpoznáno, a v~případě automaticky rozpoznaných slov \textit{confidence
measure} čili míru jistoty rozpoznání.

Převod ze~zápisu slova do~jeho fonetické podoby se děje na~základě pravidlového
algoritmu z~dílny Doc. Pavla Ircinga po~úpravě od~Mgr.~Nina Peterka, Ph.D. Tento
algoritmus zahrnuje časté výjimky z~českých výslovnostních pravidel, ale
neobsahuje rozsáhlý výslovnostní slovník cizích slov. Karel Makoň navíc nezřídka
hovoří o~osobách, jejichž jména se v~mnoha korpusech neobjeví vůbec.

Nad rámec výše popsaných funkcionalit přibyly další na~základě přání uživatelů a
autorovy potřeby:

\begin{itemize}
\item{indikace, do~jaké míry je která nahrávka přepsána,}
\item{manuální posouvání hranic přepisovaného zvukového úseku,}
\item{úprava zápisu slova s~ponecháním výslovnosti,}
\item{identifikace uživatelů včetně sezení, prohlížeče atp.,}
\item{vyhledávání v~přepisech.}
\end{itemize}

Výslovnost je interně zaznamenána pomocí mezerou oddělených fonémů, z~nichž
každý má reprezentaci z~malých písmen sady ASCII. Taková reprezentace není pro
laické uživatele praktická, proto je součástí aplikace převod do~českého
fonetického zápisu a zpět.

Uživatelé jsou proto instruováni, aby slova s~nestandardní výslovností, na~která
narazí poprvé, přepsali foneticky \textit{(džordž)} a poté, co uspěje
automatické zarovnání, slovu opravil přepis \textit{(George)}.

Tato původní verze posloužila k~přepsání asi 600 tisíc slov a běžela asi 5 let,
než bylo nutné ji nahradit.

\section{Verze 2}

Pro kompletní přepis aplikace se postupně objevilo několik důvodů. Hlavním
z~nich bylo, že původní aplikace mohla jen těžko sloužit pro širokou veřejnost
jako prostředek k~popularizaci nahrávek. Dalším důvodem bylo, že některé kýžené
funkce nebylo možné zprovoznit bez zásadních změn v~provedení. Především šlo
o~ekvalizér, čili frekvenční korekci při poslechu. Akutním důvodem pak byl fakt,
že všechny významné prohlížeče opouštěly podporu Flashe.

Pro novou verzi jsem zvolil technologie React + Redux jako aplikační rámec, Web
Audio API jako platformu pro nakládání se~zvukem a Twitter Bootstrap jako základ
pro vzhled prvků. Zdrojový kód píšu v~ECMAScript~6 a o~kompilaci se stará
webpack.

\subsection{Výběr nahrávky}

První verze obsahovala všudypřítomný rozklikávatelný kategorizovaný seznam
nahrávek. Toto jednoduché řešení mělo jen málo nevýhod. Jedna z~nich byla, že
nešlo použít běžné textové vyhledávání. V~nové verzi je proto použit
dvousloupcový formát, kde vlevo je rozbalovací seznam kategorií a vpravo
lineární seznam nahrávek. Jednotlivé kategorie jsou pak skrolovacími odkazy
do~pravého sloupce a podle stupně skrolování se příslušná kategorie sama
rozbalí (tzv. scrollspy).

Pro lepší přehlednost a v souladu s principem Separation of Concerns je seznam
nahrávek pouze na~úvodní stránce a nikoliv všude.

\subsection{Zobrazení přepisu}

V~první verzi se zobrazovaly vždy tři řádky, kde v~prostředním bylo aktuálně
přehrávané slovo. Výjimkou samozřejmě byly případy, kdy se přehrával začátek
nebo konec nahrávky. Délka řádku odpovídala šířce okna prohlížeče. Takto malý
rozsah zobrazeného textu byl zvolen proto, že aby bylo možné vizuálně odlišit
ručně přepsaná slova od~automaticky rozpoznaných a od~nich ještě slovo aktuálně
přehrávané, muselo být každé slovo obaleno ve~vlastním HTML elementu. Při~větším
množství slov pak bylo rozhraní velice náročné na~výpočetní výkon a znatelně
pomalé, což při synchronním zobrazování přepisu s~přehráváním zvuku není
přijatelné.

Zobrazení jen tří řádků mělo pochopitelně velké nevýhody. Především to, že se
člověk nemohl zorientovat v~širším rámci nahrávky (jejíž průměrná délka je kolem
hodiny) a že opět nebylo možné vyhledávat v~jejím rámci. Výhodou naopak bylo, že
aktuálně přehrávané slovo bylo vždy snadné najít. Nemožnost označit a tedy ani
přepsat příliš dlouhý úsek bylo pro uživatele možná někdy nepříjemné, ale
redukovalo chyby jak v~přepisu, tak v~automatickém zarovnávání.

Zobrazení celého přepisu při zachování plynulosti a grafickém odlišení
automaticky a manuálně přepsaných slov, slova aktuálně přehrávaného a navíc
slova vybraného kliknutím bylo první velkou výzvou pro návrh stránky přehrávání.
První, nezaslouženou pomocí k~tomu byl vývoj výkonu počítačů od~vzniku první
verze, jakož i optimalizace prohlížečů na~rychlost. Množství elementů, které lze
nyní realisticky zobrazit, se znatelně zvýšilo, ačkoliv naivní řešení zabalení
každého slova stále není praktické. Druhým pomocným faktorem je, že manuálně a
automaticky přepsaná slova se většinou vyskytují ve~větších shlucích. Jen
v~málokterých souborech se často střídají manuálně a automaticky přepsané úseky.
To vypovídá o~nevoli uživatelů k~jinému než kompletnímu, lineárnímu přepisu.
% TODO: ref. aktivní učení
Každý shluk manuálně respektive automaticky přepsaných slov stačí tedy zabalit
do~jednoho elementu.

Zvýraznění jednotlivých slov -- aktuálně přehrávaného a vybraného klikem myší --
se realizuje pomocí umístění barevného rámečku pod~zvýrazněné slovo. To lze
provést díky tomu, že prohlížeče nyní umožňují zjistit polohu označeného textu a
označení lze provést automaticky.\footnote{Viz \texttt{getClientRects} a
\texttt{Range} ve~webových standardech.}

\subsection{Web Audio API}

Přechod na~tuto technologii umožnil některé pokročilé funkce, avšak za~relativně
vysokou cenu. Web Audio API je standard pro~pokročilé zpracování zvukového
signálu v~prohlížeči. Základním konceptem je graf procesních uzlů, které mají
vstup a výstup a mohou se libovolně propojovat. K~dispozici jsou zdroje zvuku
jako oscilátory nebo přehrávače streamů, souborů (tag \texttt{<audio>}) a dat
v~paměti (\texttt{AudioBuffer}) a efekty jako zesílení, dynamická komprese,
mixování kanálů.

Velká výhoda Web Audio API oproti elementu \texttt{<audio>} je možnost přesného
časování, až na~jednotlivé samply. Přehrávání výseku odpovídajícího označenému
textu se proto nemusí provádět pomocí velice nepřesného časovače
\texttt{setTimeout}.

Bez~Web Audio API by také nebylo možné provádět frekvenční korekci při poslechu,
čili mít tzv. \textit{ekvalizér}. Ten je zapotřebí, protože některé nahrávky
mají v~určitém frekvenčním pásmu silný šum, jehož odstranění je s~ekvalizérem
snadné a komfort poslechu se tak razantně zvýší.

Další funkcí, kterou Web Audio API umožňuje, je stahování úseků. Označením
přepsaného textu se definuje úsek nahrávky a ten je možné uložit bez~dalšího
síťového přenosu. Tato funkce však vyžaduje, aby nahrávka byla dekódovaná
v~paměti. Vzhledem k~tomu, že nahrávky mají běžně i hodinu a půl, trvá její
stažení a dekódování opravdu dlouho a navíc prohlížeč kvůli tomu spotřebuje přes
gigabyte operační paměti.

Jsou plány na~to, aby Web Audio API umožnila dekódovat jen část
nahrávky,\footnote{github.com/WebAudio/web-audio-api/issues/1305} proto tento
neutěšený stav zatím nechávám být. Případným řešením by mohlo být buď
streamování (\texttt{<audio>} jako zdrojový uzel) a stažení pouze při potřebě
ukládání úseku nebo změna uložení nahrávek nikoliv jeden soubor na jednu pásku,
nýbrž např. po~minutových úsecích.

Díky tomu, že Web Audio API umožňuje přehrávání binárních dat z~proměnné
v~paměti, nabízí se dekódovanou nahrávku uložit na~persistentní úložiště
uživatelova počítače a při opětovné návštěvě stránky data místo stahování odsud
nahrát.

Moderní prohlížeče poskytují několik bran k~úložišti na~místním disku.
Nejtradičnějšími jsou bezesporu \textit{cookies}, které jsou však pro ukládání
objemnějších dat zcela nepoužitelné. Velice slibnou se jeví
\textit{localStorage}, umožňující ukládání párů klíč-hodnota. I zde však
narážíme na~příliš omezující kvóty. Kupříkladu Firefox ji má na 10MB, přičemž
potřeba je asi 1GB. Dalším kandidátem je \textit{File System API}. Tento
standard pro~izolovaný souborový systém k~dispozici webové aplikaci je zcela
ideálním řešením -- dá se zde i explicitně požádat o~konkrétní diskovou kvótu a
uživatel tak má volbu bez nutnosti práce programátora webové aplikace. Kamenem
úrazu je zde však podpora, která se momentálně omezuje pouze na Google Chrome.

Naštěstí existuje ještě standard \textit{IndexedDB API}, který má uspokojivou
podporu a uložení gigabytu dat je s~ním možné, byť ne zaručené. S~využitím
abstrahující knihovny \textit{Dexie} je proto skrz tento standard ukládání
implementováno. Pro uživatele, kteří delší dobu pracují na jedné a téže nahrávce
se tím přináší velká úspora času a přenesených dat.

\section{Rozdělení nahrávek na úseky}

Vzhledem k~tomu, že ani v~roce 2019 není kurzorový přístup ke zvukovým datům
skrze Web Audio API v~dohlednu, a k tomu jak odrazující dopad má nutnost
stahovat a dekódovat celou nahrávku aspoň při jejím prvním načtení, nezbylo mi,
než změnit způsob, jakým jsou nahrávky uloženy.

Nahrávky jsou uloženy v~několika instancích pro různé účely:

\begin{enumerate}
\item{na backendovém serveru ve formátu MFCC pro nucené zarovnávání,}
\item{v~repozitáři LINDAT ve formátu FLAC za účelem archivace a bádání,}
\item{na CDN ve formátu mp3 za účelem přímého stažení uživatelem,}
\item{taktéž na CDN ve formátech OGG/Vorbis a mp3 pro webové rozhraní.}
\end{enumerate}

Pouze poslední jmenovanou instanci je žádoucí ukládat tak, aby každý soubor byl
jen tak velký, aby jeho stažení a dekódování trvalo únosně dlouho. V~ostatních
případech je lépe zachovat uložení, kde jedna nahrávka odpovídající většinou
jedné straně kazety či jednomu průchodu pásky z~kotouč na kotouč. Třetí a čtvrtá
instance však navzdory rozdílnému účelu sdílejí tatáž data. Bylo proto nutné je
duplikovat.

\subsection{Délka segmentů}

Délka úseků, na které nahrávky rozděluji, ovlivňuje, jak dlouho se každý segment
bude stahovat a dekódovat. Čas stahování a dekódování segmentu, který obsahuje
slovo, na němž je kurzor při prvním požadavku o~přehrávání, je roven zpoždění od
uživatelské akce k začátku přehrávání. Podle internetového periodika
UXMovement\cite{foursecondrule}, začíná uživatel po čtyřech sekundách čekání
upouštět od předchozího záměru. Podle článku Nielsen Norman
Group\cite{websiteresponsetimes} je hranice únosnosti 10 sekund.

Pokud budou úseky příliš dlouhé, jejich stahování a dekódování zabere příliš
mnoho času. Na druhou stranu s každým předělem vnášíme do přehrávání bod, kde se
úseky nalepují a může tam vyvstat artefakt. Také s~každým segmentem se pojí
extra HTTP request s~nezanedbatelnou režií.

Jako vhodný kompromis se jeví segmenty o délce 30 - 120 sekund. Velikost
dvouminutového segmentu je v~komprimovaném formátu při mono/24KHz kolem 0.6MB a
na Intel Core2 o 2,5GHz se dekóduje asi 1.6 sekundy.

\subsection{Hledání bodů předělu}

Vhodným výběrem bodů předělu můžeme omezit dopad případných artefaktů
způsobených nepřesným navázáním. Ideálním by bylo dělit nahrávky v~momentech
ticha. Ne vždy jsou momenty ticha každé dvě minuty, proto z momentů ticha
ustupme k~požadavku pauzy v~řeči. Hovořit dvě minuty bez nádechu hraničí
s~nemožností. Potýkáme se tedy s~úlohou nalézt pauzy v~řeči. Jednak je třeba
ujasnit, podle jakého klíče budeme pauzy vybírat, a jednak, jak je budeme přesně
hledat.

Hledat pauzy v~řeči lze různými způsoby. Nejspolehlivější a nejnáročnější je
manuální označování pauz. Pokoušel jsem se o~to sám a dosáhl jsem rychlosti
přibližně čtyřnásobku rychlosti přehrávání, tedy jeden zapsaný bod předělu za
třicet sekund. Dalších asi pět dobrovolných anotátorů se o~tento úkol pokusilo a
došla jim trpělivost po nule až deseti minutách označkovaného materiálu.

Další velice spolehlivou metodou je hledání podle predikovaných pseudofonémů
ticha v~zarovnaném přepisu. Tuto metodu jsem mohl namnoze použít, neboť
k~většině nahrávek mám automatický nebo i manuální přepis.

Tam, kde pořízení přepisu nebo jeho automatické zarovnání selhalo, lze použít
detekci ticha prostou akustickou analýzou. Tato metoda je velice náchylná
k~chybám v~případě nahrávek s~malým poměrem signálu k~šumu, kterých se v~korpusu
Karla Makoně vyskytuje neutěšeně mnoho.

\chapter{Vyhledávání}
\label{kap:vyhledavani}

Možnost vyhledávat v~nahrávkách byl pro mne jeden z~hlavních cílů od začátku
projektu. Se~získáním přepisů náhrávek, byť kolísavé kvality, bylo možné
vyhledávání realizovat.
Fulltextové vyhledávání jsem implementoval nástroje Elastic.

Elastic je svobodný vyhledávač napsaný v~Javě, který umožňuje fulltextové
vyhledávání v~dokumentech. Dokumenty se rozumí datové struktury, které se
vyhledávači poskytnou ve formátu JSON. Elastic má mnoho funkcionalit,
z~nichž pro mne je klíčové rozhraní na základě HTTP naplňující konvence REST,
automatický stemming, zvýrazňování nalezených pasáží a možnost vyhledávat
v~libovolných položkách dokumentu.

Aby bylo možné každý nalezený výsledek proměnit v~odkaz na příslušnou pasáž
v~nahrávce, zvolil jsem za jednotlivé dokumenty nikoliv celé nahrávky, nýbrž
věty.

Ke každému dokumentu se ukládá
\begin{itemize}
\item{textová reprezentace,}
\item{posloupnost hlásek,}
\item{stupeň
manuálního přepisu, tedy zda je přepis pořízen zcela automaticky, zcela manuálně
nebo kombinací obého}
\item{a také vektor confidence measure jednotlivých automaticky přepsaných slov.}
\end{itemize}

Pro skloňování je použito pravidlového stemmingu, který je dodáván s~distribucí
Elastic a pro češtinu, obzvlášť tam, kde se vyskytuje mnoho
nestandardních a archaických slov, funguje báječně.

Momentálně je vyhledávač nainstalován na témž serveru jako API a je dostupný
z~webové aplikace. Důležitým bodem budoucí práce je automatizace indexování
manuálních oprav, jak přicházejí. Dále pak zakomponování automatického přepisu
pořízeného bez použití jazykového modelu, jak se diskutuje
v~podsekci~\ref{ssec:data:topicsearch}.

\section{Případová studie}

Vyhledávání v~přepisech mluveného korpusu našlo využití v~kompilaci referátů o
určitých tématech, kterým se Karel Makoň věnuje. V průběhu let 2018 až 2020
vznikly alespoň čtyři takové, a to na témata
\begin{itemize}
\item{karma,}
\item{převtělování,}
\item{Otčenáš,}
\item{relativní dobro a zlo.}
\end{itemize}
Každé téma bylo zpracováno do formy souboru krátkých úseků nahrávek, které se
prezentovaly sekvenčním přehráním s~podporou přepisů jako zrakového vodítka.

Autor referátu o tématu relativního dobra a zla dohledal poznámkový aparát
k~tvorbě a rekonstruoval svůj postup, který zde popíšu jako příklad použití
přepisů, z~něhož lze usoudit na efektivitu práce.

Téma relativního dobra a zla bylo předem zamyšleno a bylo vybráno pro autorův
zájem a nikoliv s~ohledem na to, jak snadné bude pro vyhledání. Dopředu byl dán
časový rámec výsledku na cca. dvě hodiny zvukového záznamu.

Autorova metodika byla následovná: vyhledal frázi ,,relativní dobro a zlo`` a
výsledky procházel ve výchozím relevančním řazení nástroje Elastic. Prošel
prvních sto z~celkových 7379 výsledků. U každého posoudil, zda se jedná o
pasáž, skutečně o tématu pojednávající, nebo jen o letmou zmínku či vůbec
falešný zásah, popřípadě o duplicitní výskyt.

Výběr vzorků probíhal ve dvou průchodech. V~prvním autor vybíral relevantní a
relativně obsáhlejší zásahy, čímž se tvořila užší množina kandidátských úseků.
V~druhém průchodu pak vybíral z~této užší množiny s~ohledem na ročník zdrojové
nahrávky, aby byl v~kompilátu zastoupen průřez vývoje Makoňova myšlení,
výjimečně podle návaznosti či pro závěrečnou část shrnující charakter výpovědi.

Po prvním průchodu se do užšího výběru dostalo 25 nalezených úseků, tedy
čtvrtina procházené množiny, a všechny byly z~prvních 53 zásahů. V~první stovce
výsledků vyhledávání autor identifikoval 16 duplicit.

\chapter{Popularizace}
\label{kap:popularizace}

TODO

\chapter{Závěr}
\label{kap:zaver}

TODO


\bibliographystyle{unsrt}
\bibliography{citace}

\end{document}
