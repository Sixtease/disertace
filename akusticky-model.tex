\chapter{Akustický model}
\label{kap:akusticky-model}

% - množina fonémů
% - vektorový formát
% - HMM X CNN
% - HTK X Kaldi X sphinx X Bourlard X TensorFlow
% - Monofonémy X trifonémy
% - mixtury
%   - individuální
%   - globální
%   - na monofonémech vs. trifonémech

Koncept celého projektu se zakládá na~přítomnosti automatického přepisu a jeho
následném zdokonalování. Jelikož nebyl k~dispozici uspokojivý hotový nástroj pro
získání automatického přepisu, nezbylo než jej vytvořit.

Zjednodušený řetězec vedoucí od~zvukových dat k~jejich přepisu v~našem případě
vypadá takto:\begin{enumerate}
\item{sběr trénovacích dat,}
\item{stavba akustického modelu,}
\item{stavba jazykového modelu,}
\item{automatické rozpoznávání.}
\end{enumerate}

V tomto řetězci je stavba akustického modelu patrně nevyznamnějším článkem a
sama tvoří řetěz o~mnoha článcích. V~této kapitole pojednáme o~krocích
podniknutých k~jeho sestavení, experimentech a různých volbách.

Do~užšího výběru potenciálních platforem tvorby akustického modelu jsem zařadil
starší systém \textit{HTK}\footnote{hmm toolkit; Hidden Markov Model Toolkit} a
modernější \textit{Kaldi}. HTK pomyslný konkurz nakonec vyhrál díky zkušenostem
Mgr. Nina Peterka, Ph.D. s~tímto systémem, z~níž jsem mohl čerpat.

Akustický model jsem trénoval výhradně z~vlastníh dat. Obětoval jsem tedy
potenciální přínos většího množství trénovacích dat a upřednostnil trénování
přímo na~konkrétního mluvčího.
%TODO: od kdy se vyplatí dělat model na konkrétního mluvčího?

Tvorba akustického modelu probíhala podle návodu v~manuálu k~HTK, \textit{HTK
Book}. Popíšu tedy jen rozhodnutí, která nejsou zřejmá.

\section{Segmentace}

Zpravidla jeden zvukový soubor odpovídá jednomu přetočení magnetofonové pásky,
obvyklá délka je tedy 45 až 120 minut. Takto dlouhé úseky nelze použít ani jako
trénovací příklady ani jako cíl automatického rozpoznávání.

Celá aplikace je pojata jako nástroj pro~zdokonalování automatického přepisu,
takže vychází z~předpokladu, že nějaký přepis již existuje. Vycházeje z~téhož
předpokladu při~segmentaci, realizoval jsem ji tak, že zvukový soubor se rozdělí
na~úseky odpovídající jednotlivým větám v~přepisu, ne však delší než patnáct
sekund. Pokud by věta byla delší, rozdělí se u nejbližšího slova
před~patnáctisekundovou hranicí.

V~případě, že pro daný záznam zatím žádný přepis neexistuje, rozdělí se naivně
na~patnáctisekundové úseky.

Nutno dodat, že rozdělování podle automaticky rozpoznaných hranic vět by šlo
snadno vylepšit. Jde-li o~ručně přepsaná data, jsou hranice vět dobrým vodítkem,
ale u~automaticky přepsaných by bylo lépe rozdělovat podle ticha mezi slovy.

\section{Fonetika}

Množinu fonémů jsem použil od~doc. Pavla Ircinga ze~Západočeské univerzity
z~jeho skriptů z~devadesátých let minulého století. Fonémy mají tzv. pražský a
plzeňský zápis. V~souladu se~zjevným územ jsem použil plzeňský zápis.

\begin{table}[htpb]
\begin{center}
\begin{tabular}{|l|l|l|l||l|l|l|l|}
\hline
IPA & plz. & praž. & grafém & IPA & plz. & praž. & grafém \\
\hline
\fontspec{DoulosSIL}
% TODO: IPA
a & a & a & a &                m & mg & mg & sa\textbf{m}ba \\
aː & aa & aa & á &               & mv & mv & tra\textbf{m}vaj \\
aʊ̯ & aw & au & au &              & n & n & \textbf{n}e \\
b & b & b & b &                 & ng & ng & pa\textbf{n}t \\
t͡s & c & c & c &                 & nj & nj & \v{n} \\
t͡ʃ & ch & cz & č &               & o & o & o \\
d & d & d & d &                 & oo & oo & ó \\
ɟ & dj & dj & \v{d} &           & ow & ou & ou \\
d͡z & dz & dz & dz &              & p & p & p \\
d͡ʒ & dzh & dzz & dž &            & r & r & r \\
ɛ & e & e & e &                 & rsh & rsz & t\textbf{\v{r}}i \\
ɛː & ee & ee & é &               & rzh & rzz & \textbf{\v{r}}íz \\
eʊ̯ & ew & eu & eu &              & s & s & s \\
f & f & f & f &                 & sh & sz & š \\
g & g & g & g &                 & t & t & t \\
ɦ & h & h & h &                 & tj & tj & \v{t} \\
i & i & i & i &                 & u & u & u \\
iː & ii & ii & í &               & uu & uu & ú, \r{u} \\
j & j & j & j &                 & v & v & v \\
k & k & k & k &                 & x & ch & ch \\
l & l & l & l &                 & z & z & z \\
m & m & m & \textbf{m}ák &      & zh & zz & ž \\
\normalfont
\hline
\end{tabular}
\caption{použité fonémy: plzeňský zápis, pražský zápis a nejčastější
odpovídající grafém}\label{tab:phones}
\end{center}
\end{table}

V~závislosti na~množství trénovacích dat bylo vhodné nahradit některé fonémy
častějšími podobnými. Jednalo se o~tyto záměny:

θɪsɪzsəmaɪpeɪ

