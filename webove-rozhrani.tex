\chapter{Webové rozhraní}
\label{kap:webove-rozhrani}

\section{Prototyp}

Webové rozhraní, umožňující přístup zájemcům k~nahrávkám a jejich přepisu, bylo
od začátku plánovanou součástí projektu. Rozhraní bylo navrženo s~těmito
požadovanými vlastnosmi:

\begin{itemize}
\item{výběr nahrávky ze~seznamu}
\item{poslech nahrávky s~obvyklými ovládacími prvky přehrávače}
\item{zobrazení přepisu nahrávky}
\item{vyznačení právě přehrávaného slova}
\item{možnost provést změnu v~přepisu}
\item{automatické zarovnání přepisu se~zvukem}
\item{případné odmítnutí přepisu, pokud zarovnání selže (přepis neodpovídá
vyřčeným slovům)}
\end{itemize}

První implementace byla založena na přehrávači \textit{jPlayer}, modulu pro
knihovnu \textit{jQuery}, který využívá standard HTML5 s~jeho elementem
\texttt{<audio>} a technologii \textit{Adobe Flash}. Pro~dynamickou odezvu
zobrazených prvků na~změny v~datovém modelu jsem použil knihovnu
\textit{knockout}.

Aplikace měla formu jediné stránky s~rozbalovatelným výběrem nahrávky,
ovládacími prvky přehrávače a třemi řádky přepisu. Při označení části
zobrazeného přepisu se stránka překryla rozhraním pro opravu přepisu, jež zvu
\textit{editačním okénkem}. V~editačním okénku se zobrazilo vstupní pole
(\texttt{<textarea>}) s předvyplněným současným přepisem, ovládací prvky pro
přehrátí odpovídající pasáže, odeslání opraveného přepisu a opuštění editačního
okénka.

Šlo o~prostou statickou HTML stránku s~JavaScriptem. K~audiu se přistupovalo
pomocí externí CDN, zatímco přepisy a API pro~zarovnávání oprav byly
na~zvláštním serveru.

Přepis byl uložen a přenášen ve~formátu \textit{JSONp}\footnote{JSON =
JavaScript Object Notation, JSONp = JSON with Padding}, čili jako \textit{JSON}
obalený v~javascriptové funkci kvůli zamezení problémů s~přístupem napříč
doménami.  Každé slovo s~sebou neslo informaci o~svojí pozici v~nahrávce
s~přesností na~setiny sekundy, výslovnost, zápis, slovníkovou formu, délku ticha
za~slověm, informaci o~tom, zda bylo manuálně přepsáno nebo automaticky
rozpoznáno, a v~případě automaticky rozpoznaných slov \textit{confidence
measure} čili míru jistoty rozpoznání.

Převod ze~zápisu slova do~jeho fonetické podoby se děje na~základě pravidlového
algoritmu z~dílny Doc. Pavla Ircinga po~úpravě od~Mgr.~Nina Peterka, Ph.D. Tento
algoritmus zahrnuje časté výjimky z~českých výslovnostních pravidel, ale
neobsahuje rozsáhlý výslovnostní slovník cizích slov. Karel Makoň navíc nezřídka
hovoří o~osobách, jejichž jména se v~mnoha korpusech neobjeví vůbec.

Nad rámec výše popsaných funkcionalit přibyly další na~základě přání uživatelů a
autorovy potřeby:

\begin{itemize}
\item{indikace, do~jaké míry je která nahrávka přepsána,}
\item{manuální posouvání hranic přepisovaného zvukového úseku,}
\item{úprava zápisu slova s~ponecháním výslovnosti,}
\item{identifikace uživatelů včetně sezení, prohlížeče atp.,}
\item{vyhledávání v~přepisech.}
\end{itemize}

Tato původní verze posloužila k~přepsání asi 600 tisíc slov a běžela asi 5 let,
než bylo nutné ji nahradit.

\section{Verze 2}

Pro kompletní přepis aplikace se postupně objevilo několik důvodů. Hlavním
z~nich bylo, že původní aplikace mohla jen těžko sloužit pro širokou veřejnost
jako prostředek k~popularizaci nahrávek. Dalším důvodem bylo, že některé kýžené
funkce nebylo možné zprovoznit bez zásadních změn v~provedení. Především šlo
o~ekvalizér, čili frekvenční korekci při poslechu. Akutním důvodem pak byl fakt,
že všechny významné prohlížeče opouštěly podporu Flashe.

Pro novou verzi jsem zvolil technologie React + Redux\cite{abramov2015redux} jako aplikační rámec, Web
Audio API\cite{adenot2013web} jako platformu pro nakládání se~zvukem a Twitter Bootstrap jako základ
pro vzhled prvků. Zdrojový kód píšu v~ECMAScript~6 a o~kompilaci se stará
webpack.

\subsection{Výběr nahrávky}

První verze obsahovala všudypřítomný rozklikávatelný kategorizovaný seznam
nahrávek. Toto jednoduché řešení mělo jen málo nevýhod. Jedna z~nich byla, že
nešlo použít běžné textové vyhledávání. V~nové verzi je proto použit
dvousloupcový formát, kde vlevo je rozbalovací seznam kategorií a vpravo
lineární seznam nahrávek. Jednotlivé kategorie jsou pak skrolovacími odkazy
do~pravého sloupce a podle stupně skrolování se příslušná kategorie sama
rozbalí (tzv. scrollspy).

Pro lepší přehlednost a v souladu s principem Separation of Concerns je seznam
nahrávek pouze na~úvodní stránce a nikoliv všude.

\subsection{Zobrazení přepisu}

V~první verzi se zobrazovaly vždy tři řádky, kde v~prostředním bylo aktuálně
přehrávané slovo. Výjimkou samozřejmě byly případy, kdy se přehrával začátek
nebo konec nahrávky. Délka řádku odpovídala šířce okna prohlížeče. Takto malý
rozsah zobrazeného textu byl zvolen proto, že aby bylo možné vizuálně odlišit
ručně přepsaná slova od~automaticky rozpoznaných a od~nich ještě slovo aktuálně
přehrávané, muselo být každé slovo obaleno ve~vlastním HTML elementu. Při~větším
množství slov pak bylo rozhraní velice náročné na~výpočetní výkon a znatelně
pomalé, což při synchronním zobrazování přepisu s~přehráváním zvuku není
přijatelné.

Zobrazení jen tří řádků mělo pochopitelně velké nevýhody. Především to, že se
člověk nemohl zorientovat v~širším rámci nahrávky (jejíž průměrná délka je kolem
hodiny) a že opět nebylo možné vyhledávat v~jejím rámci. Výhodou naopak bylo, že
aktuálně přehrávané slovo bylo vždy snadné najít. Nemožnost označit a tedy ani
přepsat příliš dlouhý úsek bylo pro uživatele možná někdy nepříjemné, ale
redukovalo chyby jak v~přepisu, tak v~automatickém zarovnávání.

Zobrazení celého přepisu při zachování plynulosti a grafickém odlišení
automaticky a manuálně přepsaných slov, slova aktuálně přehrávaného a navíc
slova vybraného kliknutím bylo první velkou výzvou pro návrh stránky přehrávání.
První, nezaslouženou pomocí k~tomu byl vývoj výkonu počítačů od~vzniku první
verze, jakož i optimalizace prohlížečů na~rychlost. Množství elementů, které lze
nyní realisticky zobrazit, se znatelně zvýšilo, ačkoliv naivní řešení zabalení
každého slova stále není praktické. Druhým pomocným faktorem je, že manuálně a
automaticky přepsaná slova se většinou vyskytují ve~větších shlucích. Jen
v~málokterých souborech se často střídají manuálně a automaticky přepsané úseky.
To vypovídá o~nevoli uživatelů k~jinému než kompletnímu, lineárnímu přepisu.
% TODO: ref. aktivní učení
Každý shluk manuálně respektive automaticky přepsaných slov stačí tedy zabalit
do~jednoho elementu.

Zvýraznění jednotlivých slov -- aktuálně přehrávaného a vybraného klikem myší --
se realizuje pomocí umístění barevného rámečku pod~zvýrazněné slovo. To lze
provést díky tomu, že prohlížeče nyní umožňují zjistit polohu označeného textu a
označení lze provést automaticky.\footnote{Viz \texttt{getClientRects} a
\texttt{Range} ve~webových standardech.}

\subsection{Web Audio API}

Přechod na~tuto technologii umožnil některé pokročilé funkce, avšak za~relativně
vysokou cenu. Web Audio API je standard pro~pokročilé zpracování zvukového
signálu v~prohlížeči. Základním konceptem je graf procesních uzlů, které mají
vstup a výstup a mohou se libovolně propojovat. K~dispozici jsou zdroje zvuku
jako oscilátory nebo přehrávače streamů, souborů (tag \texttt{<audio>}) a dat
v~paměti (\texttt{AudioBuffer}) a efekty jako zesílení, dynamická komprese,
mixování kanálů.

Velká výhoda Web Audio API oproti elementu \texttt{<audio>} je možnost přesného
časování, až na~jednotlivé samply. Přehrávání výseku odpovídajícího označenému
textu se proto nemusí provádět pomocí velice nepřesného časovače
\texttt{setTimeout}.

Bez~Web Audio API by také nebylo možné provádět frekvenční korekci při poslechu,
čili mít tzv. \textit{ekvalizér}. Ten je zapotřebí, protože některé nahrávky
mají v~určitém frekvenčním pásmu silný šum, jehož odstranění je s~ekvalizérem
snadné a komfort poslechu se tak razantně zvýší.

Další funkcí, kterou Web Audio API umožňuje, je stahování úseků. Označením
přepsaného textu se definuje úsek nahrávky a ten je možné uložit bez~dalšího
síťového přenosu. Tato funkce však vyžaduje, aby nahrávka byla dekódovaná
v~paměti. Vzhledem k~tomu, že nahrávky mají běžně i hodinu a půl, trvá její
stažení a dekódování opravdu dlouho a navíc prohlížeč kvůli tomu spotřebuje přes
gigabyte operační paměti.

Jsou plány na~to, aby Web Audio API umožnila dekódovat jen část
nahrávky,\footnote{github.com/WebAudio/web-audio-api/issues/1305} proto tento
neutěšený stav zatím nechávám být. Případným řešením by mohlo být buď
streamování (\texttt{<audio>} jako zdrojový uzel) a stažení pouze při potřebě
ukládání úseku nebo změna uložení nahrávek nikoliv jeden soubor na jednu pásku,
nýbrž např. po~minutových úsecích.

Díky tomu, že Web Audio API umožňuje přehrávání binárních dat z~proměnné
v~paměti, nabízí se dekódovanou nahrávku uložit na~persistentní úložiště
uživatelova počítače a při opětovné návštěvě stránky data místo stahování odsud
nahrát.

Moderní prohlížeče poskytují několik bran k~úložišti na~místním disku.
Nejtradičnějšími jsou bezesporu \textit{cookies}, které jsou však pro ukládání
objemnějších dat zcela nepoužitelné. Velice slibnou se jeví
\textit{localStorage}, umožňující ukládání párů klíč-hodnota. I zde však
narážíme na~příliš omezující kvóty. Kupříkladu Firefox ji má na 10MB, přičemž
potřeba je asi 1GB. Dalším kandidátem je \textit{File System API}. Tento
standard pro~izolovaný souborový systém k~dispozici webové aplikaci je zcela
ideálním řešením -- dá se zde i explicitně požádat o~konkrétní diskovou kvótu a
uživatel tak má volbu bez nutnosti práce programátora webové aplikace. Kamenem
úrazu je zde však podpora, která se momentálně omezuje pouze na Google Chrome.

Naštěstí existuje ještě standard \textit{IndexedDB API}, který má uspokojivou
podporu a uložení gigabytu dat je s~ním možné, byť ne zaručené. S~využitím
abstrahující knihovny \textit{Dexie} je proto skrz tento standard ukládání
implementováno. Pro uživatele, kteří delší dobu pracují na jedné a téže nahrávce
se tím přináší velká úspora času a přenesených dat.

\section{Rozdělení nahrávek na úseky}

Vzhledem k~tomu, že ani v~roce 2019 není kurzorový přístup ke zvukovým datům
skrze Web Audio API v~dohlednu, a k tomu jak odrazující dopad má nutnost
stahovat a dekódovat celou nahrávku aspoň při jejím prvním načtení, nezbylo mi,
než změnit způsob, jakým jsou nahrávky uloženy. Že způsob uložení souborů hraje
roli, dokládá i Reppen (2010)\cite{reppen2010building}.

Nahrávky jsou uloženy v~několika instancích pro různé účely:

\begin{enumerate}
\item{na backendovém serveru ve formátu MFCC pro nucené zarovnávání,}
\item{v~repozitáři LINDAT ve formátu FLAC za účelem archivace a bádání,}
\item{na CDN ve formátu mp3 za účelem přímého stažení uživatelem,}
\item{taktéž na CDN ve formátech OGG/Vorbis a mp3 pro webové rozhraní.}
\end{enumerate}

Pouze poslední jmenovanou instanci je žádoucí ukládat tak, aby každý soubor byl
jen tak velký, aby jeho stažení a dekódování trvalo únosně dlouho. V~ostatních
případech je lépe zachovat uložení, kde jedna nahrávka odpovídající většinou
jedné straně kazety či jednomu průchodu pásky z~kotouč na kotouč. Třetí a čtvrtá
instance však navzdory rozdílnému účelu sdílejí tatáž data. Bylo proto nutné je
duplikovat.

\subsection{Délka segmentů}

Délka úseků, na které nahrávky rozděluji, ovlivňuje, jak dlouho se každý segment
bude stahovat a dekódovat. Čas stahování a dekódování segmentu, který obsahuje
slovo, na němž je kurzor při prvním požadavku o~přehrávání, je roven zpoždění od
uživatelské akce k začátku přehrávání. Podle internetového periodika
UXMovement\cite{foursecondrule}, začíná uživatel po čtyřech sekundách čekání
upouštět od předchozího záměru. Podle článku Nielsen Norman
Group\cite{websiteresponsetimes} je hranice únosnosti 10 sekund.

Pokud budou úseky příliš dlouhé, jejich stahování a dekódování zabere příliš
mnoho času. Na druhou stranu s každým předělem vnášíme do přehrávání bod, kde se
úseky nalepují a může tam vyvstat artefakt. Také s~každým segmentem se pojí
extra HTTP request s~nezanedbatelnou režií.

Jako vhodný kompromis se jeví segmenty o délce 30 - 120 sekund. Velikost
dvouminutového segmentu je v~komprimovaném formátu při mono/24KHz kolem 0.6MB a
na Intel Core2 o 2,5GHz se dekóduje asi 1.6 sekundy.

\subsection{Metody hledání bodů předělu}

Vhodným výběrem bodů předělu můžeme omezit dopad případných artefaktů
způsobených nepřesným navázáním. Ideálním by bylo dělit nahrávky v~momentech
ticha. Ne vždy jsou momenty ticha každé dvě minuty, proto z momentů ticha
ustupme k~požadavku pauzy v~řeči. Hovořit dvě minuty bez nádechu hraničí
s~nemožností. Potýkáme se tedy s~úlohou nalézt pauzy v~řeči. Jednak je třeba
ujasnit, podle jakého klíče budeme pauzy vybírat, a jednak, jak je budeme přesně
hledat.

Hledat pauzy v~řeči lze různými způsoby. Nejspolehlivější a nejnáročnější je
manuální označování pauz. Pokoušel jsem se o~to sám a dosáhl jsem rychlosti
přibližně čtyřnásobku rychlosti přehrávání, tedy jeden zapsaný bod předělu za
třicet sekund.

Další velice spolehlivou metodou je hledání podle predikovaných pseudofonémů
ticha v~zarovnaném přepisu. Tuto metodu jsem mohl namnoze použít, neboť
k~většině nahrávek mám automatický nebo i manuální přepis.

Tam, kde pořízení přepisu nebo jeho automatické zarovnání selhalo, lze použít
detekci ticha prostou akustickou analýzou. Tato metoda je velice náchylná
k~chybám v~případě nahrávek s~malým poměrem signálu k~šumu, kterých se v~korpusu
Karla Makoně vyskytuje neutěšeně mnoho.

Kde nepomůže ani metoda detekce ticha, což se pozná podle toho, že detekované
pauzy jsou příliš daleko od sebe nebo naopak zabírají valnou část nahrávky,
nezbývá, než určit body předělu ve fixních intervalech, nehledě na to, že jich
většina padne doprostřed slova.

Pokusy dvě metody vyloučily: Manuální hledání bylo příliš neefektivní. Kromě mne
se dalších asi pět dobrovolných anotátorů o~tento úkol pokusilo a došla jim
trpělivost po nule až deseti minutách označkovaného materiálu. Detekci pomocí
zarovnaného přepisu šlo použít i na některé nahrávky, u nichž přepis selhal tím
způsobem, že se rozdělily na menší části, přepsaly se ony -- zpravidla
s~katastrofální úspěšnosti. Tento přepis opět v~některých případech selhal, ale
většina takové nahrávky byla nějakým přepisem pokryta. A jakkoliv nekvalitní
takový přepis byl, právě dlouhé mezery mezi slovy se nalezly s~uspokojivou
přesností. Krátké úseky, na nichž selhalo rozpoznávání řeči, byly pak příliš
obtížné i pro detekci pomocí ticha. Jednalo se o úseky bez řečových událostí,
nebo s~extrémním šumem.


Celkový počet různě získaných bodů předělu shrnuje
tabulka~\ref{tab:splitpoints}.

\begin{table}[htpb]
\begin{center}
\begin{tabular}{|l|l|}
\hline
metoda získání & počet použitých \\
\hline
manuálně & 0 \\
podle zarovnaného přepisu & 60424 \\
podle detekce ticha & 0 \\
fixní délkou & 22043 \\
celkem & 82467 \\
\hline
\end{tabular}
\caption{počet bodů předělu podle metody jejich získání}\label{tab:splitpoints}
\end{center}
\end{table}

\subsection{Výběr bodů předělu}

V~naivní metodě určování bodů předělu pomocí fixního intervalu jsem zvolil délku
šedesáti sekund. Že to je délka přijatelná, se diskutuje výše a s~další
optimalizací tohoto parametru jsem neztrácel čas.

Zajímavější je situace u~hledání pomocí zarovnaného přepisu. Zde se jedná
o~programátorský úkol, kde na vstupu máme posloupnost slov vyskytujících se
v~přepisu nahrávky, z~nichž každé s~sebou krom své formy a výslovnosti nese
informaci, kde začíná, a pokud obsahuje na konci ticho, pak kde začíná
pseudofoném ticha a jak je dlouhý.  Redukovat tedy vstup můžeme na posloupnost
párů čísel, kde první vždy udává počátek ticha a druhé jeho konec. Na výstupu
očekáváme posloupnost časových pozic, které rozdělují nahrávku na úseky o~délce
nejméně 30 sekund, nejvýše 120 sekund, a které jsou uprostřed co nejdelších
segmentů ticha.

Povšimněme si, že úloha nemá řešení, pokud je nahrávka kratší třiceti sekund. To
ovšem jednak v~mluveném korpusu Karla Makoně nenastává a jednak by to nevadilo,
proože takovou nahrávku bychom jednoduše nechali v~jednom souboru.

Druhým případem, kdy úloha nemá řešení, je, když mezi dvěma sousedními pauzami
je rozestup větší než 120 sekund. Takový případ nastává, když je samotné
detekované ticho velmi dlouhé. Takové případy jsem řešil manuální úpravou.

Hledaný algoritmus se zdá být typickým příkladem pro dynamické programování:
Nalezneme ideální rozdělení nahrávky, která obsahuje jen první slovo, a poté
přidáváme slova a na základě dosavadního řešení a nového slova řešení
rozšiřujeme.

Je ale i jednodušší varianta: Začneme s~množinou všech pauz a iterujeme přes ně
od nejkratší po nejdelší. Pauzu z~množiny odebereme, pokud sloučením sousedních
segmentů nevznikne segment delší než 60 sekund. Přes vybrané pauzy znova
iterujeme a pauzu odebereme, jestliže jeden z~jejích sousedů má méně než 30
sekund.

\subsection{Pojmenování souborů}

Jsou-li vybrány body předělu, mohou se nahrávky podle nich rozdělit a výsledné
segmenty uložit na disk. Zde vyvstává otázka, jak rozdělit soubory do adresářů a
jak je pojmenovat. Zvolil jsem tento formát:

$\$ID/\$format/\$ID--from-\$ZACATEK--to-\$KONEC.\$PRIPONA$

tedy například

$88-04A/ogg/88-04A--from-1155.27--to-1211.53.ogg$.

Důvody jsou tyto: Není praktické z~důvodu omezení mnoha souborových systémů mít
příliš mnoho souborů v~jednom adresáři. Proto rozdělení do adresářů podle
identifikátorů nahrávek. Že formát je právě podadresář identifikátoru a ne třeba
nadadresář, je arbitrární, obojí by bylo možné, nebo i mít soubory všech formátů
v~jednom adresáři a jen je odlišovat příponou.

Zopakovat identifikátor nahrávky i v~názvu souboru jsem se rozhodl proto, aby
případný zatoulaný soubor mohl být snáze identifikován. Díky tomu, že se do
názvu souboru uvede začátek i konec úseku v~rámci nahrávky zajistí, že název
souboru přesně popisuje jeho obsah. Oproti tomu např. lineární číslování by při
změně bodů předělu vedlo k~tomu, že jeden název souboru by byl totožný pro různé
úseky v~různých verzích korpusu. To by mohlo vést k~problémům s~cachováním.
Identifikátory pomocí kontrolních součtů nebo jinak postavených na základě
binárního obsahu souboru by vedlo k~nutnosti změnit identifikátory při každé
změně komprese apod., ačkoliv slyšitelný rozdíl by třeba nebyl žádný.

Nyní, pokud by došlo ke změně bodů předělu, by se nové i staré úseky musely
uchovávat v~jednom adresáři a až by všechny instance starých úseků byly
vyhlazeny z~paměti cache všech klientů, by se staré mohly smazat. Jedinou
nevýhodou je, že by se musely nové přebrat od starých, ale to není problém.

\subsection{Překryv úseků}

Při testování se ukázalo, že vyřízneme-li pomocí programu \texttt{sox} úsek
zvukového souboru, výsledný soubor skončí přehrávání o~několik desetin sekundy
dříve. Jako by chybělo posledních několik set samplů. Příčinu tohoto fenoménu
zatím neznám. Kompenzoval jsem jej tím, že jsem každý úsek prodloužil o~půl
sekundy. Následkem toho bylo potřeba upravit přehrávání tak, aby každý úsek
skončil tehdy, až dohraje jeho metadaty daná délka, nikoliv až do skutečného
vyčerpání zvukových dat.


\section{Použití aplikace}

\fontspec{DoulosSIL} 

Aplikace znamená použití. Použitelnost je tedy klíčovým faktorem pro její hodnocení.
Přepis, který pořizuji, je fonetický i ortografický\footnote{Na pravopis jako
takový se důraz neklade. Ortografickým přepisem myslím standardní zápis.}.

\subsection{Expertíza uživatelů}

Přepis, který pořizuji, ke na hranici toho, co se dá nazvat lingvistickou
anotací dat. V~naší požehnané části světa, kde podíl analfabetů je zanedbatelný,
je přepis mluveného slova stěží odbornou prací. Na druhou stranu zajistit, aby
přepis přesně odpovídal mluvenému projevu
\begin{itemize}
\item{jakožto vyjádření vyřčených slov a jejich významu,}
\item{na fonetické úrovni foném na foném}
\item{a na časové ose}
\end{itemize}
je za hranicemi toho, co se dá očekávat od nevyškoleného uživatele.

Lingvistická anotace dat obecně vyžaduje zaškolené pracovníky. Podíváme-li se
např. na Pražský závislostní korpus, můžeme si povšimnout, že od anotátorů
vzešla taková úroveň expertízy, že se stali spoluautory\cite{hajivc2005complex}.

Crowdsourcing, přístup založený na komunitní spolupráci nebo zapojování
dobrovolníků nabývá na popularitě při získávání hodnot, které by jinak byly
neúnosně drahé. Zmiňme např. Mihalcea (2004)\cite{mihalcea2004building}, kde se
deleguje desambiguace významů slov na dobrovolníky pomocí webového rozhraní.
Wikicorpus\cite{reese2010wikicorpus}, jakož i MASC\footnote{Manually annotated
subcorpus}\cite{ide2010manually} od dobrovolníků sbírají anotace. Bojar
(2008)\cite{bojar2008czeng} pro CzEng také sbírá mnoho překladů od dobrovolníků.
Marge (2009)\cite{5494979} zkoumá možnosti využití technologie Mechanical Turk
pro získání přepisů mluveného slova.
Naproti tomu např. Maekawa (2000)\cite{maekawa2000spontaneous} popisuje tvorbu
mluveného korpusu spontánní japonštiny s~využitím placených anotátorů.

Ve většině případů je kvalita pro anotaci dat velmi důležitá, proto je aspoň
nějaká kontrola nezbytná, ať už je odbornost anotátorů jakkoliv vysoká. Je
zřejmé, že čím méně expertízy na straně anotátorů, tím silnější kontroly je
zapotřebí.

Běžnou metodou kontroly kvality je mezianotátorská shoda. To má obrovskou
nevýhodu, že každá část dat musí být anotována aspoň dvakrát, což snižuje
výtěžnost aspoň o 50\%.

Ještě jeden důvod hovoří proti jejímu použití v~případě tohoto projektu. Webová
aplikace je dělaná pro lidi, kteří chtějí poslouchat Makoňovy nahrávky
z~vlastního zájmu a jejich přínos pro kvalitu přepisu je spíše vedlejším
produktem. Nebylo by snadné přesvědčit je, aby si vybrali právě nahrávku, kterou
už někdo jiný přepsal.

Naštěstí lze implementovat automatický mechanismus, který uživatelům dopomůže
k~vyšší kvalitě příspěvků.

Webová aplikace vychází z~předpokladu, že a-priori existuje nějaký přepis ke
každé nahrávce, takže uživatelův příspěvek je vlastně korekcí. Každý příspěvek
má formu nahrazení textového segmentu jiným. Jelikož přepisy jsou zarovnány
s~audiem na časové ose, víme také, jakému přesně úseku nahrávky daný text
odpovídá.

Díky tomu mohu provést nucené zarovnání (forced alignment) textového úseku
s~audiem. V~případě selhání zarovnání můžeme předpokládat, že úsek byl přepsán
chybně, příspěvek odmítnout a dát tím uživateli zpětnou vazbu. Jelikož
jednotlivé úseky odpovídají akustickému modelu v~různé míře, dochází k~falešně
pozitivním i negativním vyhodnocením.

Falešně pozitivní případ (když systém přijme chybný přepis) představuje skutečný
problém, protože chyba vstoupí do trénovacích dat. Falešně negativní případy
mohou uživatelé často obejít tím, že správný, leč odmítnutý přepis, pošlou
znova, rozdělený do kratších částí. Touto metodou by se pochopitelně mohlo také
podařit vnutit systému nesprávný přepis. Nepředpokládám však na straně uživatelů
zlou vůli.

Krom zachycení chybného přepisu slouží nucené zarovnání k~přesné synchronizaci
na časové ose. Tento prvek zcela chybí prakticky ve všech programech pro přepis.
Ku příkladu Transcriber\footnote{http://trans.sourceforge.net/}, vyzrálý
svobodný přepisovací program pro Linux očekává časové zarovnávání na úrovni
frází ze strany uživatele. Transcribe\footnote{https://transcribe.wreally.com/},
komerční webový nástroj pro přepis nechává uživatele přidat časové značky
kamkoliv do textu. Není zde žádný akustický model, tedy nic, vůči čemu provádět
zarovnání.

\subsection{Pořízení fonetického přepisu}
\label{ssec:porizeni-fonetickeho-prepisu}

Fonetický přepis je v~běžném případě produktem nuceného zarovnání. Pokud je více
výslovnostních variant, automaticky se zvolí ta, která lépe odpovídá akustickému
modelu. Na to je potřeba pořídit výslovnostní varianty každého slova. Používám
kombinaci pravidlového převodníku inspirovaého Psutkou et
al.\cite{psutka2004development} a dynamického výslovnostního slovníku. Dynamický
výslovnostní slovník je seznam alternativních výslovností každého slova, který
se rozšiřuje s~používáním aplikace.

Manuál k~aplikaci vyzývá uživatele, aby text přepisovali podle standardního
českého pravopisu, ale při zachování maximální věrnosti vyřčených slov, tedy aby
nekorigovali /ɲaːk/ na {\em nějak}, nýbrž
přepsali doslova jako {\em ňák}. Fonetický slovník obsahuje časté výslovnostní
varianty, např. počáteční /v/ ve slovech začínajících na /o/.

V~případě slov s~nestandardní výslovností, tedy primárně cizích slov, se od
uživatelů žádá, aby slovo přepsali foneticky. Teprve po úspěšném zarovnání a
integrace do přepisu mohou slovu nastavit kýženou pravopisnou formu. Toto je
jeden z~mála případů, kdy se od uživatele chce něco nekonvenčního.

Když je pravopisně chybný, fonetický zápis poslán, pak pokud projde fází
nuceného zarovnání, se integruje do zobrazeného přepisu. Datová reprezentace
každého slova sestává z~těchto prvků:
\begin{enumerate}
\item{výskyt:
    slovo, jak se vyskytuje v textu, včetně zachování velkých a malých písmen a
    přilehlé interpunkce,
}
\item{slovní forma:
    slovo, jak je zaneseno v~jazykovém modelu a ve výslovnostním slovníku
    (slovní forma se odvozuje algoritmicky z~výskytu převedením do malých písmen
    a odstranění neabecedních znaků, z~toho plyne, že interpunkce a všechny
    neabecední znaky jsou vždy součástí přilehlého slova a nikdy netvoří token
    samy o sobě),
}
\item{výslovnost:
    seznam fonémů,
}
\item{časová značka:
    vzdálenost počátku slova od počátku nahrávky v~sekundách s~přesností na dvě
    desetinná místa,
}
\item{manuálně přepsané:
    pravdivostní hodnota odlišující manuálně přepsaná slova od automaticky
    přepsaných,
}
\item{confidence measure:
    míra jistoty, se kterou bylo slovo predikováno, pouze u automaticky
    přepsaných slov.
}
\end{enumerate}
Jakmile je slovo součástí přepisu, lze upravit jeho {\em výskyt}, tedy jak se
jeví v~textu. Nyní může uživatel vložit správnou ortografickou formu odchylující
se od českých pravidel výslovnosti.

To má za následek přidání dvojice {\em slovní forma - výslovnost} do dynamického
výslovnostního slovníku. Tento úkon je proto nutné provést jen jednou pro každé
slovo a pokaždé, když na něho libovolný uživatel narazí znova, stačí zadat jeho
ortografickou formu a správná výslovnost se dovodí automaticky.

\subsection{Fonetický zápis}

Přese všechny výhody reprezentace fonémů podle systému PACal se nejedná o
pratktický zápis výslovnosti pro laické Čechy. Díky jednoduchému, povětšinou
deterministickému mapování mezi fonémy a grafémy je fonetický zápis, nebo jak
se tento mechanismus označuje anglicky, {\em pronunciation respelling},
v~češtině něčím přirozeným a spolehlivým. Není ani potřeba explicitního dělení
slabik, jako tomu je u angličtiny
(Wikipedie\footnote{https://en.wikipedia.org/wiki/Pronunciation\_respelling}
udává příklad {\em ``Diarrhoea'' is pronounced DYE-uh-REE-a}).
Že tato technika je přirozenou pro všechny rodilé mluvčí češtiny se základním vzděláním,
postuluji jako fakt bez podpůrného výzkumu a zakládám to čistě na vlastní
zkušenosti.

Převod z~fonetického zápisu do PACal obstarává zmíněný převodník, viz
podsekci\ref{ssec:porizeni-fonetickeho-prepisu}. Je ale zapotřebí i opačného
směru, aby se uživateli mohla dát možnost zkontrolovat, zda slovo, které
přepsal, se uložilo se správnou výslovností.

Za tímto účelem jsem vytvořil javascriptový modul pro převod mezi seznamem
fonémů a českým fonetickým zápisem.

Algoritmus je jednoduchý. Ve většině případů jeden foném odpovídá jednoznačně
jednomu písmenu ve fonetickém přepisu. Výjimky jsou tyto:
\begin{enumerate}
\item{Foném \texttt{x} se píše {\em ch}.}
\item{Fonémy \texttt{dz, dzh} se píšou {\em dz, dž}.}
\item{Dvojhlásky \texttt{aw, ew, ow} se píšou {\em au, eu, ou}.}
\item{
    Sekvence \texttt{c h, o u, a u, e u, d z, d zh} se píšou
    {\em c'h, o'u, a'u, e'u, d'z, d'ž}.
    Budiž však poznamenáno, že sekvence \texttt{c h} je ryze hypotetická, ana
    porušuje spodobu znělosti.
}
\item{
    Neznělou zvýšenou alveolární vibrantu označuji {\em ř'}.
}
\item{
    Palatální a labiodentální nazála se píšou {\em n', m'}.
}
\item{Ticho na konci slova ve fonetickém přepise nevyznačuji.}
\end{enumerate}

Modul umožňuje obousměrný převod, ačkoliv v~aplikaci je zapotřebí jen směr ze seznamu fonémů do
fonetického zápisu určeného pro člověka. Uživatel sice může explicitně vyznačit
neznělé {\em eř} oproti znělému, či posloupnost hlásek {\em o}, {\em u} oproti
dvojhlásce {\em ou} pomocí apostrofu. Za sedm let provozu však tohoto nebylo ani
jednou potřeba.

Podotýkám, že ve výstupu převodníku do fonetického zápisu se nikdy nevyskytují
sekvence {\em di, ti, ni, dě, tě, ně}. Palatální souhlásky jsou vždy vyjádřeny
explicitně a např. sekvence {\em n i} se vždy vyjádří jako {\em ny}.

V tabulce~\ref{tab:priklady-fonetiky} uvádím několik příkadů slov, jejich výslovností a fonetickým zápisem, jak jej
produkuje algoritmus, pokud na vstup dostane příslušnou výslovnost ve formátu
PACal:

\begin{table}[htpb]
\begin{center}
\begin{tabular}{|l|l|l|l}
\hline
slovo & výslovnost v~IPA & výsl. v~PACal & fonetický zápis \\
\hline
nic & ɲit͡s  & nj i c & ňic \\
kdo & gdo & g d o & gdo \\
disk & disk & d i s k & dysk \\
dřít & dr̝ iːt & d rzh ii t & dřít \\
třít & tr̝̊ iːt & t rsh ii t & tř'ít \\
auto & aʊ̯to & aw t o & auto \\
nauka & nauka & n a u k a & na'uka \\
džbán & d͡ʒ baːn  & dzh b aa n & džbán \\
odžít & odʒiːt & o d dz ii t & od'žít \\
odznak & od͡z nak  & o dz n a k & odznak \\
podzemí & podzɛmiː & p o d z e m ii & pod'zemí \\
noc & not͡s  & n o c & noc \\
tento & tɛnto & t e n t o & tento \\
hangár & haŋgaːr & h a ng g aa r & han'gár \\
samba & samba & s a m b a & samba \\
tonfa & toɱfa & t o mg f a & tom'fa \\
\hline
\end{tabular}
\caption{Příklady algoritmicky získaného fonetického zápisu}\label{tab:priklady-fonetiky}
\end{center}
\end{table}

Použití apostrofu pro rozlišení víceznačností a zvláštností není stoprocentně
intuitivní a představuje další bod, kde je zapotřebí uživatele zaškolit, aby
tuto funkcionalitu dokázal patřičně využívat.

\subsection{Vyhodnocení kvality přepisů}

Jak stojí výše, webová aplikace krom jiného slouží pro získání kvalitního
zarovnaného přepisu od laických uživatelů. Pokusím se vyhodnotit, do jaké míry
se to podařilo.

Pro vyhodnocení kvality přepisů nemám žádná referenční data. Naopak, přepisy
používám jako gold standard, tedy referenční data pro trénování akustického ba i
jazykového modelu. Lze se však namátkou podívat na několik vzorků a získat tak
představu o tom, jak si systém vede.

Jednou z~věcí, které můžeme posoudit, jsou přijetí a odmítnutí příspěvků
zarovnávačem. Celkem z~109640 pokusů o zarovnání jich 3419 bylo odmítnuto, což
jsou 3.12\%. Manuálně jsem prošel 20 náhodně vybraných odmítnutých pokusů a
přišel jsem k~těmto číslům:
\begin{itemize}
\item{
    V~\textbf{11} případech se jednalo o falešné negativum, kde přepis byl
    správný a měl být přijat,
}
\item{
    ve \textbf{4} případech byly příčinou odmítnutí akustické nedostatky jako
    např. šum,
}
\item{
    ve \textbf{4} případech se jednalo o pravdivá negativa způsobená chybně
    zvolenými hranicemi segmentu a
}
\item{
     v \textbf{1} případě se jednalo o pravdivé negativum způsobené chybným
    přepisem.
}
\end{itemize}

Ve 25\% tohoto minimalistického vzorku tedy zarovnávač splnil svoje validační
poslání, předšed tomu, aby se do trénovacích dat dostal chybný vzorek. V~55\%
případů selhal a byl jen otravnou překážkou v~práci a ve zbývajících 20\%
případů sice odmítl validní přepis, ale zabránil tomu, aby se do trénovacích dat
dostal defektní vzorek, na což se dá dívat v~pozitivním světle.

Dá se také vyhodnotit scénář s~nestandardní výslovností. Za tím účelem jsem
z~dynamického výslovnostního slovníku vybral 4 nadějné záznamy a prohlédl si
příspěvky, které je obsahují. Tabulka~\ref{tab:eval-pronunc} uvádí pro každý
z~nich správnou ortografickou formu, chybnou výslovnost získanou převodníkem,
správnou výslovnost a konečně možný fonetický zápis. Ke každému údaji je uvedeno,
kolikrát se objevil v~manuálně přepsaných datech.

\begin{table*}[htpb]
\begin{center}
\begin{tabular}{|l r|l r|l r|l r|}
\hline
psaná forma                     & \#
    & chybná výslovnost             & \#
        & správná počeštělá výslovnost  & \#
            & fonetický zápis               & \# \\
\hline
Moody & 2 & moʔodi & 0 & muːdi & 4 & múdy, můdy & 2 \\
Descartes & 2 & dɛst͡s artɛs  & 0 & dɛkaːrt & 4 & dekárt & 2   \\
Weinfurter & 30 & vɛinfʊrtɛr & 13 & vajnfʊrtr & 19 & vajnfurtr & 2 \\
Michelangelo & 6 & mixɛlaŋgɛlo & 2 & mikɛlaŋd͡ʒ ɛlo  & 4 & mikelandželo & 0 \\
\hline
\end{tabular}
\caption{Příklady nestandardní výslovnosti v~manuálních přepisech}
\label{tab:eval-pronunc}
\end{center}
\end{table*}

\begin{table*}[htpb]
\begin{center}
\begin{tabular}{|l|r|r|}
\hline
 & foneticky správně & foneticky chybně \\
\hline
ortograficky správně & 25 & 15 \\
\hline
ortograficky chybně & 6 & 0 \\
\hline
\end{tabular}
\caption{Správnost fonetické a ortografické reprezentece cizích slov na základě
tabulky~\ref{tab:eval-pronunc}}
\label{tab:pronunc-rate}
\end{center}
\end{table*}

Z~tabulky~\ref{tab:pronunc-rate} je patrno, že ve většině případů je správně jak
po stránce fonetické, tak po stránce pravopisné. Pouze asi ve 13\% případů je
uchována pravopisně nesprávná forma. To připisuji tomu, že uživatelé, kteří jsou
si této problematiky vědomi, většinou celý proces dokončí a formu upraví.

Na druhou stranu téměř třetina případů vykazuje ponechání chybné fonetické
reprezentace. To představuje závažný problém alespoň z~dvou úhlů pohledu: Jednak
se tím dokazuje, že nucené zarovnání selhává při zachycení zcela odlišné
výslovnosti, a jednak se touto cestou dostávají do trénovacích dat špatné
vzorky.

Jednou z~patrných příčin je, že dynamický slovník rozpoznává pouze exaktně
shodná slova. V~jednom souboru je například vidět, jak všechny výskyty slova
{\em Weinfurter} mají výslovnost správně, zatímco ostatní formy, jako např. {\em
Weinfurterovi}, chybně.

Krom toho jistě budou hrát roli neinformovanost a roztržitost uživatelů, což se
jim dá mít těžko za zlé, vzhledem k~tomu, jak náročná činnost na soustředění se
od nich chce.

Naproti tomu případ nesprávné ortografické formy nepředstavuje tak závažný
problém. Může to sice ztížit vyhledávání, ale to se dá provést na výslovnosti, a
to už nyní manuálně, a lze to i automatizovat.

Čtvrtá kombinace fonetického zápisu a špatné výslovnosti se pochopitelně
nevyskytuje.

\normalfont 

\section{Backend}

Popsaná webová aplikace, která je uživatelským rozhraním, spoléhá na aplikační
rozhraní (API), odkud dostává aktuální přepisy, kam posílá příspěvky od
uživatelů, a na hosting, odkud stahuje zvukové soubory.

\subsection{API}

Backendová aplikace má formu HTTP serveru s~následujícími koncovými body.

\begin{itemize}
\item{
    Odeslání manuálního přepisu segmentu \\
    cesta: \texttt{/subsubmit} \\
    metoda: \texttt{POST} \\
    parametry:
    \begin{itemize}
        \item{\texttt{trans} (řetězec): přepis jak jej vložil uživatel,}
        \item{\texttt{filestem} (řetězec): identifikátor nahrávky,}
        \item{\texttt{start} (desetinné číslo): pozice začátku přepsaného segmentu od začátku nahrávky v~sekundách,}
        \item{\texttt{end} (desetinné číslo): pozice počátku přepsaného segmentu, ditto,}
    \end{itemize}
    odpověď při úspěchu: $\{
        success: 1,
        filestem: {\em řetězec},
        start: {\em desetinné číslo},
        end: {\em desetinné číslo},
        data {\em (seznam zarovnaných slov)}: [ 
            \{
                fonet: {\em řetězec},
                wordform: {\em řetězec},
                occurrence: {\em řetězec},
                humanic: 1, {\em (znamená, že je manuálně přepsané)}
                timestamp: {\em desetinné číslo}, {\em (pozice začátku slova)}
            \},
            {\em ...}
        ]
    \}$ \\
    odpověď při selhání: ${ message: {\em řetězec} }$
}
\item{
    Požadavek na seznam přepisů \\
    cesta: \texttt{/init} \\
    metoda: \texttt{GET} \\
    odpověď: \texttt{jsonp\_init(\{ subversions => \{ {\em identifikátor} => \{em verze}, {\em ...} \} \})} \\

    Verze se inkrementuje při každém příspěvku. Slouží k~tomu, aby se mohly cachovat
    transkripce, ale aby se cache nepoužila, pokud někdo přepis změnil.
}
\item{
    Inicializace sezení \\
    cesta: \texttt{/req} \\
    metoda: \texttt{POST} \\
    parametry:
    \begin{itemize}
        \item{\texttt{username} (řetězec),}
        \item{\texttt{session} (řetězec, nepovinný),}
    \end{itemize}
    odpověď: ${ status: "OK" }$ \\

    Slouží k~detekci začátku práce na přepisech pro účely sledování času
    potřebného k~přepisům.
}
\item{
    Požadavek statistiky, z~jaké části je která nahrávka přepsána \\
    cesta: \texttt{/humpart} \\
    metoda: \texttt{GET} \\
    odpověď: $\{
        {\em identifikátor}:
            human: {\em celé číslo - počet manuálně přepsaných slov}
            total: {\em celé číslo - celkový počet slov}
        {\em ...}
    \}$ \\

    Tento endpoint momentálně nová verze webového rozhraní nepoužívá, byl
    zamýšlen jako vodítko pro uživatele při výběru nahrávky pro přepis.
}
\item{
    Změna atributů zarovnaného slova v~přepisu \\
    cesta: \texttt{/saveword}
    metoda: \texttt{POST}
    parametry:
    \begin{itemize}
        \item{\texttt{wordform} (řetězec): slovo malými písmeny bez interpunkce}
        \item{\texttt{occurrence} (řetězec): slovo, jak se vyskytuje v~textu}
        \item{\texttt{fonet} (řetězec): fonémy oddělené mezerou, zavržený endpoint, používat \texttt{subsubmit}}
        \item{\texttt{timestamp} (desetinné číslo): pozice začátku slova od začátku nahrávky v~sekundách}
        \item{\texttt{stem} (řetězec): identifikátor nahrávky}
    \end{itemize}
    odpověď: ${ success: 1 }$

    Editace slova v~přepisu. Používá se, když slovu, které je foneticky správně
    přepsané a zarovnané, je potřeba změnit ortografickou formu, např. u cizích
    slov, doplnit interpunkci atp.
}
\end{itemize}

Veškerá komunikace je kódována v~UTF-8.

Koncový bod \texttt{subsubmit} pro přijetí (nebo odmítnutí) přepisu úseku
nahrávky, provede na straně serveru nucené zarovnání přijatého přepisu
s~odpovídajícím úsekem audia. Z~toho důvodu je nutné, aby na serveru byly
nainstalované nástroje \texttt{HVite} a \texttt{HCopy} z~HTK, dále aby tam byla
kompletní sada nahrávek ve formátu MFCC a akustický model. Bohužel se zdá, že
nucené zarovnání funguje v~HTK pouze s~monofonémovým modelem, takže přesnost
v~rozlišování přesných a chybných příspěvků není optimální.

\subsection{Ukládání dat}

API používá databázi PostgreSQL pro ukládání příspěvků, metadat k~nim a nepravidelných
výslovností. Ke každému příspěvku se ukládá
\begin{itemize}
\item{samotný přepis,}
\item{identifikátor nahrávky,}
\item{časové rozmezí odpovídajícího úseku nahrávky,}
\item{zda byl přepis přijat,}
\item{datum a čas přispění,}
\item{přezdívka autora, pokud ji vyplnil,}
\item{identifikátor sezení}
\item{a verze prohlížeče.}
\end{itemize}

Každé uživatelské sezení má taktéž svůj záznam a ukládá se proň totéž, co pro
příspěvek, krom příspěvku samotného.

Dále se v~databázi ukládají verze přepisů, které se inkrementují při každém
příspěvku.

Poslední věcí v~databázi je fonetický slovník. Ten slouží ke sběru fonetických
reprezentací slov s~nestandardní výslovností, jejichž výslovnost a psanou formu
poskytují uživatelé.

Transkripce se ukládají trojitě. Primárně na serveru v~souborech ve formátu JSONP.
Tyto se při každé změně zálohují na externí cloudové úložiště. Do třetice se
denně a na požádání exportují do HTML, které je přístupné z~CDN.
CDN slouží též k~servírování samotných nahrávek.

