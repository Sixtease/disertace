\chapter{Webové rozhraní}
\label{kap:webove-rozhrani}

Webové rozhraní, umožňující přístup zájemcům k~nahrávkám a jejich přepisu, byl
od začátku plánovanou součástí projektu. Rozhraní bylo navrženo s~těmito
požadovanými vlastnosmi:

\begin{itemize}
\item{výběr nahrávky ze~seznamu}
\item{poslech nahrávky s~obvyklými ovládacími prvky přehrávače}
\item{zobrazení přepisu nahrávky}
\item{vyznačení právě přehrávaného slova}
\item{možnost provést změnu v~přepisu}
\item{automatické zarovnání přepisu se~zvukem}
\item{případné odmítnutí přepisu, pokud zarovnání selže (přepis neodpovídá
vyřčeným slovům)}
\end{itemize}

První implementace byla založena na přehrávači
\em{jPlayer}, modulu pro knihovnu \em{jQuery}, který využívá standard HTML5
s~jeho elementem \em{<audio>} a technologii \em{Adobe Flash}. Pro~dynamickou
odezvu zobrazených prvků na~změny v~datovém modelu jsem použil knihovnu
\em{knockout}.

Aplikace měla formu jediné stránky s~rozbalovatelným výběrem nahrávky,
ovládacími prvky přehrávače a třemi řádky přepisu. Při označení části
zobrazeného přepisu se stránka překryla rozhraním pro opravu přepisu, jež zvu
\em{editačním okénkem}. V~editačním okénku se zobrazilo vstupní pole
(\em{<textarea>}) s předvyplněným současným přepisem, ovládací prvky pro
přehrátí odpovídající pasáže, odeslání opraveného přepisu a opuštění editačního
okénka.

Šlo o~prostou statickou HTML stránku s~JavaScriptem. K~audiu se přistupovalo
pomocí externí CDN, zatímco přepisy a API pro~zarovnávání oprav byly
na~zvláštním serveru.

Přepis byl uložen a přenášen ve~formátu \em{JSONp}\footnote{JSON = JavaScript
Object Notation, JSONp = JSON with Padding}, čili jako \em{JSON} obalený
v~javascriptové funkci pro zamezení problémů s~přístupem napříč doménami. Každé
slovo s~sebou neslo informaci o~svojí pozici v~nahrávce s~přesností na~setiny
sekundy, 

Během provozu první verze se projevila potřeba dalších funkcionalit. Uživatelé
obecně nechtěli, aby ostatní zasahovali do nahrávky, kterou si předsevzali
přepsat. 
