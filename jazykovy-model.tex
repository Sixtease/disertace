\chapter{Jazykový model}
\label{kap:jazykovy-model}

Jazykový model obecně je pravděpodobnostní rozdělení posloupností slov
v~jazyce.\cite{ponte1998language} V~kontextu rozpoznávání řeči je tandemovým
partnerem akustického modelu.\cite{jelinek1990self} Teprve kombinace akustického
a jazykového modelu určí výsledné slovo, které se na dané pozici rozpozná jako
nejpravděpodobnější.

Výběr jazykového modelu omezen nástrojem pro rozpoznávání. Lze zvolit pouze
takový model, který nástroj dokáže využít. Všechny nástroje, které jsem použil,
podporují N-gramové jazykové modely: HVite bigramový, Julius až trigramový a
DeepSpeech taktéž trigramový.

Pro trénování jazykového modelu mám k dispozici čtyři druhy dat:
\begin{enumerate}
\item{Obecné české texty,}
\item{Makoňovy spisy,}
\item{manuální přepisy nahrávek,}
\item{automatické přepisy.}
\end{enumerate}

Každá z~těchto kategorií skýtá různé množství textu a různou věrnost
modelovanému materiálu. Nejvěrnější jsou samozřejmě manuální přepisy Makoňových
nahrávek, také je jich nejméně: přesně 709 493 slov v~okamžiku psaní tohoto
textu. Automatické přepisy, jejichž přínos pro jazykové modelování je nejasný,
se jedná o 8 119 427 slov. Makoňovy spisy obsahují 3 328 720 slov. Obecné české
texty jsou nejdostupnější z~těchto komodit. Nejobsáhlejší dostupný korpus, který
jsem nalezl, je Mononews z WMT obsahující 1 019 497 060 slov.

Manuálním testováním jsem dospěl k~vahám $1 : 2000 : 7999 : 0$ pro korpusy ve
výše uvedeném pořadí. Aplikuji vyhlazování technikou
Knesser-Ney\cite{chen1999empirical}, jak je zabudována do nástroje pro tréning
jazykových modelů KenLM\cite{heafield2011kenlm}.
