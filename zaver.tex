\chapter{Závěr}
\label{kap:zaver}

Tématem disertační práce je ,,iterativní zdokonalování přepisu zvukových
nahrávek s využitím zpětné vazby posluchačů.`` Hlavním předmětem snažení
v~souladu s~tím bylo vytvoření systému, pomocí kterého se pro existující soubor
záznamů díky laické komunitě a komputačnělingvistickým metodám opatří co
nejdokonalejší přepis.

Práce zasahuje do několika odvětví. Jednak jde o představení nového korpusu,
Mluveného korpusu Karla Makoně, a tím spadá do korpusové lingvistiky. Jednak jde
o pořízení přepisu automatickými metodami a tím spadá do oblasti rozpoznávání
řeči. Jednak jde o vývoj nového typu aplikace, a tím spadá do oblasti
softwarového inženýrství. V~neposlední řadě jde o představení samotného obsahu
korpusu a tím spadá někam mezi teologii, religionistiku a možná filozofii.

Především jde z~mé strany o snahu o zachování a zpřístupnění odkazu Karla Makoně
a poukázání na jeho kvality coby českého klenotu světového duchovního poselství.

\section{Výsledky disertační práce}

Hlavními přínosy této disertační práce jsou

\begin{enumerate}
\item{pořízení kompletního přepisu mluveného korpusu Karla Makoně,}
\item{vyvinutí webového rozhraní, které umožňuje
    \begin{itemize}
    \item{synchronní konzumaci mluveného projevu a jeho přepisu,}
    \item{
        sběr oprav přepisu od laických uživatelů použitelný jako trénovací data
        pro strojové učení,
    }
    \end{itemize}
}
\item{umožnění fulltextového vyhledávání v~korpusu Karla Makoně,}
\item{částečné zmapování obsahu korpusu Karla Makoně,}
\item{
    pořízení tisícihodinového trénovacího korpusu pro automatický přepis
    češtiny,
}
\item{
    vyvinutí obecného systému automatického přepisu češtiny s~kompetitivní
    úspěšností.
}
\end{enumerate}

Korpus je k~dispozici v~repozitáři Lindat:\\
\texttt{https://lindat.mff.cuni.cz/repository/xmlui/handle/11372/LRT-1455}.\\
Webová aplikace sídlí na adrese \texttt{radio.makon.cz}.

Všechny zdrojové kódy jsou k~dispozici na GitHubu \texttt{github.com/Sixtease}.
Relevantní repoziráře:
\begin{itemize}
\item{\texttt{Evadevi} skripty pro rozpoznávání řeči nezávislé na datech,}
\item{\texttt{cz-parliament-speech-corpus}
    kompilace záznamů jednání parlamentu ČR pro trénování ASR,
}
\item{\texttt{Lingua-CS-Num2Words} rozpis čísel do číslovek (modul pro Perl),}
\item{\texttt{MakonASR} automatický přepis Makoňova korpusu pomocí Evadevi / HTK,}
\item{\texttt{DsMakonASR} automatický přepis Makoňova korpusu pomocí DeepSpeech,}
\item{\texttt{MakonFM} backendová aplikace a prototyp front-endu,}
\item{\texttt{MakonReact} front-endová aplikace,}
\item{\texttt{CorpusMakoni} nástroje pro obsluhu dat a index k~záznamům,}
\item{\texttt{Disertace} tato disertační práce.}
\end{itemize}

\section{Budoucí práce}

Budoucí práce vidím nyní více, než jsem jí viděl na začátku projektu.

Krom v~textu práce zmíněných bodů:
\begin{itemize}
\item{nahradit nucené zarovnávání pomocí HTK a monofonémových modelů
dokonalejším systémem,}
\item{automatizovat indexaci manuálních oprav do vyhledávače,}
\item{integrovat vysvětlivky, aby se dále redukovala potřeba zaškolení,}
\item{umožnit editaci bez nutnosti předchozího označování,}
\end{itemize}
se chci zaměřit na dvě oblasti:

Za prvé bych rád zařadil integraci tematických anotací do
webové aplikace. Jednak je stávajících dat tohoto druhu k~dispozici již takové
množství, že může být k~užitku lidem, kteří hledají pasáže týkající se toho či
onoho tématu, a jednak je vhodné využívat zapojení uživatelů i jiným způsobem
než pro sběr manuálních přepisů, tím spíše, že s~rostoucím množstvím jejich
přínos pro akustické a jazykové modelování klesá.

Za druhé bych rád pokračoval v~akustickém čištění poškozených záznamů, aby bylo
nahrávkám lépe rozumět, aby vzrostla přesnost jejich přepisu a aby se daly
pohodlně poslouchat třeba v~autě.
%\cite{zur2009noise}\cite{VARGA1993247}

Konečně pokud by se našla jiná sada nahrávek s~jinou komunitou, která by
vyvinutou technologii využila, rád bych ji nasadil, aby se užitek rozšířil i
mimo příznivce Karla Makoně.

\section{Poděkování}

{\em (bez zvláštního pořadí)}\\
Děkuji svému školiteli doc. Vladislavu Kuboňovi za vřelou záštitu a svobodu při
práci.\\
Děkuji dr. Nino Peterkovi za průpravu v~akustickém modelování.\\
Děkuji dr. Davidu Klusáčkovi za mnoho rad a pomoc při akustických úpravách.\\
Děkuji dr. Ondřeji Bojarovi za nápady a konzultace.\\
Děkuji doc. Markétě Lopatkové za podporu k~cestám na konference.\\
Děkuji vážené paní Libušce Brdičkové, která mě nesčetněkrát zachránila a bez níž
bych byl ztracen.\\
Děkuji doc. Zdeňkovi Žabokrtskému za upřímnou zpětnou vazbu a významnou pomoc.\\
Děkuji Petru Kazdovi z~Konicy Minolty, že mi umožnil skloubit práci na disertaci
s~prací pro chléb vezdejší.\\
Děkuji Alence Valentové, že mi představila dílo Karla Makoně.\\
Děkuji MUDr. Vítu Elgrovi za desítky let pořizování nahrávek, péči o ně a jejich
poskytnutí.\\
Děkuji ing. Milanu Tulachovi za spolupráci, testování a používání aplikace.\\
Děkuji všem příznivcům Karla Makoně za spolupráci a zpětnou vazbu.\\
Děkuji Karlu Makoňovi za moudrost, o kterou se s~námi dělí.
