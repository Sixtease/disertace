%%% Hlavní soubor. Zde se definují základní parametry a odkazuje se na ostatní části. %%%

%% Verze pro jednostranný tisk:
% Okraje: levý 40mm, pravý 25mm, horní a dolní 25mm
% (ale pozor, LaTeX si sám přidává 1in)
\documentclass[12pt,a4paper]{report}
\setlength\textwidth{145mm}
\setlength\textheight{247mm}
\setlength\oddsidemargin{15mm}
\setlength\evensidemargin{15mm}
\setlength\topmargin{0mm}
\setlength\headsep{0mm}
\setlength\headheight{0mm}
% \openright zařídí, aby následující text začínal na pravé straně knihy

%% Pokud tiskneme oboustranně:
% \documentclass[12pt,a4paper,twoside,openright]{report}
% \setlength\textwidth{145mm}
% \setlength\textheight{247mm}
% \setlength\oddsidemargin{14.2mm}
% \setlength\evensidemargin{0mm}
% \setlength\topmargin{0mm}
% \setlength\headsep{0mm}
% \setlength\headheight{0mm}
% \let\openright=\cleardoublepage

%% Vytváříme PDF/A-2u
\usepackage[a-2u]{pdfx}

%% Přepneme na českou sazbu a fonty Latin Modern
\usepackage[czech]{babel}
\usepackage{lmodern}
\usepackage[T1]{fontenc}
\usepackage{textcomp}

%% Použité kódování znaků: obvykle latin2, cp1250 nebo utf8:
\usepackage[utf8]{inputenc}

%%% Další užitečné balíčky (jsou součástí běžných distribucí LaTeXu)
% Abstrakt (doporučený rozsah cca 80-200 slov; nejedná se o zadání práce)


%% Balíček hyperref, kterým jdou vyrábět klikací odkazy v PDF,
%% ale hlavně ho používáme k uložení metadat do PDF (včetně obsahu).
%% Většinu nastavítek přednastaví balíček pdfx.
\hypersetup{unicode}
\hypersetup{breaklinks=true}

%% Definice různých užitečných maker (viz popis uvnitř souboru)

\begin{document}

\chapter*{Iterative Improving of Transcribed Speech Recordings Exploiting Listeners' Feedback}

\section*{Abstract}

\thispagestyle{empty}

This Ph.D. thesis deals with making a corpus of audio
recordings of a single speaker accessible to wide public and interested community.

The work has been motivated by the existence of a set of perishing recordings
of the Czech philosopher Karel Makoň on magnetophone tapes. The aim is to conserve
the material for future generations and to make it accessible using
digital technologies, in particular publishing the recordings online
and enabling the users to search through them.

The thesis introduces the creation of a system for transcribing a large set of
speech recordings employing a lay community. The solution designed is based on
obtaining a baseline low-quality transcription by means of automated speech
recognition and developing an application that allows for collecting corrections
of the automatic transcription in a fashion that makes it usable as training
data for further improvement of said transcription.

The spoken corpus itself is described. The
author and his works, topics covered in the talks, the process of recording
and digitization as well as the gained transcription are introduced.
Next, the development of a system for automated transcription of
the corpus, from collecting data, to acoustic and language modeling, various experiments
undertaken and evaluation are presented. Then, the web application for gathering
manual transcript corrections is described.
Differences to other settings, design and implementation details, a way to
compensate high demand for transcription quality and low demand for worker
expertise, as well as an evaluation of the system's performance after nine
years of operation are covered.

\end{document}
