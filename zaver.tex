\chapter{Závěr}
\label{kap:zaver}

Tématem disertační práce je iterativní zdokonalování přepisu zvukových
nahrávek s využitím zpětné vazby posluchačů. Hlavním předmětem snažení
tedy bylo vytvoření systému, pomocí kterého se pro existující soubor
záznamů opatří co nejdokonalejší přepis pomocí komputačnělingvistických metod a
zapojení laické komunity.

Práce zasahuje do několika odvětví.
Za prvé jde o digitalizaci a uchování fondu nahrávek a tím spadá do
archivnictví.
Za druhé jde o představení nového korpusu a tím spadá do korpusové lingvistiky.
Za třetí jde o pořízení přepisu automatickými metodami a tím spadá do oblasti
rozpoznávání řeči.
Za čtvrté jde o vývoj nového typu aplikace a tím spadá do oblasti
softwarového inženýrství.
Krom toho se dotýká obsahu
mluveného korpusu Karla Makoně, čímž se tento projekt dotýká i některých
odvětví věd humanitních. Z~hlediska obsahu textů se jedná o filosofii, teologii
a religionistiku. Z~hlediska téměř zapomenutého odkazu ing. Karla Makoně jde i o
téma české historie.

\section{Výsledky disertační práce}
\label{sec:zaver:vysledky}

Hlavními přínosy této disertační práce jsou

\begin{enumerate}
\item{pořízení kompletního přepisu mluveného korpusu,}
\item{vyvinutí webového rozhraní, které umožňuje
    \begin{itemize}
    \item{synchronní konzumaci mluveného projevu a jeho přepisu,}
    \item{
        sběr oprav přepisu od laických uživatelů použitelný jako trénovací data
        pro strojové učení,
    }
    \end{itemize}
}
\item{umožnění fulltextového vyhledávání v~korpusu,}
\item{částečné zmapování obsahu korpusu,}
\item{
    pořízení tisícihodinového trénovacího korpusu pro automatický přepis
    češtiny,
}
\item{
    vyvinutí obecného systému automatického přepisu češtiny s~kompetitivní
    úspěšností,
}
\item{
    negativní výsledky několika experimentů, najmě
    \begin{itemize}
    \item{kepstrální normalizace na izolovaných řečových událostech,}
    \item{použití standardní CycleGAN pro snížení chybovosti rozpoznávání řeči,}
    \item{
        získávání specifických trénovacích dat pro aktivní učení od
        dobrovolných anotátorů.
    }
    \end{itemize}
}
\end{enumerate}

Korpus je k~dispozici v~repozitáři Lindat:\\
\texttt{https://lindat.mff.cuni.cz/repository/xmlui/handle/11372/LRT-1455}.\\
Webová aplikace sídlí na adrese \texttt{http://radio.makon.cz}.

Všechny zdrojové kódy jsou k~dispozici na GitHubu \texttt{github.com/Sixtease}.
Relevantní repoziráře:
\begin{itemize}
\item{\texttt{Evadevi} skripty pro rozpoznávání řeči nezávislé na datech,}
\item{\texttt{cz-parliament-speech-corpus}
    kompilace záznamů jednání parlamentu ČR pro trénování ASR,
}
\item{\texttt{Lingua-CS-Num2Words} rozpis čísel do číslovek (modul pro Perl),}
\item{\texttt{MakonASR} automatický přepis Makoňova korpusu pomocí Evadevi / HTK,}
\item{\texttt{DsMakonASR} automatický přepis Makoňova korpusu pomocí DeepSpeech,}
\item{\texttt{MakonFM} backendová aplikace a prototyp front-endu,}
\item{\texttt{MakonReact} front-endová aplikace,}
\item{\texttt{CorpusMakoni} nástroje pro obsluhu dat a index k~záznamům,}
\item{\texttt{Disertace} tato disertační práce.}
\end{itemize}

\section{Budoucí práce}

Během práce na tomto projektu se postupně odhalila další možná témata, která určitě stojí za povšimnutí.
Při řešením úkolů, které si tato práce kladla za cíl, se ukázaly nové možnosti
vylepšení celého přístupu. Budoucí témata jsem již zmínil již výše, tak jak mě
napadala během práce. Zde jsou shrnuty v bodech:

\begin{itemize}
\item{nahradit nucené zarovnávání pomocí HTK a monofonémových modelů
dokonalejším systémem,}
\item{automatizovat indexaci manuálních oprav do vyhledávače,}
\item{integrovat vysvětlivky, aby se dále redukovala potřeba zaškolení,}
\item{umožnit editaci bez nutnosti předchozího označování.}
\end{itemize}

Kromě toho bych rád zařadil integraci tematických anotací do webové aplikace.
Stávající data tohoto druhu jsou k~dispozici ve velkém množství, proto mohou být
k~užitku lidem, kteří hledají pasáže týkající se konkrétního tématu. Zároveň je
možné využít zapojení uživatelů i jiným způsobem než pro sběr manuálních
přepisů, a to tím spíše, že s~rostoucím množstvím jejich přínos pro akustické a
jazykové modelování klesá.


Pochopitelně bych také rád pokračoval v~akustickém čištění poškozených záznamů,
aby vzrostla přesnost jejich přepisu i srozumitelnost lidskému uchu.

Jistě bych uvítal, kdyby se mnou vyvinutá technologie mohla použít i na jiné
sady nahrávek. Jsem přesvědčen, že by se přepis mluveného korpusu a jeho další
aplikace daly využít například v historii či jiných humanitních vědách. Jakékoli
velké soubory audio nahrávek s~komunitou příznivců by tak mohly být přepsány a
dále podrobněji zpracovávány metodami, které jsem v této práci rozvinul.

%\cite{zur2009noise}\cite{VARGA1993247}

\section{Poděkování}

{\em (bez zvláštního pořadí)}\\

Děkuji
svému školiteli doc. RNDr. Vladislavu Kuboňovi, Ph.D. za vřelou záštitu a svobodu při práci,
doc. RNDr. Markétě Lopatkové, Ph.D. za pomoc a podporu při účasti na konferencích,
Mgr. Nino Peterkovi, Ph.D. za průpravu v~akustickém modelování, 
Mgr. Davidu Klusáčkovi, Ph.D. za mnoho rad a pomoc při akustických úpravách,
doc. RNDr. Ondřeji Bojarovi, Ph.D.
a RNDr. Zdeňku Morávkovi, Ph.D. za nápady a konzultace,
doc. Ing. Zdeňkovi Žabokrtskému, Ph.D. za významnou pomoc i upřímnou zpětnou vazbu
a vážené paní Libušce Brdičkové za její neúnavnou ochotu a vstřícnost.

Děkuji
váženému MUDr. Vítu Elgrovi,
Ing. Milanu Tulachovi,
Mgr. Lence Vinklerové,
Alence Valentové,
a dalším za zprostředkování díla Karla Makoně, za zapůjčení nahrávek,
spolupráci, testování a používání aplikace i za zpětnou vazbu.

Děkuji
Mgr. Petru Kazdovi z~Konicy Minolty, že mi umožnil skloubit zaměstnání s~prací na disertaci

Děkuji, že mi bylo umožněno se tomuto projektu věnovat.
