\chapter{Závěr}
\label{kap:zaver}

\section{Bilance}

Nabízí se otázka, zda cíl disertační práce byl splněn. Odpověď je jednoznačně
negativní, a to ze dvou důvodů:
\begin{enumerate}
\item{V~případě akusticky obtížných nahrávek je přepis
zatím natolik nedokonalý, že neumožňuje uspokojivé vyhledávání, takže tyto
nahrávky nadále zůstávají využitelné jen pro lineární nebo namátkový poslech.}
\item{Webové rozhraní, zpřístupňující nahrávky veřejnosti, navzdory své
nepopiratelné a osvědčené funkčnosti, vyžaduje pro efektivní využití aspoň
zběžné zaškolení a tím se vylučuje široká veřejnost z~oboru lidí, kterým může
sloužit.}
\end{enumerate}

Kusých cílů však bylo dosaženo se stejnou jednoznačností: Přepis je
v~použitelné kvalitě pro velkou část korpusu, stávající webová aplikace je
solidním základem pro zlepšení v~oblasti uživatelského prožitku (UX) a hlavně, pro
vážné zájemce jsou nahrávky k~dispozici.
Za největší přínos práce nevztahující se konkrétně ke korpusu Karla Makoně
považuji vyvinutí a nasazení systému pro komunitní korekce přepisů.

Korpus je k~dispozici v~repozitáři Lindat:\\
\texttt{https://lindat.mff.cuni.cz/repository/xmlui/handle/11372/LRT-1455}.\\
Webová aplikace sídlí na adrese \texttt{radio.makon.cz}.

Všechny zdrojové kódy jsou k~dispozici na GitHubu \texttt{github.com/Sixtease}.
Relevantní repoziráře:
\begin{itemize}
\item{\texttt{Evadevi} skripty pro rozpoznávání řeči nezávislé na datech,}
\item{\texttt{MakonASR} automatický přepis Makoňova korpusu pomocí Evadevi / HTK,}
\item{\texttt{DsMakonASR} automatický přepis Makoňova korpusu pomocí DeepSpeech,}
\item{\texttt{MakonFM} backendová aplikace a prototyp front-endu,}
\item{\texttt{MakonReact} front-endová aplikace,}
\item{\texttt{CorpusMakoni} nástroje pro obsluhu dat a index k~záznamům,}
\item{\texttt{Disertace} tato disertační práce.}
\end{itemize}

\section{Budoucí práce}

Budoucí práce vidím nyní více, než jsem jí viděl na začátku projektu:

Prvním a předním bodem je zmíněná optimalizace webového rozhraní s~ohledem na
design a UX. Za tímto cílem je potřeba jednak konzultovat s~odborníky v~oboru,
neboť jakkoliv jsem zkušený programátor, designéra to ze mne zatím neudělalo, jednak provádět průběžně testování
s~uživateli, jak to radí např. Jan Řezáč ve své knize Web ostrý jako
břitva\cite{rezac2016web}.

Druhým bodem v~pořadí důležitosti, jak se mi nyní jeví, je důslednější aplikace
moderních neuronových sítí na rozpoznávač řeči. Současný automatický přepis je
sice získán pomocí DeepSpeech, což je systém postavený na TensorFlow, ovšem byl
%aplikován s~trigramovým jazykovým modelem a byl 
natrénován pouze na manuálních
přepisech Karla Makoně, což možná byla nejlepší varianta pro GMM, ale u
hlubokých neuronových sítí to tak vůbec být nemusí.

Jako třetí bod budoucí práce bych zařadil integraci tematických anotací do
webové aplikace. Jednak je stávajících dat tohoto druhu k~dispozici již takové
množství, že může být k~užitku lidem, kteří hledají pasáže týkající se toho či
onoho tématu, a jednak je vhodné využívat zapojení uživatelů i jiným způsobem
než pro sběr manuálních přepisů, tím spíše, že s~rostoucím množstvím jejich
přínos pro akustické a jazykové modelování klesá.

Konečně olbřímím úkolem bude pokus o kompenzaci akustických nedostatků
v~korpusu. Jednak pro kvalitu poslechu by bylo dobré vyčistit nahrávky od šumu,
aplikovat vícepásmovou dynamickou kompresi a kompenzaci zrychlení krom jiných
úprav, které se mohou ukázat jako vhodné. Jednak pro úspěšnost automatického
přepisu by bylo záhodno pokusit se o sdružení akusticky podobných částí korpusu
a natrénovat pro každou kategorii zvláštní akustický model, například pomocí
přidaného šumu.\cite{zur2009noise}\cite{VARGA1993247}

\section{Shrnutí}

Tato disertační práce se zabývá problematikou přepisu a zpřístupnění mluveného
korpusu se~zapojením komunity. Vychází z~konkrétní množiny nahrávek českého mystika Karla Makoně.
V~jejím rámci byly vytvořeny 1. systém pro automatický přepis aplikovaný na zmíněném
korpusu a v~komerční aplikaci, 2. webová aplikace pro přehrávání audia se
synchronním zobrazením přepisu a sběr oprav přepisu od uživatelů a 3.
vyhledávač nad danými nahrávkami. Pomocí systému bylo shromážděno toho času
přibližně 89 hodin manuálních přepisů. Automatický přepis dosahuje word error
rate 46,32\%. Technologie se kromě tohoto projektu využívá v~komerčním sektoru.

\section{Poděkování}

{\em (bez zvláštního pořadí)}

Děkuji svému školiteli doc. Vladislavu Kuboňovi za vřelou záštitu a svobodu při
práci.

Děkuji dr. Nino Peterkovi za průpravu v~akustickém modelování.

Děkuji dr. Davidu Klusáčkovi a dr. Ondřeji Bojarovi. za nápady a konzultace.

Děkuji paní docentce Markétě Lopatkové za podporu cestovat na konference.

Děkuji vážené paní Libušce Brdičkové, která mě nesčetněkrát zachránila a bez níž
bych byl ztracen.

Děkuji doc. Zdeňkovi Žabokrtskému za upřímnou zpětnou vazbu a významnou pomoc.

Děkuji Petru Kazdovi z~Konicy Minolty, že mi umožnil zužitkovat moje
technologie v~komerčním sektoru.

Děkuji Alence Valentové, že mi představila dílo Karla Makoně.

Děkuji MUDr. Vítu Elgrovi za desítky let pořizování nahrávek, péči o ně a jejich
poskytnutí.

Děkuji ing. Milanu Tulachovi za spolupráci, testování a používání aplikace.

Děkuji všem příznivcům Karla Makoně za spolupráci a zpětnou vazbu.

Děkuji Karlu Makoňovi za moudrost, o kterou se s~námi dělí.
