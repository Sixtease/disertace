\noindent Název práce: Zpřístupnění korpusu zvukových nahrávek jednoho mluvčího\\
Autor: Mgr. Oldřich Krůza\\
Katedra (ústav): Ústav formální a aplikované lingvistiky\\
Vedoucí diplomové práce: Doc. Vladislav Kuboň\\
E-mail vedoucího: Vladislav.Kubon@mff.cuni.cz\\

\noindent Abstrakt:  Tato disertační práce se zabývá zpřístupněním zvukových
záznamů jednoho mluvčího úzké i široké veřejnosti.

Motivací práce byla existence chátrajících nahrávek hovorů českého mystika
Karla Makoně na kazetách a kotoučích. Cílem je zachování materiálu pro
budoucí generace a zpřístupnění nahrávek pomocí digitálních technologií.
Především přístpnost na nahrávek na internetu a možnost vyhledávání v nich.

Největší důraz je kladen na krok převodu nahrávek do textu. K tomu je použito
automatické rozpoznávání řeči pomocí vlastního akustického i jazykového modelu
a manuální přepisy získané od členů komunity kolem Makoňova odkazu přes zvlášť
navržené webové rozhraní.

\vspace{10mm}

\noindent
Title: Making accessible a corpus of audio recordings of a single speaker\\
Author: Mgr. Oldřich Krůza\\
Department: Institute of Formal and Applied Linguistics\\
Supervisor: Doc. Vladislav Kuboň\\
Supervisor's e-mail address: Vladislav.Kubon@mff.cuni.cz\\

\noindent Abstract:  This Ph.D. thesis deals with making a corpus of audio
recordings accessible to wide public and interested community.

The work has been motivated by the existence of a set of perishing recordings
of the Czech mystic Karel Makoň on magnetophone tapes. The aim is to conserve
the material for the future generations and making them accessible using
digital technologies. This boils down mainly to publishing the recordings online
and enabling the users to search through them.

The main point of focus is the conversion of the recordings to textual form.
Automated speech recognition with a custom acoustic and language model, as well
as manual transcriptions from the members of the community around Makoň's legacy
through a specially crafted web interface have been employed.
