\chapter{Aplikace technologie na jiná data}
\label{kap:jina-data}

% TODO: diagram výběru gramatik

Bylo mojí představou, že jakmile systém pro komunitní opravy přepisů vytvořím,
bude o jeho aplikaci na ostatní sbírky dat zájem z~mnoha stran. Makoň přece není
jediným duchovním učitelem, kterého jeho žáci nahrávali. Faktem však je, že se mi
nepodařilo najít žádnou jinou komunitu, kde by tento scénář fungoval a kde by
byl o společné přepisy zájem.

Jsem přesvědčen o tom, že v~budoucnu, až celý systém ještě vyzraje, bude moci
sloužit u nahrávek jiných autorit, než je Karel Makoň. Mezi tím se však objevila jiná
forma využití.

\section{Rozpoznávání řeči}

Firma Konica Minolta vyvíjí produkt s~názvem AIR\textsuperscript{e} Lens. Jde o platformu
chytrých brýlí určených pro průmysl. Po hardwarové stránce se výrobek vyznačuje
tím, že brýle samotné jsou velmi lehké, jen 35g, a výpočetní jednotka je k~nim
připojena kabelem. Ta sestává z~počítače s~jedním gigabytem paměti a procesorem
architektury ARM. Kvůli tomu, že již výrobcem není podporovaná, nelze na ni
nainstalovat novější operační systém než Ubuntu 12.04.

Kvůli zmíněným limitacím dlouho nebylo možné zařízení opatřit hlasovým
ovládáním. První systém rozpoznávání řeči, který na brýlích AIR\textsuperscript{e} Lens
funguje, je právě ten, který vznikl v~rámci této disertační práce.

Systém byl adaptován pro angličtinu. Použil jsem výslovnostní slovník a
prediktor výslovnosti pro neznámá slova projektu CMU Sphinx\cite{huggins2006pocketsphinx}\cite{lamere2003cmu}.
Pro natrénování akustického modelu jsem použil svobodný korpus CommonVoice od
organizace Mozilla.

Hlasové ovládání funguje na principu rozpoznávání některého z~předem známé
množiny výrazů, popřípadě jednoduché gramatiky. Robustní jazykový model by
zařízení neúměrně zatížil, takže rozpoznávání řeči s~velkým slovníkem nepřipadá
v~úvahu. Proto jsem vyvinul mechanismus, pomocí něhož
rozpoznávání řeči má formu serveru a každá klientská aplikace rozhoduje o tom,
jaká gramatika se bude právě používat. Umožňuje to systém rozpoznávání řeči
Julius\cite{lee2001julius}\cite{lee2009recent}.

\section{Komunitní přepisy}

V~témže projektu se využívá i webová aplikace popsaná
v~kapitole~\ref{kap:webove-rozhrani}. AIR\textsuperscript{e} Lens poskytuje funkcionalitu
vzdálené podpory, kde pracovník s~brýlemi sdílí svůj pohled se vzdáleným
operátorem, který mu může kreslit do obrazu, a se kterým vede rozhovor. Tato
sezení vzdálené podpory představují pro zákaznické firmy cennou znalostní bázi,
neboť se v~nich ilustrativně a návodně řeší problémy jejich servisních techniků.

S~využitím zmíněné webové aplikace se proto vyvíjí systém, který umožní
zaznamenaná sezení vzdálené podpory zpětně přehrát i s~automaticky získaným
přepisem komunikace a případné chyby v~něm opravit.

Pro toto využití je potřeba aplikaci adaptovat na rozlišování mluvčích.
