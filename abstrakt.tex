\noindent Název práce: Iterativní zdokonalování přepisu zvukových nahrávek s využitím zpětné vazby posluchačů\\
Autor: Mgr. Jan Oldřich Krůza\\
Katedra (ústav): Ústav formální a aplikované lingvistiky\\
Vedoucí diplomové práce: Doc. RNDr. Vladislav Kuboň, Ph.D.\\
E-mail vedoucího: Vladislav.Kubon@mff.cuni.cz\\
Konzultant: Mgr. Nino Peterek, Ph.D.\\
E-mail konzultanta: Nino.Peterek@mff.cuni.cz\\

\noindent Abstrakt: Tato disertační práce se zabývá zpřístupněním zvukových
záznamů jednoho mluvčího úzké i široké veřejnosti.

Motivací práce byla existence chátrajících nahrávek hovorů českého filozofa
ing. Karla Makoně na kazetách a kotoučích. Cílem je zachování materiálu pro
budoucí generace a zpřístupnění nahrávek pomocí digitálních technologií,
především přístupnosti nahrávek na internetu a možnosti vyhledávání v~nich.

Práce představuje tvorbu systému pro přepis velké sady zvukových záznamů
se zapojením laické komunity. Navržené řešení spočívá ve vytvoření základního
přepisu nízké kvality pomocí automatického rozpoznávání řeči a vyvinutí
aplikace, která umožní od členů komunity i nahodilých zájemců získávat opravy
automatického přepisu, použitelné jako trénovací data pro další zlepšování.

Popíše se samotný mluvený korpus. Představí se autor a
jeho dílo,
témata v~nahrávkách, nahrávání samotné, digitalizace a získané přepisy.
Dále se rozvede tvorba systému pro automatický
přepis korpusu od akustického modelování, přes jazykové modelování, různé
provedené experimenty až k~vyhodnocení úspěšnosti. V~neposlední řadě se popíše
webová aplikace pro sběr manuálních přepisů. Zmíní se odlišnosti od ostatních mně známých
systémů, detaily návrhu a řešení, mechanismus pro kompenzaci vysokých nároků na kvalitu
přepisu a nízkých nároků na odbornost přispěvatelů a vyhodnocení funkčnosti
po osmi letech provozu.

% Výsledkem práce je především samotný mluvený korpus Karla Makoně,
% implementovaný systém pro komunitní přepis, 98 manuálně
% přepsaných hodin a automatické rozpoznání akusticky obtížného materiálu
% s~word error rate 38,48\%.

\vspace{10mm}

\noindent
Title: Iterative Improving of Transcribed Speech Recordings Exploiting Listener's Feedback\\
Author: Mgr. Jan Oldřich Krůza\\
Department: Institute of Formal and Applied Linguistics\\
Supervisor: Doc. RNDr. Vladislav Kuboň, Ph.D.\\
Supervisor's e-mail address: Vladislav.Kubon@mff.cuni.cz\\
Consultant: Mgr. Nino Peterek, Ph.D.\\
Consultant's e-mail address: Nino.Peterek@mff.cuni.cz\\

\noindent Abstract:  This Ph.D. thesis deals with making a corpus of audio
recordings of a single speaker accessible to wide public and interested community.

The work has been motivated by the existence of a set of perishing recordings
of the Czech philosopher Karel Makoň on magnetophone tapes. The aim is to conserve
the material for future generations and making it accessible using
digital technologies, in particular publishing the recordings online
and enabling the users to search through them.

The thesis introduces the creation of a system for transcribing a large set of
speech recordings employing a lay community. The solution designed is based on
obtaining a baseline low-quality transcription by means of automated speech
recognition and developing an application that allows for collecting corrections
of the automatic transcription in a fashion that makes it usable as training
data for further improvement of said transcription.

The spoken corpus itself is described. The
author and his works, topics covered in the talks, the process of recording
and digitization as well as the gained transcription are introduced.
Next, the development of a system for automated transcription of
the corpus, from acoustic modeling, to language modeling, various experiments
undertaken and evaluation are presented. Then, the web application for gathering
manual transcript corrections is described.
Differences to other settings, design and implementation details, a way to
compensate high demand for transcription quality and low demand for worker
expertise, as well as an evaluation of the system's performance after eight
years of operation are covered.

% The results comprise mainly the spoken corpus of Karel Makoň itself,
% the implemented system for community-driven transcription,
% 98 manually transcribed hours
% and automatically transcribed acoustically difficult material with word error
% rate of 38.48\%.
