\chapter{Data}
\label{kap:data}

Zvukový odkaz Karla Makoně je původní a ústřední motivací pro tuto práci. Osobně
považuji Makoňovo dílo za jedno z~nejzásadnějších vůbec v~oblasti duchovního
průkopnictví, a to jeho systematičností, obsáhlostí, návodností, novátorstvím a
především hloubkou. Jeho nauka se od moderních duchovních směrů odlišuje kladným
postojem k~civilizačnímim trendům, nikoliv jejich zavrhováním, dále
konzistentním souladem s~rozumovým poznáním a pevnými základy v~náboženských
tradicích. Od vědeckého bádání se odlišuje zejména tím, že rozum a hmotu
považuje za odrazové můstky k~hlubšímu poznání, nikoliv za vrchol a jedinou
platformu lidského poznání. Trvá však na tom, že duchovní zákonitosti jsou
stejně tak pevně dané, univerzální a ověřitelné (ovšem pouze osobní, subjektivní
zkušeností), jako zákony přírodní, popsané vědecky. Do třetice od klasické
křesťanské literatury se liší obzvláště tím, že Ježíšovu nauku považuje za návod
prvotřídní kvality pro vědomý vstup do věčného života zde na zemi a v~těle,
nikoliv po smrti. Věčným životem se míní stav, kdy člověk je vědomě věčnou
bytostí nezávislou na pomíjejícím těle. Tvrdě kritizuje překonaný a naivní
výklad, podle nějž se ctnostným životem dá dojít po smrti do nebe, dále dojít
spásy pouhou proklamací o~víře v~Krista a dodržováním přikázání a náboženských
obřadů.

Odkrývá smysl života a návod na jeho uskutečnění, který nestojí na slepé víře
ani na vlasní omezené lidské invenci.

\section{Karel Makoň}

\subsection{Duchovní vzestup Karla Makoně}

Ing. Karel Makoň se narodil 12. prosince 1912. Ve věku dvou let ho postihl zánět
levého ramene. Lékaři doporučovali amputaci ruky, k~čemuž jeho matka nedala
souhlas a na vlastní zodpovědnost nechala dítě operovat. Vzhledem k~tomu, že
ještě nebyly objeveny krevní skupiny, nebyla možná transfúze a proto musela být
operace prováděna opakovaně, aby dítě nevykrvácelo. Tehdejší anestetika nebylo
možné tak často podat tak mladému organizmu, proto byly operace prováděny při
vědomí. Malý Karel Makoň se zažívaje nesnesitelnou bolest naučil v~raném věku
opouštět svoje tělo. Tato opakovaná zkušenost měla u něho následek, že po
určitou dobu nepoznával svoji matku, zato začal spontánně rozpoznávat správné od
nesprávného a důsledně činit, co poznával jako správné.

Pro omezení rizika komplikací s~operovanou rukou bylo Karlovi zakázáno hrát si
s~dětmi. Svůj předškolní čas proto trávil sám na venkově jen se zvířaty. Díky
extrakorporálním zkušenostem se naučil rozumět řeči zvířat, obzvláště hus, které
ho vzaly za svého a s~nimiž pronikal do stavu zvířecího ráje.

Období ,,činění správného'', kdy si kupříkladu zapověděl kouření, alkohol i
veškerý pohlavní život, vyvrcholilo v~Makoňových sedmnácti letech, kdy si
přečetl myšlenku, že ,,tento život je mostem do věčnosti''. Tím započalo období
extází a vědomí, že je nesmrtelnou bytostí a smyslem jeho života je spojení
s~Bohem. Svojí matkou a prarodiči byl sice veden ke tradiční katolické víře, ale
nikdy na ni nepřistoupil, protože ,,v~nebi, kde by se jen díval na Boží tvář by
byla strašná nuda''. Nikdy tedy nevěřil, v~sedmnácti letech poznal.

V~tomto období se ustavičně modlil za to, aby dokázal Boha více milovat. Tato
modlitba trvala devět let a jejím vyvrcholením byla deportace do koncentračního
tábora v~Sachsenhausen v~roce 1939, coby českého vysokoškolského studenta.

V~koncentračním táboře byl Makoň sužován více, než ostatní: měl jakýsi
obzvláštní talent chytat rány a kopance. Prožíval nesmírné zmatení a frustraci
nad tím, že tak dlouho tak věrně sloužil Bohu, a teď se s~ním jedná jako s~kusem
hadru. Po čtyřech dnech utrpení nastal zlomový okamžik. Tamějším vězňům bylo
zakázáno pod trestem smrti přihlížet zabití spoluvězňů příslušníky SS. Makoň si
však nedal pozor a hleděl na právě takovou scénu. Vykonávající Němec si toho
povšiml a vyzval Karla Makoně, ať zůstane stát na místě, že hned, jak dobije
svoji momentální oběť, přijde zabít i jeho. Karel Makoň v~tu chvíli raději
odevzdal svůj život Bohu, a to bez přemýšlení a bezpodmínečně, se silou, nabytou
onou devítiletou modlitbou. Překvapivým výsledkem toho bylo, že SS-Mann tváří
v~tvář Makoňovi zbledl a v~hrůze se obrátil na útěk.

Makoň tehdy obdržel všeobjímající poznání smyslu života, jak svého, tak obecně,
a absolutní svobodu. Pohyboval se volně od baráku k~baráku, nezažíval hlad ani
jiný nedostatek, esesáci ho neviděli. Trávil svoje dny v~koncentračním táboře
burcováním ostatních k~probuzení k~pravdě, kterou sám zažíval.

Zanedlouho byl z~koncentračního tábora propuštěn a celý zbytek svého života
věnoval předávání svojí zkušenosti a hlavně návodu, jak k~ní přijít bez nutnosti
tak výrazných krizí, jakými sám v~životě procházel, a které považuje za
nepříkladné.

Zemřel v~roce 1993.

\subsection{Spisovatelská a přednášková činnost}

Karel Makoň ve svých duchovních dílech čerpá z~poznání, kterého nabyl
v~koncentračním táboře. Ještě předtím však napsal dopis {\em Utrpení a láska}
roku 1936 na podporu trpícímu příteli a roku 1939 přeložil spis {\em Bhakti
jóga}\cite{vivekananda2003bhakti} od Svámího Vivekanandy.

Do roku 1991, kdy v~činnosti ustal, napsal a přeložil dílo o celkovém rozsahu
3.613.211 slov nebo též 25.069.991 znaků. Nejrozsáhlejším jeho dílem je trilogie
{\em Mystika}, která sestává z~dílů
\begin{enumerate}
\item{{\em Západní starověká tradice} z~roku 1948,}
\item{
    {\em Srovnání jógy s~křesťanskou mystikou}
    z~let 1986 až 1989, obsahující překlad děl
    {\em Syntéza jógy}\cite{aurobindo1999synthesis} od Šrí Aurobinda Ghose
    a {\em Hrad nitra}\cite{teresa1588interior} od Sv. Terezie z~Avily
}
\item{a konečně {\em Výklad evangelia Sv. Jana} z~let 1950 až 1953.}
\end{enumerate}

Dalšími rozsáhlými knihami jsou {\em Cesta vědomí} (1973-1974),
výklad církevního roku a církví doporučených biblických čtení
{\em Postila} (1965),
{\em Sladké jho} (1977 - 1980) obsahující překlad části díla
{\em Précis de Théologie Ascétique et Mystique}
od Adolphe-Alfreda Tanquerey,
{\em Umění následovat Krista} (1971),
autobiografické {\em Umění žít} (1969)
a {\em Základní kurs nadživotnosti pro ty, kteří si myslí, že nevěří a základní
kurs náboženství pro ty, kteří si myslí, že věří, neboť jsou si všichni rovni,
pokud umírají, aniž by se během života znovu narodili} (1967-1968).

Nikoliv pro jejich rozsah, ale význam si dovolím zmínit spisy
{\em Oběť mše svaté} (1951), kde je účast na katolické liturgii podána jako
návod pro spojení s~věčností,
{\em Pohádka na dobrou noc} (1980), kde je odhalen duchovní smysl pohádky o
Honzovi
a {\em Blahoslavenství} (1973).

Makoňovo dílo se šířilo převážně samizdatem, za socialismu výhradně, ale i po
Sametové revoluci převážně. Psané dílo bylo několikrát kompletně přepsáno na
psacích strojích a po příchodu osobních počítačů ještě jednou. Všechny knihy a
spisy jsou volně k~dispozici na stránkách \texttt{makon.cz}. Jen hrstka knih
byla vydána, a sice
\begin{enumerate}
\item{Umění následovat Krista v~roce 1992\cite{makon1995umeni},}
\item{
    Pohádky nejen pro děti pod názvem {\em Odkrytá moudrost starých
    pravd}\cite{makon1992odkryta} v~roce 1992,
}
\item{Utrpení a láska v~roce 1995\cite{makon1995utrpeni},},
\item{
    Mystická koncentrace pod názvem
    {\em Mystická koncentrace a příprava k~ní}
    v~roce 1995\cite{makon1995mysticka},
}
\item{
    Otázky a odpovědi I - IV pod názvem {\em Světlo na cestu}
    v~roce 1999\cite{makon1999svetlo},
}
\item{
    Blahoslavenství v~roce
    2000\cite{makon2000blahoslavenstvi}\footnote{\label{note1}
        Duchovní úlohy a Blahoslavenství mají, zdá se, totožné ISBN.
    },
}
\item{
    Úlohy pod názvem {\em Duchovní úlohy} v~roce
    2002\cite{makon2002ulohy}\footnotemark[\ref{note1}]
}
\item{a Základní kurs nadživotnosti v~roce 2005\cite{makon2005zakladni}.}
\end{enumerate}

Většina děl se snaží podat více či méně ucelený návod pro vědomý vstup do
věčnosti, vždy z~jiného východiska nebo pro jiný typ čtenáře. Například {\em Umění
žít} je věnováno Makoňově nejmladší dceři a je z~velké části vlastním
životopisem. {\em Základní kurs nadživotnosti, pro ty, kteří si myslí, že nevěří, a
základní kurs náboženství pro ty, kteří si myslí, že věří} je kniha, která má
společný úvod a závěr, ale hlavní stať je rozdělena na dvě oddělené části, jedna
pro lidi nevěřící v~Boha, kde se autor opírá o experimenty s~tělem a o vědecký
přístup, zatímco v~druhé části se důkladně rozebírá smysl Otčenáše a radí se,
jak tuto modlitbu praktikovat pro její spojovací účel. {\em Umění následovat
Krista} se zase soustředí na systematizaci cesty v~rozdělení na očistnou,
osvěcovací a spojovací část, jak to vykládá křesťanská mystická tradice.

\subsection{Rysy Makoňovy nauky}

Nauka Karla Makoně je dle mého názoru jedinečná. Pokusím se krátce představit
její charakteristiky, jak je hodnotím podle svojí osobní zkušenosti a svého
názoru. Serióznější porovnání by bylo námětem na jinou disertaci na jiné
fakultě.

Karel Makoň je moderním učitelem duchovní moudrosti. Takových se v~poslední době
objevily mraky, jmenujme jen namátkou: Saibaba, Eckhart Tolle, Anthony DeMello,
S.N. Lazarev, Narópa, Ken Wilber, Osho, Nisargadatta, Elisabeth Haich a mnoho
dalších.

Jako mnozí vychází Karel Makoň z~křesťanství. Považuje Bibli za vrcholný zdroj
moudrosti a Ježíšův život a výroky za nejdokonalejší návod k~duchovní realizaci,
jaký je nám momentálně k~dispozici. Zdůrazňuje, že je vždy potřeba se řídit
Ježíšovým příkladem jako celkem, nikdy částí vytrženou z~kontextu. Každý Ježíšův
výrok a úkon má svůj protiklad. Např. hlásá Ježíš nenásilí jednak příkladem,
jednak radou nastavit druhou tvář tomu, kdo člověka udeří do jedné. Jindy však
bičem vyhání kupce z~chrámu. Jedině syntéza výroků a činů s~jeho protiklady
mohou podle Makoně poskytnout použitelný návod, kterým se dá v~životě obecně
řídit.

Striktně zavrhuje doslovný výklad tzv. nadpřirozených událostí. Například
apokalypsa -- konec světa a druhý příchod Ježíšův, je podle něho ryze individuální záležitostí, která se
stane každému člověku v~jiný okamžik podle jeho vývoje. Dokládá to Ježíšovým
výrokem, že ,,nepomine toto pokolení, než se to všecko stane,'' (Mk 13,30) dále
srovnání s~fenoménem mystické smrti, známým od mnoha jednotlivců i rozných
tradic a vlastní zkušeností. Stejně tak stvoření světa je podle něho popisem
vývoje lidského jedince, obzvlášť jeho nitra. K~tomu zdůrazňuje, že nejde o
jednorázově proběhnuvší čin, nýbrž soustavné tvoření, které neustále probíhá, a
sedm dní stvoření je sedm kvalit, které jsou ve stvoření neustále přítomny.

Rozlišuje mezi Ježíšem, který symbolizuje naši věčnou podstatu, a Kristem, který
symbolizuje spasitelský úkol boží. Že tento není závislý na fyzické osobě Ježíše,
dokládá jeho výrokem: ,,Dříve, než Abraham byl, já jsem.'' (J 8,58)

Karel Makoň má několik oblíbených pasáží z~Bible, ke kterým se často vrací. Asi
nejvýznamnějšími z~nich je podobenství o marnotratném synu a podobenství o
hřivnách. V~podobenství o marnotratném synu popisuje symbol plně rozvinutého
lidského života, kde promrhání znamená investice do pomíjejícího, a je nezbytnou
podmínkou pro vzpomínku na otcův dům, tedy uvědomění si vlastní věčné podstaty a
sjednocení bytosti kolem touhy po návratu. Otcovo ocenění syna šatem, prstenem a
zabitím telete ukazuje na fakt, že jde o vyšší a tedy žádoucí stav oproti
,,dobrému synovi'', který otcovo dědictví nepromarnil. Podobenství o hřivnách
předestírá jako návod pro životní situace obecně, kde se doporučuje spatřovat
nejen ve statcích, ale i v~situacích hřivny dané od Boha, se kterými nakládáme
nikoliv pro sebe, ale pro něho. Zúčtování, kdy hospodář přichází, aby si vzal
výtěžek z~hřiven, máme vidět ve situacích, kdy přicházíme o kontrolu nad
výsledkem svého snažení či nad situací. Moment odevzdání hřiven hospodáři bez
sebemenší pohnutky nechat si z~výtěžku něco pro sebe je klíčovým a následné udělení měst
namísto hřiven k~hospodaření je univerzálním pravidlem.

Křesťanské tradici vyčítá operování s~nevyzpytatelnou boží milostí. Bůh podle
Karla Makoně není člověk a tím méně náladový člověk, aby se mu tu něco zlíbilo a
ondy nezlíbilo nebo znelíbilo. I města za hřivny v~podobenství obdrželi služebníci nikoliv na
základě toho, jakou měl hospodář náladu, nýbrž podle míry svého hospodaření.
Stejně tak je zákonité, kdy člověka potká mystická zkušenost a vůbec cokoliv, co
tradice připisuje nevyzpytatelné boží milosti. Podmínky pro tuto dispozici
důkladně popisuje. Já jen shrnu, že stěžejním bodem je, jak to vyplývá
z~podobenství o hřivnách, vynaložení veškerých lidských sil (znásobení hřiven
bez přítomnosti hospodáře) pro nadpozemský cíl (hospodaření pro hospodáře) a
následné dokonalé odevzdání, když jsou lidské síly vyčerpány.

Dalším výrazným rysem Makoňovy nauky je, že vše podřizuje dosažení království
božího, čímž se odlišuje od mnoha moderních duchovních učitelů, kteří mnohdy
vycházejí z~lidských potřeb šťastného života, vztahů, hojnosti a podobně. Jednak
ho to připodobňuje ke klasickým katolickým mystikům a jednak (podle mne právě
proto) jeho nauka zaujme jen nepatrný zlomek lidí, kteří mají zájem o duchovno
či ezoteriku. Světské lidské problémy sice nikterak nebagatelizuje, naopak, radí
univerzální metodu pro jejich řešení v~díle {\em Zlatý klíč}, ale pro člověka,
který nehledá důsledně království boží především je v~podstatě nepřístupná.

\section{Nahrávání}

O průběhu nahrávání mám pouze kusé a anekdotické informace. Nejstarší datum, na
které jsem u nahrávky narazil, je z~roku 1970. Je možné, že některé nahrávky
jsou staršího data, ale nic nenasvědčuje tomu, že by jich bylo mnoho a byly
výrazně starší. Vzhledem k~tomu, že Karel Makoň začal evangelizovat už
v~koncentračním táboře na konci roku 1939 a přestal až v~roce 1991, je
pravděpodobné, že záznam je k~méně než polovině jeho slov.

Naprostou většinu nahrávek, které mám k~dispozici, pořídil MUDr. Vít Elger ze
Zlína. Nahrávalo se z~menší části

% získávání přepisů
%  iniciální přepis v Praatu nebo to byl Transcriber?
%  graf přibývání podle přispěvatelů, času, nahrávek atp.
% nahrávky samotné (z předchozích článků)
% grafické indexy
