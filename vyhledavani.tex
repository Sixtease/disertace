\chapter{Vyhledávání}
\label{kap:vyhledavani}

Možnost vyhledávat v~nahrávkách byl pro mne jeden z~hlavních cílů od začátku
projektu. Se~získáním přepisů náhrávek, byť kolísavé kvality, bylo možné hledání
implementovat.

Vyhledávání lze delegovat na existující vyhledávače jako Google
prostě tím, že se přepis vystaví na internetu. Není to však uspokojivé řešení,
protože z~hlediska vyhledávačů jsou přepisy nahrávek nezajímavé, takže je
vzhledem k~jejich rozsahu kompletně neindexují. Dá se to ověřit tím, že do
Googlu vložíme dotaz
\texttt{inurl:http://storage.positron.cz/makon/transkripce}, což je prostor na
CDN shodou okolností taktéž od Googlu, kam se přepisy exportují ve formátu HTML.
Dostaneme několik výsledků, každý směřující na některý z~přepisů. Vyhledávač
tedy o nahrávkách ví a má je aspoň částečně indexované.

Když však vložíme dotaz \texttt{Ta pětadvacátá si sedla, skrčila se a čekala, až
ji uklovají.}, což je věta, vyskytující se v~nahrávce s~identifikátorem
\texttt{85-05A}, žádný výsledek nedostaneme. To ilustruje, že nahrávky nejsou
kompletně indexované, byť k~nim vyhledávač má přístup.

Jak adresovat tento problém a přimět populární vyhledávače, aby přepisy
důkladněji indexovaly, neřku-li umisťovaly je na první stránky výsledků, je mimo
rozsah této disertační práce.

Mít vlastní vyhledávač je však tak jako tak u dat takového rozsahu dobré.
Implementoval jsem jej pomocí produktu Elasticsearch.

Aby bylo možné každý nalezený výsledek proměnit v~odkaz na příslušnou pasáž
v~nahrávce, zvolil jsem za jednotlivé dokumenty nikoliv celé nahrávky, nýbrž
věty.

Ke každému dokumentu se ukládá
\begin{itemize}
\item{textová reprezentace,}
\item{posloupnost fonémů,}
\item{stupeň
manuálního přepisu, tedy zda je přepis pořízen zcela automaticky, zcela manuálně
nebo kombinací obého}
\item{a také vektor confidence measure jednotlivých automaticky přepsaných slov.}
\end{itemize}

Pro skloňování je použito pravidlového stemmingu, který je dodáván s~distribucí
Elasticsearch a pro češtinu, obzvlášť toho druhu, kde se vyskytuje mnoho
nestandardních a archaických slov, funguje báječně.
