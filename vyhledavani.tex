\chapter{Vyhledávání}
\label{kap:vyhledavani}

Možnost vyhledávat v~nahrávkách byl pro mne jeden z~hlavních cílů od začátku
projektu. Se~získáním přepisů náhrávek, byť kolísavé kvality, bylo možné
vyhledávání realizovat.
Fulltextové vyhledávání jsem implementoval nástroje Elastic.

Elastic je svobodný vyhledávač napsaný v~Javě, který umožňuje fulltextové
vyhledávání v~dokumentech. Dokumenty se rozumí datové struktury, které se
vyhledávači poskytnou ve formátu JSON. Elastic má mnoho funkcionalit,
z~nichž pro mne je klíčové rozhraní na základě HTTP naplňující konvence REST,
automatický stemming, zvýrazňování nalezených pasáží a možnost vyhledávat
v~libovolných položkách dokumentu.

Aby bylo možné každý nalezený výsledek proměnit v~odkaz na příslušnou pasáž
v~nahrávce, zvolil jsem za jednotlivé dokumenty nikoliv celé nahrávky, nýbrž
věty.

Ke každému dokumentu se ukládá
\begin{itemize}
\item{textová reprezentace,}
\item{posloupnost hlásek,}
\item{stupeň
manuálního přepisu, tedy zda je přepis pořízen zcela automaticky, zcela manuálně
nebo kombinací obého}
\item{a také vektor confidence measure jednotlivých automaticky přepsaných slov.}
\end{itemize}

Pro skloňování je použito pravidlového stemmingu, který je dodáván s~distribucí
Elastic a pro češtinu, obzvlášť tam, kde se vyskytuje mnoho
nestandardních a archaických slov, funguje báječně.

Momentálně je vyhledávač nainstalován na témž serveru jako API a je dostupný
z~webové aplikace. Důležitým bodem budoucí práce je automatizace indexování
manuálních oprav, jak přicházejí. Dále pak zakomponování automatického přepisu
pořízeného bez použití jazykového modelu, jak se diskutuje
v~podsekci~\ref{ssec:data:topicsearch}.

\section{Případová studie}

Vyhledávání v~přepisech mluveného korpusu našlo využití v~kompilaci referátů o
určitých tématech, kterým se Karel Makoň věnuje. V průběhu let 2018 až 2020
vznikly alespoň čtyři takové, a to na témata
\begin{itemize}
\item{karma,}
\item{převtělování,}
\item{Otčenáš,}
\item{relativní dobro a zlo.}
\end{itemize}
Každé téma bylo zpracováno do formy souboru krátkých úseků nahrávek, které se
prezentovaly sekvenčním přehráním s~podporou přepisů jako zrakového vodítka.

Autor referátu o tématu relativního dobra a zla dohledal poznámkový aparát
k~tvorbě a rekonstruoval svůj postup, který zde popíšu jako příklad použití
přepisů, z~něhož lze usoudit na efektivitu práce.

Téma relativního dobra a zla bylo předem zamyšleno a bylo vybráno pro autorův
zájem a nikoliv s~ohledem na to, jak snadné bude pro vyhledání. Dopředu byl dán
časový rámec výsledku na cca. dvě hodiny zvukového záznamu.

Autorova metodika byla následovná: vyhledal frázi ,,relativní dobro a zlo`` a
výsledky procházel ve výchozím relevančním řazení nástroje Elastic. Prošel
prvních sto z~celkových 7379 výsledků. U každého posoudil, zda se jedná o
pasáž, skutečně o tématu pojednávající, nebo jen o letmou zmínku či vůbec
falešný zásah, popřípadě o duplicitní výskyt.

Výběr vzorků probíhal ve dvou průchodech. V~prvním autor vybíral relevantní a
relativně obsáhlejší zásahy, čímž se tvořila užší množina kandidátských úseků.
V~druhém průchodu pak vybíral z~této užší množiny s~ohledem na ročník zdrojové
nahrávky, aby byl v~kompilátu zastoupen průřez vývoje Makoňova myšlení,
výjimečně podle návaznosti či pro závěrečnou část shrnující charakter výpovědi.

Po prvním průchodu se do užšího výběru dostalo 25 nalezených úseků, tedy
čtvrtina procházené množiny, a všechny byly z~prvních 53 zásahů. V~první stovce
výsledků vyhledávání autor identifikoval 16 duplicit.
