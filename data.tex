\chapter{Data}
\label{kap:data}

Zvukový odkaz Karla Makoně je původní a ústřední motivací pro tuto práci. Osobně
považuji Makoňovo dílo za jedno z~nejzásadnějších vůbec v~oblasti duchovního
průkopnictví, a to jeho systematičností, obsáhlostí, návodností, novátorstvím a
především hloubkou. Jeho nauka se od moderních duchovních směrů odlišuje kladným
postojem k~civilizačnímim trendům, nikoliv jejich zavrhováním, dále
konzistentním souladem s~rozumovým poznáním a pevnými základy v~náboženských
tradicích. Od vědeckého bádání se odlišuje zejména tím, že rozum a hmotu
považuje za odrazové můstky k~hlubšímu poznání, nikoliv za vrchol a jedinou
platformu lidského poznání. Trvá však na tom, že duchovní zákonitosti jsou
stejně tak pevně dané, univerzální a ověřitelné (ovšem pouze osobní, subjektivní
zkušeností), jako zákony přírodní, popsané vědecky. Do třetice od klasické
křesťanské literatury se liší obzvláště tím, že Ježíšovu nauku považuje za návod
prvotřídní kvality pro vědomý vstup do věčného života zde na zemi a v~těle,
nikoliv po smrti. Věčným životem se míní stav, kdy člověk je vědomě věčnou
bytostí nezávislou na pomíjejícím těle. Tvrdě kritizuje překonaný a naivní
výklad, podle nějž se ctnostným životem dá dojít po smrti do nebe, dále dojít
spásy pouhou proklamací o~víře v~Krista a dodržováním přikázání a náboženských
obřadů.

Odkrývá smysl života a návod na jeho uskutečnění, který nestojí na slepé víře
ani na vlasní omezené lidské invenci.

\section{Karel Makoň}

\subsection{Duchovní vzestup Karla Makoně}

Ing. Karel Makoň se narodil 12. prosince 1912. Ve věku dvou let ho postihl zánět
levého ramene. Lékaři doporučovali amputaci ruky, k~čemuž jeho matka nedala
souhlas a na vlastní zodpovědnost nechala dítě operovat. Vzhledem k~tomu, že
ještě nebyly objeveny krevní skupiny, nebyla možná transfúze a proto musela být
operace prováděna opakovaně, aby dítě nevykrvácelo. Tehdejší anestetika nebylo
možné tak často podat tak mladému organizmu, proto byly operace prováděny při
vědomí. Malý Karel Makoň se zažívaje nesnesitelnou bolest naučil v~raném věku
opouštět svoje tělo. Tato opakovaná zkušenost měla u něho následek, že po
určitou dobu nepoznával svoji matku, zato začal spontánně rozpoznávat správné od
nesprávného a důsledně činit, co poznával jako správné.

Pro omezení rizika komplikací s~operovanou rukou bylo Karlovi zakázáno hrát si
s~dětmi. Svůj předškolní čas proto trávil sám na venkově jen se zvířaty. Díky
extrakorporálním zkušenostem se naučil rozumět řeči zvířat, obzvláště hus, které
ho vzaly za svého a s~nimiž pronikal do stavu zvířecího ráje.

Období ,,činění správného'', kdy si kupříkladu zapověděl kouření, alkohol i
veškerý pohlavní život, vyvrcholilo v~Makoňových sedmnácti letech, kdy si
přečetl myšlenku, že ,,tento život je mostem do věčnosti''. Tím započalo období
extází a vědomí, že je nesmrtelnou bytostí a smyslem jeho života je spojení
s~Bohem. Svojí matkou a prarodiči byl sice veden ke tradiční katolické víře, ale
nikdy na ni nepřistoupil, protože ,,v~nebi, kde by se jen díval na Boží tvář by
byla strašná nuda''. Nikdy tedy nevěřil, v~sedmnácti letech poznal.

V~tomto období se ustavičně modlil za to, aby dokázal Boha více milovat. Tato
modlitba trvala devět let a jejím vyvrcholením byla deportace do koncentračního
tábora v~Sachsenhausen v~roce 1939, coby českého vysokoškolského studenta.

V~koncentračním táboře byl Makoň sužován více, než ostatní: měl jakýsi
obzvláštní talent chytat rány a kopance. Prožíval nesmírné zmatení a frustraci
nad tím, že tak dlouho tak věrně sloužil Bohu, a teď se s~ním jedná jako s~kusem
hadru. Po čtyřech dnech utrpení nastal zlomový okamžik. Tamějším vězňům bylo
zakázáno pod trestem smrti přihlížet zabití spoluvězňů příslušníky SS. Makoň si
však nedal pozor a hleděl na právě takovou scénu. Vykonávající Němec si toho
povšiml a vyzval Karla Makoně, ať zůstane stát na místě, že hned, jak dobije
svoji momentální oběť, přijde zabít i jeho. Karel Makoň v~tu chvíli raději
odevzdal svůj život Bohu, a to bez přemýšlení a bezpodmínečně, se silou, nabytou
onou devítiletou modlitbou. Překvapivým výsledkem toho bylo, že SS-Mann tváří
v~tvář Makoňovi zbledl a v~hrůze se obrátil na útěk.

Makoň tehdy obdržel všeobjímající poznání smyslu života, jak svého, tak obecně,
a absolutní svobodu. Pohyboval se volně od baráku k~baráku, nezažíval hlad ani
jiný nedostatek, esesáci ho neviděli. Trávil svoje dny v~koncentračním táboře
burcováním ostatních k~probuzení k~pravdě, kterou sám zažíval.

Zanedlouho byl z~koncentračního tábora propuštěn a celý zbytek svého života
věnoval předávání svojí zkušenosti a hlavně návodu, jak k~ní přijít bez nutnosti
tak výrazných krizí, jakými sám v~životě procházel, a které považuje za
nepříkladné.

Zemřel v~roce 1993.

\subsection{Spisovatelská a přednášková činnost}

% získávání přepisů
%  iniciální přepis v Praatu nebo to byl Transcriber?
%  graf přibývání podle přispěvatelů, času, nahrávek atp.
% nahrávky samotné (z předchozích článků)
% grafické indexy
